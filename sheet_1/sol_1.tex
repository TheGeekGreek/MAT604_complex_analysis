%%%%%%%%%%%%%%%%%%%%%%%%%%%%%%%%%%%%%%%%%%%%%%%%%%%%%%%%%%%%%%%%%%%%%%%%%%
%Author:																 %
%-------																 %
%Yannis Baehni at University of Zurich									 %
%baehni.yannis@uzh.ch													 %
%																		 %
%Version log:															 %
%------------															 %
%06/02/16 . Basic structure												 %
%04/08/16 . Layout changes including section, contents, abstract.		 %
%%%%%%%%%%%%%%%%%%%%%%%%%%%%%%%%%%%%%%%%%%%%%%%%%%%%%%%%%%%%%%%%%%%%%%%%%%

%Page Setup
\documentclass[
	11pt, 
	oneside, 
	a4paper,
	reqno,
	final
]{amsart}

\usepackage[
	left = 3cm, 
	right = 3cm, 
	top = 3cm, 
	bottom = 3cm
]{geometry}

%Headers and footers
\usepackage{fancyhdr}
	\pagestyle{fancy}
	%Clear fields
	\fancyhf{}
	%Header right
	\fancyhead[R]{
		\footnotesize
		Yannis B\"{a}hni\\
		\href{mailto:yannis.baehni@uzh.ch}{yannis.baehni@uzh.ch}
	}
	%Header left
	\fancyhead[L]{
		\footnotesize
		MAT604: Complex Analysis\\
		Spring Semester 2017
	}
	%Page numbering in footer
	\fancyfoot[C]{\thepage}
	%Separation line header and footer
	\renewcommand{\headrulewidth}{0.4pt}
	%\renewcommand{\footrulewidth}{0.4pt}
	
	\setlength{\headheight}{19pt} 

%Title
\usepackage[foot]{amsaddr}
%\usepackage{mathptmx}
\usepackage{xspace}
\makeatletter
\def\@textbottom{\vskip \z@ \@plus 1pt}
\let\@texttop\relax
\usepackage{etoolbox}
\patchcmd{\abstract}{\scshape\abstractname}{\textbf{\abstractname}}{}{}

%Switching commands for different section formats
%Mainsectionsytle
\newcommand{\mainsectionstyle}{%
  	\renewcommand{\@secnumfont}{\bfseries}
  	\renewcommand\section{\@startsection{section}{1}%
    	\z@{.5\linespacing\@plus.7\linespacing}{-.5em}%
    	{\normalfont\bfseries}}%
	\renewcommand\subsection{\@startsection{subsection}{2}%
    	\z@{.5\linespacing\@plus.7\linespacing}{-.5em}%
    	{\normalfont\bfseries}}%
	\renewcommand\subsubsection{\@startsection{subsubsection}{3}%
    	\z@{.5\linespacing\@plus.7\linespacing}{-.5em}%
    	{\normalfont\bfseries}}%
}
\newcommand{\originalsectionstyle}{%
\def\@secnumfont{\bfseries}%\mdseries
\def\section{\@startsection{section}{1}%
  \z@{.7\linespacing\@plus\linespacing}{.5\linespacing}%
  {\normalfont\bfseries\centering}}
}
%Formatting title of TOC
\renewcommand{\contentsnamefont}{\bfseries}
%Table of Contents
\setcounter{tocdepth}{3}

% Add bold to \section titles in ToC and remove . after numbers
\renewcommand{\tocsection}[3]{%
  \indentlabel{\@ifnotempty{#2}{\bfseries\ignorespaces#1 #2\quad}}\bfseries#3}
% Remove . after numbers in \subsection
\renewcommand{\tocsubsection}[3]{%
  \indentlabel{\@ifnotempty{#2}{\ignorespaces#1 #2\quad}}#3}
\let\tocsubsubsection\tocsubsection% Update for \subsubsection
%...

\newcommand\@dotsep{4.5}
\def\@tocline#1#2#3#4#5#6#7{\relax
  \ifnum #1>\c@tocdepth % then omit
  \else
    \par \addpenalty\@secpenalty\addvspace{#2}%
    \begingroup \hyphenpenalty\@M
    \@ifempty{#4}{%
      \@tempdima\csname r@tocindent\number#1\endcsname\relax
    }{%
      \@tempdima#4\relax
    }%
    \parindent\z@ \leftskip#3\relax \advance\leftskip\@tempdima\relax
    \rightskip\@pnumwidth plus1em \parfillskip-\@pnumwidth
    #5\leavevmode\hskip-\@tempdima{#6}\nobreak
    \leaders\hbox{$\m@th\mkern \@dotsep mu\hbox{.}\mkern \@dotsep mu$}\hfill
    \nobreak
    \hbox to\@pnumwidth{\@tocpagenum{\ifnum#1=1\bfseries\fi#7}}\par% <-- \bfseries for \section page
    \nobreak
    \endgroup
  \fi}
\AtBeginDocument{%
\expandafter\renewcommand\csname r@tocindent0\endcsname{0pt}
}
\def\l@subsection{\@tocline{2}{0pt}{2.5pc}{5pc}{}}
\def\l@subsubsection{\@tocline{2}{0pt}{4.5pc}{5pc}{}}
\makeatother

\advance\footskip0.4cm
\textheight=54pc    %a4paper
\textheight=50.5pc %letterpaper
\advance\textheight-0.4cm
\calclayout

%Font settings
%\usepackage{anyfontsize}
%Footnote settings
%\usepackage{mathptmx}
\usepackage{footmisc}
%	\renewcommand*{\thefootnote}{\fnsymbol{footnote}}
\usepackage{commath}
%Further math environments
%Further math fonts (loads amsfonts implicitely)
\usepackage{amssymb}
%Redefinition of \text
%\usepackage{amstext}
\usepackage{upref}
%Graphics
%\usepackage{graphicx}
%\usepackage{caption}
%\usepackage{subcaption}
%Frames
\usepackage{mdframed}
\allowdisplaybreaks
%\usepackage{interval}
\newcommand{\toup}{%
  \mathrel{\nonscript\mkern-1.2mu\mkern1.2mu{\uparrow}}%
}
\newcommand{\todown}{%
  \mathrel{\nonscript\mkern-1.2mu\mkern1.2mu{\downarrow}}%
}
\AtBeginDocument{\renewcommand*\d{\mathop{}\!\mathrm{d}}}
\renewcommand{\Re}{\operatorname{Re}}
\renewcommand{\Im}{\operatorname{Im}}
\DeclareMathOperator\Log{Log}
\DeclareMathOperator\Arg{Arg}
\DeclareMathOperator\sech{sech}
\DeclareMathOperator*\esssup{ess.sup}
%\usepackage{hhline}
%\usepackage{booktabs} 
%\usepackage{array}
%\usepackage{xfrac} 
%\everymath{\displaystyle}
%Enumerate
\usepackage{tikz}
\usetikzlibrary{external}
\tikzexternalize % activate!
\usetikzlibrary{patterns}
\pgfdeclarepatternformonly{adjusted lines}{\pgfqpoint{-1pt}{-1pt}}{\pgfqpoint{40pt}{40pt}}{\pgfqpoint{39pt}{39pt}}%
{
  \pgfsetlinewidth{.8pt}
  \pgfpathmoveto{\pgfqpoint{0pt}{0pt}}
  \pgfpathlineto{\pgfqpoint{39.1pt}{39.1pt}}
  \pgfusepath{stroke}
}
\usepackage{enumitem} 
%\renewcommand{\labelitemi}{$\bullet$}
%\renewcommand{\labelitemii}{$\ast$}
%\renewcommand{\labelitemiii}{$\cdot$}
%\renewcommand{\labelitemiv}{$\circ$}
%Colors
%\usepackage{color}
%\usepackage[cmtip, all]{xy}
%Theorems
\newtheoremstyle{bold}              	 %Name
  {}                                     %Space above
  {}                                     %Space below
  {\itshape}		                     %Body font
  {}                                     %Indent amount
  {\scshape}                             %Theorem head font
  {.}                                    %Punctuation after theorem head
  { }                                    %Space after theorem head, ' ', 
  										 %	or \newline
  {} 
\theoremstyle{bold}
\newtheorem*{definition*}{Definition}
\newtheorem{definition}{Definition}[section]
\newtheorem*{lemma*}{Lemma}
\newtheorem{lemma}{Lemma}[section]
\newtheorem{Proof}{Proof}[section]
\newtheorem{proposition}{Proposition}[section]
\newtheorem{properties}{Properties}[section]
\newtheorem{corollary}{Corollary}[section]
\newtheorem*{theorem*}{Theorem}
\newtheorem{theorem}{Theorem}[section]
\newtheorem{example}{Example}[section]
\newtheorem*{remark*}{Remark}
\newtheorem{remark}{Remark}[section]
%German non-ASCII-Characters
%Graphics-Tool
%\usepackage{tikz}
%\usepackage{tikzscale}
%\usepackage{bbm}
%\usepackage{bera}
%Listing-Setup
%Bibliographie
\usepackage[backend=bibtex, style=alphabetic]{biblatex}
%\usepackage[babel, german = swiss]{csquotes}
\bibliography{Bibliography}
%PDF-Linking
%\usepackage[hyphens]{url}
\usepackage[bookmarksopen=true,bookmarksnumbered=true]{hyperref}
%\PassOptionsToPackage{hyphens}{url}\usepackage{hyperref}
\hypersetup{
  colorlinks   = true, %Colours links instead of ugly boxes
  urlcolor     = blue, %Colour for external hyperlinks
  linkcolor    = blue, %Colour of internal links
  citecolor    = blue %Colour of citations
}
%Weierstrass-P symbol for power set
\newcommand{\powerset}{\raisebox{.15\baselineskip}{\Large\ensuremath{\wp}}}


\title{Solutions Sheet 1}
\author{Yannis B\"{a}hni}
\address[Yannis B\"{a}hni]{University of Zurich, R\"{a}mistrasse 71, 8006 Zurich}
\email[Yannis B\"{a}hni]{\href{mailto:yannis.baehni@uzh.ch}{yannis.baehni@uzh.ch}}

\begin{document}
\maketitle
\thispagestyle{fancy}
\begin{enumerate}[label = \textbf{Exercise \arabic*.},wide = 0pt, itemsep=1.5ex]
\item
	~ 
	\begin{enumerate}[label = \alph*),wide = 0pt, itemsep=1.5ex]
		\item It is well known, that the set $M_2(\mathbb{R})$ is a ring with identity. It is also clear, that the given set together with the usual operations constitutes a subring with identity of $M_2(\mathbb{R})$. Therefore it is enough to show commutativity and the existence of inverse elements regarding multiplication. Let $x,y,x',y' \in \mathbb{R}$. Then we have
\begin{align*}
	\begin{pmatrix}
		x & y\\
		-y & x
	\end{pmatrix}
	\begin{pmatrix}
		x' & y'\\
		-y' & x'
	\end{pmatrix}
	&= \begin{pmatrix}
		xx' -yy' & xy'+yx'\\
		-yx'-xy' & -yy'+xx'
	\end{pmatrix}\\
	&= \begin{pmatrix}
		x'x-y'y & x'y +y'x\\
		-y'x-x'y & -y'y+x'x
	\end{pmatrix}\\
	&= \begin{pmatrix}
		x' & y'\\
		-y' & x'
	\end{pmatrix}
	\begin{pmatrix}
		x & y\\
		-y & x
	\end{pmatrix}
\end{align*}

Since $\mathbb{R}$ is a field. Furthermore we have
\begin{equation}
	\det\begin{pmatrix}
		x & y\\
		-y & x
	\end{pmatrix} = x^2 + y^2
\end{equation}

Hence $\begin{pmatrix}
		x & y\\
		-y & x
	\end{pmatrix}$ is invertible if and only if $(x,y) \neq (0,0)$, which means that every non-zero element is invertible. Thus the set constitutes a field under the given operations. Now define a mapping $\iota$ by
	\begin{equation}
	\iota(x + iy) := \begin{pmatrix}
		x & y\\
		-y & x
	\end{pmatrix}
	\label{eq:iso}
	\end{equation}

	We have 
	\begin{equation*}
		\iota((x + iy) + (u + iv)) = \begin{pmatrix}
			x + u & y + v\\
			-y-v & x + u
		\end{pmatrix} = \begin{pmatrix}
		x & y\\
		-y & x
	\end{pmatrix} + \begin{pmatrix}
		u & v\\
		-v & u
	\end{pmatrix} = \iota(x + iy) + \iota(u + iv)
	\end{equation*}

	\noindent and 
	\begin{equation*}
		\iota((x + iy)(u +iv)) = \begin{pmatrix}
		xu - yv & xv + uy\\
		-xv-uy & xu-yv
	\end{pmatrix} = \begin{pmatrix}
		x & y\\
		-y & x
	\end{pmatrix}\begin{pmatrix}
		u & v\\
		-v & u
	\end{pmatrix}
	= \iota(x + iy)\iota(u +iv).
	\end{equation*}

	Obviously $\ker(\iota) = \cbr[0]{0}$ and $\iota$ is surjective, hence $\iota$ is an isomorphism of fields.

\item Consider the abelian group $(\mathbb{C},\cdot)$. We show $(S^1, \cdot) \leq (\mathbb{C},\cdot)$, where
	\begin{equation*}
		S^1 := \partial\mathbb{E} = \cbr[0]{z \in \mathbb{C} : \abs[0]{z} = 1}.
	\end{equation*}
	Clearly $1 \in S^1$. If $z,z' \in S^1$, we have $\abs[0]{z} = \abs[0]{z'} = 1$ and therefore $\abs[0]{zz'} = \abs[0]{z}\abs[0]{z'} = 1$ which implies $zz' \in S^1$. Also we have $\abs[0]{1/z} = 1/\abs[0]{z} = 1$ for $z \in S^1$ which implies $1/z \in S^1$. With the terminology established in (\ref{eq:iso}) we consider the restriction
	\begin{equation}
		\iota\vert_{S^1}: S^1 \to \iota(S^1)
	\end{equation}

	\noindent which is an isomorphism of groups. We will show that 
	\begin{equation}
		\iota(S^1) = \operatorname{SO}(2).
	\end{equation}

	Let $x + iy \in S^1$. Then $1 = \abs[0]{x + iy}^2 = x^2 + y^2$ and so
	\begin{equation*}
		\iota(x + iy)(\iota(x + iy))^t = \begin{pmatrix}
		x & y\\
		-y & x
	\end{pmatrix}\begin{pmatrix}
		x & -y\\
		y & x
	\end{pmatrix} = \begin{pmatrix}
		x^2 + y^2 & -xy + yx\\
		-yx + xy & x^2 + y^2
	\end{pmatrix} = \begin{pmatrix}
		1 & 0\\
		0 & 1
	\end{pmatrix}.
	\end{equation*}

	Since $\iota(S^1)$ is a subgroup of the abelian group in a), we have also that $(\iota(x + iy))^t\iota(x + iy)$ is the identity matrix. Also $\det(\iota(x + iy)) = x^2 + y^2 = 1$ so $\iota(S^1) \subseteq \operatorname{SO}(2)$. Now we have 
	\begin{equation*}
		\operatorname{SO}(2) = \cbr[3]{\begin{pmatrix}
		\cos \varphi & -\sin \varphi\\
		\sin\varphi & \cos \varphi
	\end{pmatrix} : \varphi \in \mathbb{R}}
	\end{equation*}

	By 
	\begin{equation*}
		\begin{pmatrix}
		\cos \varphi & -\sin \varphi\\
		\sin\varphi & \cos \varphi
	\end{pmatrix} = \begin{pmatrix}
		\cos (-\varphi) & \sin (-\varphi)\\
		-\sin(-\varphi) & \cos (-\varphi)
	\end{pmatrix}
	\end{equation*}

	\noindent we have 
	\begin{equation*}
		\iota(\cos(-\varphi) + i\sin(-\varphi)) = \begin{pmatrix}
		\cos (-\varphi) & \sin (-\varphi)\\
		-\sin(-\varphi) & \cos (-\varphi)
	\end{pmatrix}
	\end{equation*}

	\noindent and with
	\begin{equation*}
		\abs[0]{\cos(-\varphi) + i\sin(-\varphi)}^2 = \cos^2(-\varphi) + \sin^2(-\varphi) = 1 
	\end{equation*}

	\noindent this implies $\operatorname{SO}(2) \subseteq \iota(S^1)$.
	\end{enumerate}
\item See separate sheet.
\item Clearly $\varnothing \in \mathcal{Z}$ and $\mathbb{C} \in \mathcal{Z}$ since $\mathbb{C}^c = \varnothing$ which is finite. Let $U,V \in \mathcal{Z}$. Then $U^c$ and $V^c$ are both finite and so is
	\begin{equation*}
		(U \cap V)^c = U^c \cup V^c.
	\end{equation*}

	Hence $U \cap V \in \mathcal{Z}$. Let $(U_\alpha)_{\alpha \in A}$ be a sequence in $\mathcal{Z}$. Then we have that $U_\alpha^c$ is finite for any $\alpha \in A$. Therefore by
\begin{equation*}
	\del[4]{\bigcup_{\alpha \in A} U_\alpha}^c = \bigcap_{\alpha \in A} U_\alpha^c \subseteq U_{\beta}^c
\end{equation*}
for any $\beta \in A$ we have that $\bigcup_{\alpha \in A}U_\alpha \in \mathcal{Z}$. Hence $(\mathbb{C},\mathcal{Z})$ is a topological space. We show the following result:
\begin{lemma}
	Let $X$ be an infinite set. Then $(X,\mathcal{Z})$ is not Hausdorff.
\end{lemma}

\begin{proof}
	Towards a contradiction assume that $(X,\mathcal{Z})$ is Hausdorff. Hence for any $p,q \in X$ we find (open) neighbourhoods $U$ of $p$ and $V$ of $q$ such that $U \cap V = \varnothing$. Thus $U^c$ is finite and since $U \cap V = \varnothing$ we have that $U \subseteq V^c$ and thus $U$ is finite. But this would imply that 
	\begin{equation*}
		X = U \cup U^c
	\end{equation*}
	\noindent is the union of finite sets which would mean that $X$ itself is finite. Contradiction.
\end{proof}

\item We use the terminology established in \cite[8--16]{fischer2005funktionentheorie} which results in considering a function $f: M \to \mathbb{C}$. We show the implications (i) $\Rightarrow$ (ii) $\Rightarrow$ (iii) $\Rightarrow$ (i).\\
	Assume $f$ is continuous in $z^* \in M$ and fix $\varepsilon > 0$. Now consider the set $B_\varepsilon(f(z^*))$ which is a neighbourhood of $f(z^*)$. By (i) there exists a neighbourhood $U$ of $z^*$ such that $f(U \cap M) \subseteq B_\varepsilon(f(z^*))$. Since $U$ is a neighbourhood of $z^*$ it contains a $\delta$-neighbourhood $B_\delta(z^*)$ of $z^*$.\\
	Assume that (ii) holds. Let $(z_\nu)_{\nu \in \mathbb{N}}$ be a sequence in $M$ such that $z_\nu \to z^*$ and $U$ be any neighbourhood of $f(z^*)$. Since $U$ is a neighbourhood of $f(z^*)$ it contains a $\varepsilon$-neighbourhood $B_\varepsilon(f(z^*))$. By (ii) we find $\delta > 0$ such that $z \in B_\delta(z^*)$ implies $f(z) \in B_\varepsilon(f(z^*))$. Since $z_\nu \to z^*$ we also have $z_\nu \in B_\delta(z^*)$ for almost all $z_\nu$. In conclusion, $f(z_\nu) \in U$ for almost all $f(z_\nu)$.\\ 
	Assume that (iii) holds. Towards a contradiction assume that (i) does not hold. Hence there exists a neighbourhood $V$ of $f(z^*)$ such that for any neighbourhood $U$ of $z^*$ we have $f(U \cap M) \not\subseteq V$. The latter means that $z \in f(U \cap M)$ but $z \notin V$ for some $z$. Consider the sets $B_{1/\nu}(z^*)$ for $\nu \in \mathbb{N}_{>0}$. They are clearly neighbourhoods of $z^*$. Now by assumption, for any $\nu \in \mathbb{N}_{>0}$ there is some $f(z_\nu) \in f(B_{1/\nu}(z^*) \cap M)$ which is not in $V$. This defines a sequence $(z_\nu)_{n \in \mathbb{N}_{>0}}$ in $M$. We claim that $z_\nu \to z^*$. Indeed, if we have any neighbourhood $U$ of $z^*$ we find by definition an $\varepsilon$-neighbourhood $B_\varepsilon(z^*) \subseteq U$. But by the archimidean principle we have  $1/\nu < \varepsilon$ for $\nu$ small enough, thus $z_\nu \in B_\varepsilon(z^*)$ for almost all $\nu$. Now $z_\nu \to z^*$ but clearly $f(z_\nu) \not\to f(z^*)$ since none of the $z_\nu$ is in $V$. Contradiction.  
\end{enumerate}
%\originalsectionstyle
\printbibliography
\end{document}
