%%%%%%%%%%%%%%%%%%%%%%%%%%%%%%%%%%%%%%%%%%%%%%%%%%%%%%%%%%%%%%%%%%%%%%%%%%
%Author:																 %
%-------																 %
%Yannis Baehni at University of Zurich									 %
%baehni.yannis@uzh.ch													 %
%																		 %
%Version log:															 %
%------------															 %
%06/02/16 . Basic structure												 %
%04/08/16 . Layout changes including section, contents, abstract.		 %
%%%%%%%%%%%%%%%%%%%%%%%%%%%%%%%%%%%%%%%%%%%%%%%%%%%%%%%%%%%%%%%%%%%%%%%%%%

%Page Setup
\documentclass[
	11pt, 
	oneside, 
	a4paper,
	reqno,
	final
]{amsart}

\usepackage[
	left = 3cm, 
	right = 3cm, 
	top = 3cm, 
	bottom = 3cm
]{geometry}

%Headers and footers
\usepackage{fancyhdr}
	\pagestyle{fancy}
	%Clear fields
	\fancyhf{}
	%Header right
	\fancyhead[R]{
		\footnotesize
		Yannis B\"{a}hni\\
		\href{mailto:yannis.baehni@uzh.ch}{yannis.baehni@uzh.ch}
	}
	%Header left
	\fancyhead[L]{
		\footnotesize
		MAT604: Complex Analysis\\
		Spring Semester 2017
	}
	%Page numbering in footer
	\fancyfoot[C]{\thepage}
	%Separation line header and footer
	\renewcommand{\headrulewidth}{0.4pt}
	%\renewcommand{\footrulewidth}{0.4pt}
	
	\setlength{\headheight}{19pt} 

%Title
\usepackage[foot]{amsaddr}
\usepackage{upref}
%\usepackage{mathptmx}
\usepackage{xspace}
\makeatletter
\def\@textbottom{\vskip \z@ \@plus 1pt}
\let\@texttop\relax
\usepackage{etoolbox}
\patchcmd{\abstract}{\scshape\abstractname}{\textbf{\abstractname}}{}{}

\usepackage[all,cmtip]{xy}

%Switching commands for different section formats
%Mainsectionsytle
\newcommand{\mainsectionstyle}{%
  	\renewcommand{\@secnumfont}{\bfseries}
  	\renewcommand\section{\@startsection{section}{1}%
    	\z@{.5\linespacing\@plus.7\linespacing}{-.5em}%
    	{\normalfont\bfseries}}%
	\renewcommand\subsection{\@startsection{subsection}{2}%
    	\z@{.5\linespacing\@plus.7\linespacing}{-.5em}%
    	{\normalfont\bfseries}}%
	\renewcommand\subsubsection{\@startsection{subsubsection}{3}%
    	\z@{.5\linespacing\@plus.7\linespacing}{-.5em}%
    	{\normalfont\bfseries}}%
}
\newcommand{\originalsectionstyle}{%
\def\@secnumfont{\bfseries}%\mdseries
\def\section{\@startsection{section}{1}%
  \z@{.7\linespacing\@plus\linespacing}{.5\linespacing}%
  {\normalfont\bfseries\centering}}
}
%Formatting title of TOC
\renewcommand{\contentsnamefont}{\bfseries}
%Table of Contents
\setcounter{tocdepth}{3}

% Add bold to \section titles in ToC and remove . after numbers
\renewcommand{\tocsection}[3]{%
  \indentlabel{\@ifnotempty{#2}{\bfseries\ignorespaces#1 #2\quad}}\bfseries#3}
% Remove . after numbers in \subsection
\renewcommand{\tocsubsection}[3]{%
  \indentlabel{\@ifnotempty{#2}{\ignorespaces#1 #2\quad}}#3}
\let\tocsubsubsection\tocsubsection% Update for \subsubsection
%...

\newcommand\@dotsep{4.5}
\def\@tocline#1#2#3#4#5#6#7{\relax
  \ifnum #1>\c@tocdepth % then omit
  \else
    \par \addpenalty\@secpenalty\addvspace{#2}%
    \begingroup \hyphenpenalty\@M
    \@ifempty{#4}{%
      \@tempdima\csname r@tocindent\number#1\endcsname\relax
    }{%
      \@tempdima#4\relax
    }%
    \parindent\z@ \leftskip#3\relax \advance\leftskip\@tempdima\relax
    \rightskip\@pnumwidth plus1em \parfillskip-\@pnumwidth
    #5\leavevmode\hskip-\@tempdima{#6}\nobreak
    \leaders\hbox{$\m@th\mkern \@dotsep mu\hbox{.}\mkern \@dotsep mu$}\hfill
    \nobreak
    \hbox to\@pnumwidth{\@tocpagenum{\ifnum#1=1\bfseries\fi#7}}\par% <-- \bfseries for \section page
    \nobreak
    \endgroup
  \fi}
\AtBeginDocument{%
\expandafter\renewcommand\csname r@tocindent0\endcsname{0pt}
}
\def\l@subsection{\@tocline{2}{0pt}{2.5pc}{5pc}{}}
\def\l@subsubsection{\@tocline{2}{0pt}{4.5pc}{5pc}{}}
\makeatother

\advance\footskip0.4cm
\textheight=54pc    %a4paper
\textheight=50.5pc %letterpaper
\advance\textheight-0.4cm
\calclayout

%Font settings
%\usepackage{anyfontsize}
%Footnote settings
%\usepackage{mathptmx}
\usepackage{footmisc}
%	\renewcommand*{\thefootnote}{\fnsymbol{footnote}}
\usepackage{commath}
%Further math environments
%Further math fonts (loads amsfonts implicitely)
\usepackage{amssymb}
%Redefinition of \text
%\usepackage{amstext}
\usepackage{upref}
%Graphics
%\usepackage{graphicx}
%\usepackage{caption}
%\usepackage{subcaption}
%Frames
\usepackage{mdframed}
\allowdisplaybreaks
%\usepackage{interval}
\newcommand{\toup}{%
  \mathrel{\nonscript\mkern-1.2mu\mkern1.2mu{\uparrow}}%
}
\newcommand{\todown}{%
  \mathrel{\nonscript\mkern-1.2mu\mkern1.2mu{\downarrow}}%
}
\AtBeginDocument{\renewcommand*\d{\mathop{}\!\mathrm{d}}}
\renewcommand{\Re}{\operatorname{Re}}
\renewcommand{\Im}{\operatorname{Im}}
\DeclareMathOperator\Log{Log}
\DeclareMathOperator\Arg{Arg}
\DeclareMathOperator\sech{sech}
\DeclareMathOperator*\esssup{ess.sup}
\DeclareMathOperator\id{id}
%\usepackage{hhline}
%\usepackage{booktabs} 
%\usepackage{array}
%\usepackage{xfrac} 
%\everymath{\displaystyle}
%Enumerate
\usepackage{tikz}
\usetikzlibrary{external}
\tikzexternalize % activate!
\usetikzlibrary{patterns}
\pgfdeclarepatternformonly{adjusted lines}{\pgfqpoint{-1pt}{-1pt}}{\pgfqpoint{40pt}{40pt}}{\pgfqpoint{39pt}{39pt}}%
{
  \pgfsetlinewidth{.8pt}
  \pgfpathmoveto{\pgfqpoint{0pt}{0pt}}
  \pgfpathlineto{\pgfqpoint{39.1pt}{39.1pt}}
  \pgfusepath{stroke}
}
\usepackage{enumitem} 
%\renewcommand{\labelitemi}{$\bullet$}
%\renewcommand{\labelitemii}{$\ast$}
%\renewcommand{\labelitemiii}{$\cdot$}
%\renewcommand{\labelitemiv}{$\circ$}
%Colors
%\usepackage{color}
%\usepackage[cmtip, all]{xy}
%Theorems
\newtheoremstyle{bold}              	 %Name
  {}                                     %Space above
  {}                                     %Space below
  {\itshape}		                     %Body font
  {}                                     %Indent amount
  {\bfseries}                             %Theorem head font
  {.}                                    %Punctuation after theorem head
  { }                                    %Space after theorem head, ' ', 
  										 %	or \newline
  {\thmname{#1}\thmnumber{ #2}\thmnote{ (#3)}} 
\theoremstyle{bold}
\newtheorem*{definition*}{Definition}
\newtheorem{definition}{Definition}[section]
\newtheorem*{lemma*}{Lemma}
\newtheorem{lemma}{Lemma}[section]
\newtheorem{Proof}{Proof}[section]
\newtheorem{proposition}{Proposition}[section]
\newtheorem{properties}{Properties}[section]
\newtheorem{corollary}{Corollary}[section]
\newtheorem*{theorem*}{Theorem}
\newtheorem{theorem}{Theorem}[section]
\newtheorem{example}{Example}[section]
\newtheorem*{remark*}{Remark}
\newtheorem{remark}{Remark}[section]
%German non-ASCII-Characters
%Graphics-Tool
%\usepackage{tikz}
%\usepackage{tikzscale}
%\usepackage{bbm}
%\usepackage{bera}
%Listing-Setup
%Bibliographie
\usepackage[backend=bibtex, style=alphabetic]{biblatex}
%\usepackage[babel, german = swiss]{csquotes}
\bibliography{Bibliography}
%PDF-Linking
%\usepackage[hyphens]{url}
\usepackage[bookmarksopen=true,bookmarksnumbered=true]{hyperref}
%\PassOptionsToPackage{hyphens}{url}\usepackage{hyperref}
\hypersetup{
  colorlinks   = true, %Colours links instead of ugly boxes
  urlcolor     = blue, %Colour for external hyperlinks
  linkcolor    = blue, %Colour of internal links
  citecolor    = blue %Colour of citations
}
%Weierstrass-P symbol for power set
\newcommand{\powerset}{\raisebox{.15\baselineskip}{\Large\ensuremath{\wp}}}
\newcommand{\bld}[1]{\boldmath\textit{\textbf{#1}}\unboldmath}


\title{Solutions Sheet 3}
\author{Yannis B\"{a}hni}
\address[Yannis B\"{a}hni]{University of Zurich, R\"{a}mistrasse 71, 8006 Zurich}
\email[Yannis B\"{a}hni]{\href{mailto:yannis.baehni@uzh.ch}{yannis.baehni@uzh.ch}}

\begin{document}
\maketitle
\thispagestyle{fancy}
\begin{enumerate}[label = \textbf{Exercise \arabic*.},wide = 0pt, itemsep=1.5ex]
	\item 
		Let $D \subseteq \mathbb{C}$ be non-empty and open in $\mathbb{C}$ and $f_1,f_2: D \to \mathbb{C}$ be real differentiable. Fix some $z_0 \in D$. Since $f_1$ and $f_2$ are real differentiable in $z_0$ there exists $\varphi_1,\varphi_2,\psi_1,\psi_2: D \to \mathbb{C}$ continuous at $z_0$ such that
				\begin{align}
					f_1(z) &= f_1(z_0) + (z - z_0)\varphi_1(z) + (\overline{z} - \overline{z_0})\psi_1(z)\label{eq:f_1}\\
					f_2(z) &= f_2(z_0) + (z - z_0)\varphi_2(z) + (\overline{z} - \overline{z_0})\psi_2(z)\label{eq:f_2}
				\end{align}
				\noindent for all $z \in D$.
		\begin{enumerate}[label = (\roman*),wide = 0pt, itemsep=1.5ex]
			\item Let $a,b \in \mathbb{C}$. Multiplying (\ref{eq:f_1}) by $a$, (\ref{eq:f_2}) by $b$ and adding both equations yields
				\begin{equation}
					af_1(z) + bf_2(z) = af_1(z_0) + bf_2(z_0) + (z - z_0)(a\varphi_1(z) + b\varphi_2(z)) + (\overline{z} - \overline{z_0})(a\psi_1(z) + b\psi_2(z))
					\label{eq:af_1+bf_2}
				\end{equation} 

				\noindent for all $z \in D$. Clearly, $a\varphi_1 + b \varphi_2$ and $a\psi_1 + b\psi_2$ are continuous functions in $z_0$ and from (\ref{eq:af_1+bf_2}) we deduce
				\begin{equation}
					\pd{(af_1 + bf_2)}{z}(z_0) = a\pd{f_1}{z}(z_0) + b\pd{f_2}{z}(z_0)
				\end{equation}

				\noindent and
				\begin{equation}
					\pd{(af_1 + bf_2)}{\overline{z}}(z_0) = a\pd{f_1}{\overline{z}}(z_0) + b\pd{f_2}{\overline{z}}(z_0).
				\end{equation}

				Since $z_0 \in D$ was arbitrary, we conclude
				\begin{equation}
					\pd{(af_1 + bf_2)}{z} = a\pd{f_1}{z} + b\pd{f_2}{z} \qquad \text{and} \qquad \pd{(af_1 + bf_2)}{\overline{z}} = a\pd{f_1}{\overline{z}} + b\pd{f_2}{\overline{z}}.
				\end{equation}

			\item Multiplying (\ref{eq:f_1}) and (\ref{eq:f_2}) yields
				\begin{align*}
					f_1f_2 = f_1(z_0)f_2(z_0) &+ (z - z_0) \sbr[0]{\varphi_1f_2(z_0) + f_1(z_0)\varphi_2 + (z - z_0)\varphi_1\varphi_2 + (\overline{z} - \overline{z_0})\psi_1\varphi_2}\\
					&+ (\overline{z} - \overline{z_0})\sbr[0]{\psi_1f_2(z_0) + f_1(z_0)\psi_2 + (z - z_0)\psi_2\varphi_1 + (\overline{z} - \overline{z_0})\psi_1\psi_2}
				\end{align*}

				\noindent where the argument $z$ is omitted. Clearly, the two functions in the square brackets are continuous at $z_0$ and evaluating them at $z_0$ yields
				\begin{equation}
					\pd{(f_1f_2)}{z}(z_0) = \pd{f_1}{z}(z_0)f_2(z_0) + f_1(z_0)\pd{f_2}{z}(z_0)
				\end{equation}
				\noindent and
				\begin{equation}
					\pd{(f_1f_2)}{\overline{z}}(z_0) = \pd{f_1}{\overline{z}}(z_0)f_2(z_0) + f_1(z_0)\pd{f_2}{\overline{z}}(z_0).
				\end{equation}

				Since $z_0 \in D$ was arbitrary, we conclude
				\begin{equation}
					\pd{(f_1f_2)}{z} = \pd{f_1}{z}f_2 + f_1\pd{f_2}{z} \qquad \text{and} \qquad \pd{(f_1f_2)}{\overline{z}} = \pd{f_1}{\overline{z}}f_2 + f_1\pd{f_2}{\overline{z}}.
				\end{equation}
			\item Conjugating (\ref{eq:f_1}) yields
				\begin{equation}
					\overline{f_1}(z) = \overline{f_1}(z_0) + (\overline{z} - \overline{z_0})\overline{\varphi_1}(z) + (z - z_0)\overline{\psi_1}(z).
					\label{eq:f_1_conj}
				\end{equation}

				From (\ref{eq:f_1_conj}) we deduce 
				\begin{equation}
					\pd{\overline{f_1}}{\overline{z}}(z_0) = \overline{\varphi_1}(z_0) = \overline{\frac{\partial f_1}{\partial z}}(z_0)
					\label{eq:conj}
				\end{equation}

				\noindent since $\varphi_1$ and $\psi_1$ are also continuous at $z_0$. Taking conjugates in (\ref{eq:conj}) and use that $z_0 \in D$ was arbitrary finally yields
				\begin{equation}
					\overline{\frac{\partial\overline{f_1}}{\partial\overline{z}}} = \pd{f_1}{z}.
				\end{equation}

			\item This follows directly from
				\begin{equation}
					z = z_0 + (z - z_0) \qquad \text{and} \qquad \overline{z} = \overline{z_0} + (\overline{z} - \overline{z_0}).
				\end{equation}

			\item See separate sheet.
			\item See separate sheet.
			\item Let $t_0 \in I$. The function $\varphi: I \to U \subseteq \mathbb{C}$ is differentiable if and only if there exists a function $\varphi_1: I \to U$ which is continuous at $t_0$ and such that
				\begin{equation}
					\varphi(t) = \varphi(t_0) + (t - t_0)\varphi_1(t)
					\label{eq:ordinary}
				\end{equation}

				\noindent for all $t \in I$ (this was proven in Analysis I). Furthermore there exists $f_1,f_2: U \to \mathbb{C}$ continuous at $\varphi(t_0)$ such that 
				\begin{equation}
					f(z) = f(\varphi(t_0)) + (z - \varphi(t_0))f_1(z) + (\overline{z} - \overline{\varphi(t_0)})f_2(z)
					\label{eq:f_1_ordinary}
				\end{equation}
				\noindent for all $z \in U$. Combining (\ref{eq:ordinary}) and (\ref{eq:f_1_ordinary}) yields
				\begin{align*}
					f(\varphi(t)) &= f(\varphi(t_0)) + (\varphi(t) - \varphi(t_0))f_1(z) + (\overline{\varphi(t)} - \overline{\varphi(t_0)})f_2(z)\\
					&= f(\varphi(t_0)) + (t - t_0)\sbr[0]{\varphi_1(t)f_1(\varphi(t)) + \overline{\varphi_1(t)}f_2(\varphi(t))}
				\end{align*}

				\noindent for all $t \in I$. Again, $\varphi_1(t)f(\varphi(t)) + \overline{\varphi_1(t)}f_2(\varphi(t))$ is clearly continuous at $t_0$ since composited functions are and thus we conclude
				\begin{equation}
					\od{(f \circ \varphi)}{t}(t_0) = \od{\varphi}{t}(t_0)\pd{f}{z}(\varphi(t_0)) + \frac{\d \overline{\varphi}}{\d t}(t_0)\pd{f}{\overline{z}}(\varphi(t_0)).
				\end{equation}

				Since $t_0 \in I$ was arbitrary we conclude
				\begin{equation}
					\od{(f \circ \varphi)}{t} = \od{\varphi}{t}\pd{f}{z} + \frac{\d \overline{\varphi}}{\d t}\pd{f}{\overline{z}}.
				\end{equation}
				\end{enumerate}

	\item We show the implications (i) $\Rightarrow$ (ii) $\Rightarrow$ (iii) $\Rightarrow$ (iv) $\Rightarrow$ (i). Since the proofs are of a relatively simple nature, we focus on the formal part. The complex numbers $\mathbb{C}$ are a vector space over $\mathbb{R}$ (as a field extension). So the situation of the exercise can be sumarized by the following commutative diagram:
		\[
			\xymatrix@=2cm{
				\mathbb{C} \ar[r]^T &\mathbb{C}\\
				\mathbb{R}^2 \ar[u]^{\Phi_B} \ar[r]_{M_B(T)} &\mathbb{R}^2 \ar[u]_{\Phi_B}}
	 	\]
		\noindent where $T$ is $\mathbb{R}$-linear, $\Phi_B$ denotes the basis-isomorphism which is in this case given by $\Phi_B(x,y) := x + iy$ and $M_B(T)$ is defined by
		\begin{equation}
			\begin{pmatrix}
				x\\
				y
			\end{pmatrix} \mapsto
			\begin{pmatrix}
				a & b\\
				c & d
			\end{pmatrix}
			\begin{pmatrix}
				x\\
				y
			\end{pmatrix}.
		\end{equation}
		The first implication is evident by the definition of $\mathbb{C}$-linearity. Assume that (ii) holds. By 
		\begin{equation}
			T(i) = (\Phi_B \circ M_B(T) \circ \Phi_B^{-1})(i) = b + id
		\end{equation}
		\noindent and
		\begin{equation}
			iT(1) = i(\Phi_B \circ M_B(T) \circ \Phi_B^{-1})(1) = i(a + ic) = -c + ia 
		\end{equation}

		\noindent we get the requirement $b + id = -c + ia$. Hence $b = -c$ and $a = d$. Assume that (iii) holds. Then we have for $z := x + iy \in \mathbb{C}$
		\begin{equation}
			T(z) = (\Phi_B \circ M_B(T) \circ \Phi_B^{-1})(x + iy) = (ax - cy) + i(cx + ay) = (a + ic)z.
		\end{equation}

		Finally, assume that (iv) holds. Then $T$ is clearly $\mathbb{C}$-linear since 
		\begin{equation}
			T(\lambda z + w) = (a + ic)(\lambda z + w) = \lambda(a + ic)z + (a + ic)w = \lambda T(z) + T(w)
		\end{equation}

		\noindent for $\lambda,z,w \in \mathbb{C}$.
\end{enumerate}
%\originalsectionstyle
\printbibliography
\end{document}
