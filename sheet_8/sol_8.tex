%%%%%%%%%%%%%%%%%%%%%%%%%%%%%%%%%%%%%%%%%%%%%%%%%%%%%%%%%%%%%%%%%%%%%%%%%%
%Author:																 %
%-------																 %
%Yannis Baehni at University of Zurich									 %
%baehni.yannis@uzh.ch													 %
%																		 %
%Version log:															 %
%------------															 %
%06/02/16 . Basic structure												 %
%04/08/16 . Layout changes including section, contents, abstract.		 %
%%%%%%%%%%%%%%%%%%%%%%%%%%%%%%%%%%%%%%%%%%%%%%%%%%%%%%%%%%%%%%%%%%%%%%%%%%

%Page Setup
\documentclass[
	11pt, 
	oneside, 
	a4paper,
	reqno,
	final
]{amsart}

\usepackage[
	left = 3cm, 
	right = 3cm, 
	top = 3cm, 
	bottom = 3cm
]{geometry}

%Headers and footers
\usepackage{fancyhdr}
	\pagestyle{fancy}
	%Clear fields
	\fancyhf{}
	%Header right
	\fancyhead[R]{
		\footnotesize
		Yannis B\"{a}hni\\
		\href{mailto:yannis.baehni@uzh.ch}{yannis.baehni@uzh.ch}
	}
	%Header left
	\fancyhead[L]{
		\footnotesize
		MAT604: Complex Analysis\\
		Spring Semester 2017
	}
	%Page numbering in footer
	\fancyfoot[C]{\thepage}
	%Separation line header and footer
	\renewcommand{\headrulewidth}{0.4pt}
	%\renewcommand{\footrulewidth}{0.4pt}
	
	\setlength{\headheight}{19pt} 

%Title
\usepackage[foot]{amsaddr}
%\usepackage{mathptmx}
\usepackage{xspace}
\makeatletter
\def\@textbottom{\vskip \z@ \@plus 1pt}
\let\@texttop\relax
\usepackage{etoolbox}
\patchcmd{\abstract}{\scshape\abstractname}{\textbf{\abstractname}}{}{}

%Switching commands for different section formats
%Mainsectionsytle
\newcommand{\mainsectionstyle}{%
  	\renewcommand{\@secnumfont}{\bfseries}
  	\renewcommand\section{\@startsection{section}{1}%
    	\z@{.5\linespacing\@plus.7\linespacing}{-.5em}%
    	{\normalfont\bfseries}}%
	\renewcommand\subsection{\@startsection{subsection}{2}%
    	\z@{.5\linespacing\@plus.7\linespacing}{-.5em}%
    	{\normalfont\bfseries}}%
	\renewcommand\subsubsection{\@startsection{subsubsection}{3}%
    	\z@{.5\linespacing\@plus.7\linespacing}{-.5em}%
    	{\normalfont\bfseries}}%
}
\newcommand{\originalsectionstyle}{%
\def\@secnumfont{\bfseries}%\mdseries
\def\section{\@startsection{section}{1}%
  \z@{.7\linespacing\@plus\linespacing}{.5\linespacing}%
  {\normalfont\bfseries\centering}}
}
%Formatting title of TOC
\renewcommand{\contentsnamefont}{\bfseries}
%Table of Contents
\setcounter{tocdepth}{3}

% Add bold to \section titles in ToC and remove . after numbers
\renewcommand{\tocsection}[3]{%
  \indentlabel{\@ifnotempty{#2}{\bfseries\ignorespaces#1 #2\quad}}\bfseries#3}
% Remove . after numbers in \subsection
\renewcommand{\tocsubsection}[3]{%
  \indentlabel{\@ifnotempty{#2}{\ignorespaces#1 #2\quad}}#3}
\let\tocsubsubsection\tocsubsection% Update for \subsubsection
%...

\newcommand\@dotsep{4.5}
\def\@tocline#1#2#3#4#5#6#7{\relax
  \ifnum #1>\c@tocdepth % then omit
  \else
    \par \addpenalty\@secpenalty\addvspace{#2}%
    \begingroup \hyphenpenalty\@M
    \@ifempty{#4}{%
      \@tempdima\csname r@tocindent\number#1\endcsname\relax
    }{%
      \@tempdima#4\relax
    }%
    \parindent\z@ \leftskip#3\relax \advance\leftskip\@tempdima\relax
    \rightskip\@pnumwidth plus1em \parfillskip-\@pnumwidth
    #5\leavevmode\hskip-\@tempdima{#6}\nobreak
    \leaders\hbox{$\m@th\mkern \@dotsep mu\hbox{.}\mkern \@dotsep mu$}\hfill
    \nobreak
    \hbox to\@pnumwidth{\@tocpagenum{\ifnum#1=1\bfseries\fi#7}}\par% <-- \bfseries for \section page
    \nobreak
    \endgroup
  \fi}
\AtBeginDocument{%
\expandafter\renewcommand\csname r@tocindent0\endcsname{0pt}
}
\def\l@subsection{\@tocline{2}{0pt}{2.5pc}{5pc}{}}
\def\l@subsubsection{\@tocline{2}{0pt}{4.5pc}{5pc}{}}
\makeatother

\advance\footskip0.4cm
\textheight=54pc    %a4paper
\textheight=50.5pc %letterpaper
\advance\textheight-0.4cm
\calclayout

%Font settings
%\usepackage{anyfontsize}
%Footnote settings
%\usepackage{mathptmx}
\usepackage{footmisc}
%	\renewcommand*{\thefootnote}{\fnsymbol{footnote}}
\usepackage{commath}
%Further math environments
%Further math fonts (loads amsfonts implicitely)
\usepackage{amssymb}
%Redefinition of \text
%\usepackage{amstext}
\usepackage{upref}
%Graphics
%\usepackage{graphicx}
%\usepackage{caption}
%\usepackage{subcaption}
%Frames
\usepackage{mdframed}
\allowdisplaybreaks
%\usepackage{interval}
\newcommand{\toup}{%
  \mathrel{\nonscript\mkern-1.2mu\mkern1.2mu{\uparrow}}%
}
\newcommand{\todown}{%
  \mathrel{\nonscript\mkern-1.2mu\mkern1.2mu{\downarrow}}%
}
\AtBeginDocument{\renewcommand*\d{\mathop{}\!\mathrm{d}}}
\renewcommand{\Re}{\operatorname{Re}}
\renewcommand{\Im}{\operatorname{Im}}
\DeclareMathOperator\Log{Log}
\DeclareMathOperator\Arg{Arg}
\DeclareMathOperator\sech{sech}
\DeclareMathOperator*\esssup{ess.sup}
%\usepackage{hhline}
%\usepackage{booktabs} 
%\usepackage{array}
%\usepackage{xfrac} 
%\everymath{\displaystyle}
%Enumerate
\usepackage{tikz}
\usetikzlibrary{external}
\tikzexternalize % activate!
\usetikzlibrary{patterns}
\pgfdeclarepatternformonly{adjusted lines}{\pgfqpoint{-1pt}{-1pt}}{\pgfqpoint{40pt}{40pt}}{\pgfqpoint{39pt}{39pt}}%
{
  \pgfsetlinewidth{.8pt}
  \pgfpathmoveto{\pgfqpoint{0pt}{0pt}}
  \pgfpathlineto{\pgfqpoint{39.1pt}{39.1pt}}
  \pgfusepath{stroke}
}
\usepackage{enumitem} 
%\renewcommand{\labelitemi}{$\bullet$}
%\renewcommand{\labelitemii}{$\ast$}
%\renewcommand{\labelitemiii}{$\cdot$}
%\renewcommand{\labelitemiv}{$\circ$}
%Colors
%\usepackage{color}
%\usepackage[cmtip, all]{xy}
%Theorems
\newtheoremstyle{bold}              	 %Name
  {}                                     %Space above
  {}                                     %Space below
  {\itshape}		                     %Body font
  {}                                     %Indent amount
  {\scshape}                             %Theorem head font
  {.}                                    %Punctuation after theorem head
  { }                                    %Space after theorem head, ' ', 
  										 %	or \newline
  {} 
\theoremstyle{bold}
\newtheorem*{definition*}{Definition}
\newtheorem{definition}{Definition}[section]
\newtheorem*{lemma*}{Lemma}
\newtheorem{lemma}{Lemma}[section]
\newtheorem{Proof}{Proof}[section]
\newtheorem{proposition}{Proposition}[section]
\newtheorem{properties}{Properties}[section]
\newtheorem{corollary}{Corollary}[section]
\newtheorem*{theorem*}{Theorem}
\newtheorem{theorem}{Theorem}[section]
\newtheorem{example}{Example}[section]
\newtheorem*{remark*}{Remark}
\newtheorem{remark}{Remark}[section]
%German non-ASCII-Characters
%Graphics-Tool
%\usepackage{tikz}
%\usepackage{tikzscale}
%\usepackage{bbm}
%\usepackage{bera}
%Listing-Setup
%Bibliographie
\usepackage[backend=bibtex, style=alphabetic]{biblatex}
%\usepackage[babel, german = swiss]{csquotes}
\bibliography{Bibliography}
%PDF-Linking
%\usepackage[hyphens]{url}
\usepackage[bookmarksopen=true,bookmarksnumbered=true]{hyperref}
%\PassOptionsToPackage{hyphens}{url}\usepackage{hyperref}
\hypersetup{
  colorlinks   = true, %Colours links instead of ugly boxes
  urlcolor     = blue, %Colour for external hyperlinks
  linkcolor    = blue, %Colour of internal links
  citecolor    = blue %Colour of citations
}
%Weierstrass-P symbol for power set
\newcommand{\powerset}{\raisebox{.15\baselineskip}{\Large\ensuremath{\wp}}}


\title{Solutions Sheet 8}
\author{Yannis B\"{a}hni}
\address[Yannis B\"{a}hni]{University of Zurich, R\"{a}mistrasse 71, 8006 Zurich}
\email[Yannis B\"{a}hni]{\href{mailto:yannis.baehni@uzh.ch}{yannis.baehni@uzh.ch}}

\begin{document}
\maketitle
\thispagestyle{fancy}
\begin{enumerate}[label = \textbf{Exercise \arabic*.},wide = 0pt, itemsep=1.5ex]
	\item We follow \cite[99--101]{fischer2003funktionentheorie}. Let $U := \mathbb{C} \setminus \cbr[0]{\pm e^{\pm i\pi/4}}$ and define $F: U \to \mathbb{C}$ by
		\begin{equation}
			F(z) := \frac{1}{1 + z^4}.	
		\end{equation}
		Clearly $F \in \mathcal{O}(U)$ as a well-defined rational function, $U$ is open in $\mathbb{C}$ and $\mathbb{R} \subseteq U$. Furthermore $F\vert_\mathbb{R} = f$. Hence $F$ is a holomorphic continuation of $f$. Since having an analytic continuation is equivalent to be real-analytic (see \cite[100]{fischer2003funktionentheorie}), we have that $f$ is real-analytic.\\
		Let $x_0 \in \mathbb{R}$. The Taylor series expansion of $f$ is completely determined by the one of $F$. So the only thing which restricts the radius of convergence of the Taylor series expansions are the singularities of $F$. I will again formalize why this is the case. Let 
		\begin{equation}
			F(z) = \sum_{\nu = 0}^\infty a_\nu(z - x_0)^\nu
			\label{eq:expansion}
		\end{equation}
		\noindent be the Taylor expansion of $F$ around $x_0$. By Cauchy-Taylor the radius of convergence of the expansion (\ref{eq:expansion}) is at least $\abs[0]{x_0 - e^{i\pi/2}}$ if $x_0 \geq 0$ and $\abs[0]{x_0 + e^{i\pi/4}}$ if $x_0 \leq 0$. Let $r:= \abs[0]{x_0 - e^{i\pi/4}}$ and assume $x_0 \geq 0$. (the case $x_0 \leq 0$ is similar) and $R > r$. Hence the series in (\ref{eq:expansion}) converges in $B_R(x_0)$. Hence it defines a function $G: B_R(x_0) \to \mathbb{C}$ by
		\begin{equation}
			G(z) := \sum_{\nu = 0}^\infty a_\nu(z - x_0)^\nu 
		\end{equation}
		\noindent with $G\vert_{B_r}(x_0) = F$. Since $G$ is expandable in a power series, we have $G \in \mathcal{O}(B_R(x_0))$ by \cite[187]{remmert2002funktionentheorie}. Since any holomorphic functionis continuous, we have $G \in \mathscr{C}(B_R(x_0))$. Let $(z_\nu)_{\nu \in \mathbb{N}}$ be a sequence in $B_r(x_0)$ such that $\lim_{\nu \to \infty} z_\nu = e^{i\pi/4}$. Clearly
		\begin{equation}
			\lim_{\nu \to \infty}F(z_\nu) = \infty
		\end{equation}
		\noindent and since $G\vert_{B_r(x_0)} = F$ we have
		\begin{equation}
			\lim_{\nu \to \infty} G(z_\nu) = \infty.	
		\end{equation}
		\noindent But since $R > r$, $G$ is continuous at $e^{i\pi/4}$ and so we must have
		\begin{equation}
			G(e^{i\pi/4}) = \lim_{\nu \to \infty}G(z_\nu) = \infty.
		\end{equation}
		Thus the series $G$ diverges at $e^{i\pi/4}$, contradicting that $e^{i\pi/4} \in B_R(x_0)$. Now for general $x_0 \in \mathbb{R}$, the radius of convergence $R$ of the Taylor series expansion of $f$ in $x_0$ is the radius of convergence of the restriction of the Taylor series expansion of $F$ in $x_0$ on $\mathbb{R}$, hence
		\begin{align*}
			R = \begin{cases}
				\abs[0]{x_0 - e^{i\pi/4}} & x_0 \geq 0,\\
				\abs[0]{x_0 + e^{i\pi/4}} & x_0 \leq 0.
			\end{cases}
		\end{align*}

	\item
		~
		\begin{enumerate}[label = (\roman*),wide = 0pt, itemsep=1.5ex]
			\item Since $f \in \mathcal{O}(\mathbb{E})$ we have that $f \in \mathscr{C}(\mathbb{E})$. Thus since $\partial B_r(0)$, $0 \leq r < 1$, is compact we have that $\abs{f}$ attains its supremum on $\partial B_r(0)$. Hence we have 
				\begin{equation}
					M(r) = \max_{\abs[0]{z} = r} \abs[0]{f(z)}.
				\end{equation}
				First we show monotonicity. Let $0 \leq r_1 < r_2 < 1$. We have $\overline{B_{r_2}}(0) \subseteq \mathbb{E}$. Thus $f$ is holomorphic in the bounded domain $B_{r_2}(0)$ and continuous on $\overline{B_{r_2}}(0)$. Then the maximum principle implies that 
				\begin{equation}
					\abs[0]{f(z)} \leq \max_{\zeta \in \partial B_{r_2}(0)}\abs[0]{f(\zeta)} = M(r_2)
				\end{equation}
				\noindent for all $z \in \overline{B_{r_2}}(0)$. In particular 
				\begin{equation}
					M(r_1) = \max_{\zeta \in \partial B_{r_1}(0)}\abs[0]{f(\zeta)} \leq \max_{\zeta \in \partial B_{r_2}(0)}\abs[0]{f(\zeta)} = M(r_2).
				\end{equation}
				Thus $M$ is monotonically increasing.
			\item Proof by contradiction. Assume that $f$ is not constant and that $M$ is not strictly increasing. Hence we find $0 \leq r_1 < r_2 < 1$ such that $M(r_1) = M(r_2)$ since by part (i) we already know that $M$ is monotone increasing. Thus we find $z_0 \in B_{r_1}(0)$ such that $M(r_1) = \abs[0]{f(z_0)}$. An application of the maximum principle similar to part (i) yields
				\begin{equation}
					\abs[0]{f(z)} \leq \max_{\zeta \in \partial B_{r_2}(0)}	\abs[0]{f(\zeta)} = M(r_2) = M(r_1) = \abs[0]{f(z_0)}
				\end{equation}
				\noindent for all $z \in \overline{B_{r_2}}(0)$. But $z_0 \in B_{r_1}(0)$ and $r_1 < r_2$, thus $B_{r_2 - r_1}(z_0) \neq \cbr[0]{z_0}$. Hence $\abs[0]{f}$ has a local maximum in $B_{r_2}(0)$ and thus by the maximum principle, $f$ is constant in $B_{r_2}(0)$. Since $0 < r_2$, $B_{r_2}(0)$ is not discrete in $\mathbb{E}$, hence if we define $g: \mathbb{E} \to \mathbb{C}$ by $g(z) := f(z_0)$, clearly $g \in \mathcal{O}(\mathbb{R})$ and $f = g$ on $B_{r_2}(0)$. Hence by the second version of the identity principle we have $f = g$ on $\mathbb{E}$ which implies that $f$ is constant on $\mathbb{E}$. Contradiction. 
		\end{enumerate}

	\item Proof by contradiction. Since no point $\zeta \in \partial B_R(z_0)$ is singular, we find a neighbourhood $U_\zeta$ of $\zeta$ and a function $f_\zeta \in \mathcal{O}(U_\zeta)$, such that 
		\begin{equation}
			f_\zeta(z) = \sum_{\nu = 0}^\infty a_\nu(z - z_0)^\nu
		\end{equation}
		\noindent for all $z \in U_\zeta \cap B_R(z_0)$. Since each $U_\zeta$ contains an open set containing $\zeta$, we find $r_\zeta$, such that $B_{r_\zeta}(\zeta) \subseteq U_\zeta$. Therefore we have that
		\begin{equation}
			\partial B_R(z_0) \subseteq \bigcup_{\zeta \in \partial B_R(z_0)} B_{r_\zeta}(\zeta)
		\end{equation}
		\noindent is an open cover of $\partial B_R(z_0)$. Since $\partial B_R(z_0)$ is compact, we find $\zeta_1,\dots,\zeta_n$ such that $B_{r_{\zeta_1}}(\zeta_1),\dots,B_{r_{\zeta_n}}(\zeta_n)$ still covers $\partial B_R(z_0)$. The next step is conceptually easy, but notationally ugly. We will explain it in a quite informal way. Now the intersection $B_{r_{\zeta_\nu}}(\zeta_\nu) \cap B_{r_{\zeta_\mu}}(\zeta_\mu)$ is open and thus if $B_{r_{\zeta_\nu}}(\zeta_\nu) \cap B_{r_{\zeta_\mu}}(\zeta_\mu) \neq \varnothing$, we find an open ball contained in the intersection $B_{r_{\zeta_\nu}}(\zeta_\nu) \cap B_{r_{\zeta_\mu}}(\zeta_\mu) \neq \varnothing$. Taking the minimum of all radii of those balls lying in the intersection (this is possible since there are only finitely many ones), we find $\hat{R} > R$ such that
		\begin{equation}
			\partial B_{\hat{R}}(z_0) \subseteq \bigcup_{k = 1}^n B_{r_{\zeta_k}}(\zeta_k).
		\end{equation}
		Next we construct a function $g: B_{\hat{R}}(z_0) \to \mathbb{C}$. Define $g(z) := \sum_{\nu = 0}^\infty a_\nu (z - z_0)^\nu$ if $z \in B_R(z_0)$. If $z \in \del[0]{B_{r_{\zeta_\nu}}(\zeta_\nu) \cap B_{r_{\zeta_\mu}}(\zeta_\mu)} \setminus B_R(z_0)$, we have that $f_{\zeta_\nu} = f_{\zeta_\mu}$ in $B_{r_{\zeta_\nu}}(\zeta_\nu) \cap B_{r_{\zeta_\mu}}(\zeta_\mu) \cap B_R(z_0)$, which is open and therefore not discrete in $B_{r_{\zeta_\nu}}(\zeta_\nu) \cap B_{r_{\zeta_\mu}}(\zeta_\mu)$. Thus by the second version of the identity principle we have $f_{\zeta_\nu} = f_{\zeta_{\mu}}$ on $B_{r_{\zeta_\nu}}(\zeta_\nu) \cap B_{r_{\zeta_\mu}}(\zeta_\mu)$. Therefore $g(z) := f_{\zeta_{\nu}}(z) = f_{\zeta_\mu}(z)$ is well defined. In the remaining cases, define $g(z) := f_{\zeta_\nu}(z)$ if $z \in B_{r_{\zeta_\nu}}(\zeta_\nu)$. Since $f_{\zeta_\nu} \in \mathcal{O}(\zeta_\nu)$ and by the theorem on interchangeability of differentiation and summation we have that any power series is holomorphic within its radius of convergence, we have that $g \in \mathcal{O}(B_{\hat{R}}(z_0))$. An application of Cauchy-Taylor yields
		\begin{equation}
			g(z) = \sum_{\nu = 0}^\infty \frac{g^{(\nu)}(z_0)}{\nu !}(z - z_0)^\nu = \sum_{\nu = 0}^\infty \frac{f^{(\nu)}(z_0)}{\nu !}(z - z_0)^\nu
		\end{equation}
		\noindent for all $z \in B_{\hat{R}}(z_0)$ since $g = f$ on $B_R(z_0)$. Furthermore, since $\hat{R} > R$ we have that $\sum_{\nu = 0}^\infty a_\nu(z - z_0)^\nu$ is convergent in $B_{\hat{R}}(z_0) \setminus \overline{B_R}(z_0)$, contradicting that $\sum_{\nu = 0}^\infty a_{\nu} (z - z_0)^\nu$ is divergent there by the definition of the radius of convergence. Contradiction.

	\item Central is Weierstrass' differentiation theorem for compact convergent series. For each $\nu \in \mathbb{N}_0$ let
		\begin{equation}
			f_\nu(z) := \sum_{\mu = 0}^\infty c_{\nu \mu}(z - z_0)^\mu
		\end{equation}
		\noindent be convergent in $B_r(z_0)$, $r > 0$, $z_0 \in \mathbb{C}$. Furthermore, assume that
		\begin{equation}
			f(z) := \sum_{\nu = 0}^\infty f_\nu(z) = \sum_{\nu = 0}^\infty \sum_{\mu = 0}^\infty c_{\nu \mu}(z - z_0)^\mu
		\end{equation}
		\noindent is normally convergent in $B_r(z_0)$. Since $r > 0$, the theorem on interchangeability of differentiation and summation of power series implies that $f_\nu \in \mathcal{O}(B_r(z_0))$ for all $\nu \in \mathbb{N}_0$. Since $\sum_{\nu = 0}^\infty f_\nu$ is normally convergent in $B_r(z_0)$, we have that $\sum_{\nu = 0}^\infty f_\nu$ is locally uniformly convergent in $B_r(z_0)$ and thus compactly convergent in $B_r(z_0)$. Hence Weierstrass' theorem implies that the limit function $f$ is holomorphic in $B_r(z_0)$. Thus by the expansion theorem of Cauchy-Taylor, for any $z \in B_r(z_0)$ we find a disc centered at $z$ where $f$ is expandable in a Taylor seriers. This implies that $f$ is analytic in $B_r(z_0)$. Furthermore, the same theorem implies that for any $k \in \mathbb{N}_0$ we have
		\begin{equation}
			f^{(k)}(z) = \sum_{\nu = 0}^\infty f_\nu^{(k)}(z) = \sum_{\nu = 0}^\infty \sum_{\mu = k}^\infty k! {\mu \choose k}c_{\nu \mu}(z - z_0)^{\mu - k} 
		\end{equation}
		\noindent for all $z \in B_r(z_0)$ by the theorem on interchangeability of differentiation and summation of power series. Since $f \in \mathcal{O}(B_r(z_0))$, the expansion theorem of Cauchy-Taylor implies that 
		\begin{equation}
			f(z) = \sum_{k = 0}^\infty \frac{f^{(k)}(z_0)}{k !} (z - z_0)^k = \sum_{k = 0}^\infty \del[4]{\sum_{\nu = 0}^\infty c_{\nu k}} (z - z_0)^k
		\end{equation}
		\noindent for all $z \in B_r(z_0)$.
\end{enumerate}
%\originalsectionstyle
\printbibliography
\end{document}
