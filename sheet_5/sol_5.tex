%%%%%%%%%%%%%%%%%%%%%%%%%%%%%%%%%%%%%%%%%%%%%%%%%%%%%%%%%%%%%%%%%%%%%%%%%%
%Author:																 %
%-------																 %
%Yannis Baehni at University of Zurich									 %
%baehni.yannis@uzh.ch													 %
%																		 %
%Version log:															 %
%------------															 %
%06/02/16 . Basic structure												 %
%04/08/16 . Layout changes including section, contents, abstract.		 %
%%%%%%%%%%%%%%%%%%%%%%%%%%%%%%%%%%%%%%%%%%%%%%%%%%%%%%%%%%%%%%%%%%%%%%%%%%

%Page Setup
\documentclass[
	11pt, 
	oneside, 
	a4paper,
	reqno,
	final
]{amsart}

\usepackage[
	left = 3cm, 
	right = 3cm, 
	top = 3cm, 
	bottom = 3cm
]{geometry}

%Headers and footers
\usepackage{fancyhdr}
	\pagestyle{fancy}
	%Clear fields
	\fancyhf{}
	%Header right
	\fancyhead[R]{
		\footnotesize
		Yannis B\"{a}hni\\
		\href{mailto:yannis.baehni@uzh.ch}{yannis.baehni@uzh.ch}
	}
	%Header left
	\fancyhead[L]{
		\footnotesize
		MAT604: Complex Analysis\\
		Spring Semester 2017
	}
	%Page numbering in footer
	\fancyfoot[C]{\thepage}
	%Separation line header and footer
	\renewcommand{\headrulewidth}{0.4pt}
	%\renewcommand{\footrulewidth}{0.4pt}
	
	\setlength{\headheight}{19pt} 

%Title
\usepackage[foot]{amsaddr}
%\usepackage{mathptmx}
\usepackage{xspace}
\makeatletter
\def\@textbottom{\vskip \z@ \@plus 1pt}
\let\@texttop\relax
\usepackage{etoolbox}
\patchcmd{\abstract}{\scshape\abstractname}{\textbf{\abstractname}}{}{}

%Switching commands for different section formats
%Mainsectionsytle
\newcommand{\mainsectionstyle}{%
  	\renewcommand{\@secnumfont}{\bfseries}
  	\renewcommand\section{\@startsection{section}{1}%
    	\z@{.5\linespacing\@plus.7\linespacing}{-.5em}%
    	{\normalfont\bfseries}}%
	\renewcommand\subsection{\@startsection{subsection}{2}%
    	\z@{.5\linespacing\@plus.7\linespacing}{-.5em}%
    	{\normalfont\bfseries}}%
	\renewcommand\subsubsection{\@startsection{subsubsection}{3}%
    	\z@{.5\linespacing\@plus.7\linespacing}{-.5em}%
    	{\normalfont\bfseries}}%
}
\newcommand{\originalsectionstyle}{%
\def\@secnumfont{\bfseries}%\mdseries
\def\section{\@startsection{section}{1}%
  \z@{.7\linespacing\@plus\linespacing}{.5\linespacing}%
  {\normalfont\bfseries\centering}}
}
%Formatting title of TOC
\renewcommand{\contentsnamefont}{\bfseries}
%Table of Contents
\setcounter{tocdepth}{3}

% Add bold to \section titles in ToC and remove . after numbers
\renewcommand{\tocsection}[3]{%
  \indentlabel{\@ifnotempty{#2}{\bfseries\ignorespaces#1 #2\quad}}\bfseries#3}
% Remove . after numbers in \subsection
\renewcommand{\tocsubsection}[3]{%
  \indentlabel{\@ifnotempty{#2}{\ignorespaces#1 #2\quad}}#3}
\let\tocsubsubsection\tocsubsection% Update for \subsubsection
%...

\newcommand\@dotsep{4.5}
\def\@tocline#1#2#3#4#5#6#7{\relax
  \ifnum #1>\c@tocdepth % then omit
  \else
    \par \addpenalty\@secpenalty\addvspace{#2}%
    \begingroup \hyphenpenalty\@M
    \@ifempty{#4}{%
      \@tempdima\csname r@tocindent\number#1\endcsname\relax
    }{%
      \@tempdima#4\relax
    }%
    \parindent\z@ \leftskip#3\relax \advance\leftskip\@tempdima\relax
    \rightskip\@pnumwidth plus1em \parfillskip-\@pnumwidth
    #5\leavevmode\hskip-\@tempdima{#6}\nobreak
    \leaders\hbox{$\m@th\mkern \@dotsep mu\hbox{.}\mkern \@dotsep mu$}\hfill
    \nobreak
    \hbox to\@pnumwidth{\@tocpagenum{\ifnum#1=1\bfseries\fi#7}}\par% <-- \bfseries for \section page
    \nobreak
    \endgroup
  \fi}
\AtBeginDocument{%
\expandafter\renewcommand\csname r@tocindent0\endcsname{0pt}
}
\def\l@subsection{\@tocline{2}{0pt}{2.5pc}{5pc}{}}
\def\l@subsubsection{\@tocline{2}{0pt}{4.5pc}{5pc}{}}
\makeatother

\advance\footskip0.4cm
\textheight=54pc    %a4paper
\textheight=50.5pc %letterpaper
\advance\textheight-0.4cm
\calclayout

%Font settings
%\usepackage{anyfontsize}
%Footnote settings
%\usepackage{mathptmx}
\usepackage{footmisc}
%	\renewcommand*{\thefootnote}{\fnsymbol{footnote}}
\usepackage{commath}
%Further math environments
%Further math fonts (loads amsfonts implicitely)
\usepackage{amssymb}
%Redefinition of \text
%\usepackage{amstext}
\usepackage{upref}
%Graphics
%\usepackage{graphicx}
%\usepackage{caption}
%\usepackage{subcaption}
%Frames
\usepackage{mdframed}
\allowdisplaybreaks
%\usepackage{interval}
\newcommand{\toup}{%
  \mathrel{\nonscript\mkern-1.2mu\mkern1.2mu{\uparrow}}%
}
\newcommand{\todown}{%
  \mathrel{\nonscript\mkern-1.2mu\mkern1.2mu{\downarrow}}%
}
\AtBeginDocument{\renewcommand*\d{\mathop{}\!\mathrm{d}}}
\renewcommand{\Re}{\operatorname{Re}}
\renewcommand{\Im}{\operatorname{Im}}
\DeclareMathOperator\Log{Log}
\DeclareMathOperator\Arg{Arg}
\DeclareMathOperator\sech{sech}
\DeclareMathOperator*\esssup{ess.sup}
%\usepackage{hhline}
%\usepackage{booktabs} 
%\usepackage{array}
%\usepackage{xfrac} 
%\everymath{\displaystyle}
%Enumerate
\usepackage{tikz}
\usetikzlibrary{external}
\tikzexternalize % activate!
\usetikzlibrary{patterns}
\pgfdeclarepatternformonly{adjusted lines}{\pgfqpoint{-1pt}{-1pt}}{\pgfqpoint{40pt}{40pt}}{\pgfqpoint{39pt}{39pt}}%
{
  \pgfsetlinewidth{.8pt}
  \pgfpathmoveto{\pgfqpoint{0pt}{0pt}}
  \pgfpathlineto{\pgfqpoint{39.1pt}{39.1pt}}
  \pgfusepath{stroke}
}
\usepackage{enumitem} 
%\renewcommand{\labelitemi}{$\bullet$}
%\renewcommand{\labelitemii}{$\ast$}
%\renewcommand{\labelitemiii}{$\cdot$}
%\renewcommand{\labelitemiv}{$\circ$}
%Colors
%\usepackage{color}
%\usepackage[cmtip, all]{xy}
%Theorems
\newtheoremstyle{bold}              	 %Name
  {}                                     %Space above
  {}                                     %Space below
  {\itshape}		                     %Body font
  {}                                     %Indent amount
  {\scshape}                             %Theorem head font
  {.}                                    %Punctuation after theorem head
  { }                                    %Space after theorem head, ' ', 
  										 %	or \newline
  {} 
\theoremstyle{bold}
\newtheorem*{definition*}{Definition}
\newtheorem{definition}{Definition}[section]
\newtheorem*{lemma*}{Lemma}
\newtheorem{lemma}{Lemma}[section]
\newtheorem{Proof}{Proof}[section]
\newtheorem{proposition}{Proposition}[section]
\newtheorem{properties}{Properties}[section]
\newtheorem{corollary}{Corollary}[section]
\newtheorem*{theorem*}{Theorem}
\newtheorem{theorem}{Theorem}[section]
\newtheorem{example}{Example}[section]
\newtheorem*{remark*}{Remark}
\newtheorem{remark}{Remark}[section]
%German non-ASCII-Characters
%Graphics-Tool
%\usepackage{tikz}
%\usepackage{tikzscale}
%\usepackage{bbm}
%\usepackage{bera}
%Listing-Setup
%Bibliographie
\usepackage[backend=bibtex, style=alphabetic]{biblatex}
%\usepackage[babel, german = swiss]{csquotes}
\bibliography{Bibliography}
%PDF-Linking
%\usepackage[hyphens]{url}
\usepackage[bookmarksopen=true,bookmarksnumbered=true]{hyperref}
%\PassOptionsToPackage{hyphens}{url}\usepackage{hyperref}
\hypersetup{
  colorlinks   = true, %Colours links instead of ugly boxes
  urlcolor     = blue, %Colour for external hyperlinks
  linkcolor    = blue, %Colour of internal links
  citecolor    = blue %Colour of citations
}
%Weierstrass-P symbol for power set
\newcommand{\powerset}{\raisebox{.15\baselineskip}{\Large\ensuremath{\wp}}}


\title{Solutions Sheet 5}
\author{Yannis B\"{a}hni}
\address[Yannis B\"{a}hni]{University of Zurich, R\"{a}mistrasse 71, 8006 Zurich}
\email[Yannis B\"{a}hni]{\href{mailto:yannis.baehni@uzh.ch}{yannis.baehni@uzh.ch}}

\begin{document}
\maketitle
\thispagestyle{fancy}
\begin{enumerate}[label = \textbf{Exercise \arabic*.},wide = 0pt, itemsep=1.5ex]
	\item 
		~
	\begin{enumerate}[label = (\alph*),wide = 0pt, itemsep=1.5ex]
		\item We summarize the result in a lemma. 
			\begin{lemma}
				The power series
				\begin{equation}
					\sum_{\nu = 0}^\infty \frac{z^{2\nu}}{(2\nu)!} \qquad \text{and} \qquad \sum_{\nu = 0}^\infty \frac{z^{2\nu + 1}}{(2\nu + 1)!}
				\end{equation}

				\noindent have both radius of convergence $R = \infty$. Furthermore, for all $z \in \mathbb{C}$
				\begin{equation}
					\cosh z = \sum_{\nu = 0}^\infty \frac{z^{2\nu}}{(2\nu)!} \qquad \text{and} \qquad \sinh z = \sum_{\nu = 0}^\infty \frac{z^{2\nu + 1}}{(2\nu + 1)!}
				\end{equation}
				\noindent holds.
				\label{lem:pow_cosh}
			\end{lemma}

			\begin{proof}
				Fix $z \in \mathbb{C}$. We have 
			\begin{equation}
				\limsup_{\nu \to \infty} \abs[3]{\frac{z^{2\nu + 2}}{(2\nu + 2)!} \frac{(2\nu)!}{z^{2\nu}}} = \abs[0]{z}^2\limsup_{\nu \to \infty} \frac{1}{(2\nu + 2)(2\nu + 1)} = 0 < 1
			\end{equation}

			\noindent and
			\begin{equation}
			\limsup_{\nu \to \infty} \abs[3]{\frac{z^{2\nu + 3}}{(2\nu + 3)!} \frac{(2\nu + 1)!}{z^{2\nu + 1}}} = \abs[0]{z}^2\limsup_{\nu \to \infty} \frac{1}{(2\nu + 3)(2\nu + 2)} = 0 < 1.
			\end{equation}

			Since $z$ was arbitrary we conclude by the ratio test for series that both radii of convergence are $\infty$. Using the identities 
			\begin{equation}
				\cosh z = \cos(iz) \qquad \text{and} \qquad \sinh z = -i\sin(iz) \qquad \forall z \in \mathbb{C}
			\end{equation}
			\noindent and the definition of the trigonometric functions by series
			\begin{equation}
				\cos z := \sum_{\nu = 0}^\infty \frac{(-1)^\nu}{(2\nu)!}z^{2\nu} \qquad \text{and} \qquad \sin z := \sum_{\nu = 0}^\infty \frac{(-1)^{\nu}}{(2\nu + 1)!}z^{2\nu + 1}
			\end{equation}

			\noindent we get
			\begin{equation}
				\cosh z = \cos(iz) = \sum_{\nu = 0}^\infty \frac{(-1)^\nu}{(2\nu)!}(iz)^{2\nu} = \sum_{\nu = 0}^\infty \frac{(-1)^{2\nu}}{(2\nu)!}z^{2\nu} = \sum_{\nu = 0}^\infty \frac{z^{2\nu}}{(2\nu)!}
			\end{equation}

			\noindent and
			\begin{equation}
				\sinh z = -i\sin(iz) = -i \sum_{\nu = 0}^\infty \frac{(-1)^{\nu}}{(2\nu + 1)!}(iz)^{2\nu + 1} = \sum_{\nu = 0}^\infty \frac{(-1)^{2\nu}}{(2\nu + 1)!}z^{2\nu + 1} = \sum_{\nu = 0}^\infty \frac{z^{2\nu + 1}}{(2\nu + 1)!}
			\end{equation}
			\noindent for all $z \in \mathbb{Z}$.
			\end{proof}
			\begin{remark}
				The power series given in lemma \ref{lem:pow_cosh} can be rewritten into the standard form
				\begin{equation}
					\sum_{\nu = 0}^\infty a_\nu (z - z_0)^\nu
				\end{equation}

				\noindent by considering apropriate sequences $(a_\nu)_{\nu \in \mathbb{N}}$. Also it is clearly seen that $z_0 = 0$ is the point of expansion.
			\end{remark}

		\item Define $a_\nu := (-1)^{\nu - 1}/\nu$ for $\nu \in \mathbb{N}$ and $a_0 := 0$. Since $(a_\nu)_{\nu \in \mathbb{N}}$ is convergent, the quotient criterion yields
			\begin{equation}
				R = \lim_{\nu \to \infty} \abs[3]{\frac{a_\nu}{a_{\nu + 1}}} = \lim_{\nu \to \infty}\abs[3]{\frac{(-1)^{\nu - 1}}{\nu} \frac{\nu + 1}{(-1)^{\nu}}} = 1 + \lim_{\nu \to \infty}\frac{1}{\nu} = 1. 
			\end{equation}
			Thus the logarithmic series converges in $\mathbb{E}$ since the point of expansion $z_0$ is clearly $0$. Since $R > 0$ we have that the limit function $f$ is holomorphic in $\mathbb{E}$ by the theorem on the \emph{interchangeability of differentiation and summation}. Furthermore, from the same theorem also follows that the derivative of the limit function coincides with the naive termwise differentiation of the power series within $\mathbb{E}$. Thus we get
			\begin{equation}
				f'(z) = \sum_{\nu = 1}^\infty \nu a_\nu z^{\nu-1} = \sum_{\nu = 1}^\infty (-z)^{\nu-1} = \sum_{\mu = 0}^\infty (-z)^\mu = \frac{1}{1 + z}
			\end{equation}

			\noindent by the formula for the sum of a geometric series (if $z \in \mathbb{E}$ so is $-z \in \mathbb{E}$).

		\item Fix $z \in \mathbb{C}$ and let $a_\nu := (-1)^{\nu}/(2\nu + 1)z^{2\nu + 1}$ for $\nu \in \mathbb{N}_0$. By
			\begin{align*}
				\limsup_{\nu \to \infty} \abs[3]{\frac{a_{\nu + 1}}{a_{\nu}}} &= \limsup_{\nu \to \infty} \abs[3]{\frac{(-1)^{\nu + 1}z^{2\nu + 3}}{2\nu + 3}\frac{2\nu + 1}{(-1)^\nu z^{2\nu + 1}}}\\
				&= \abs[0]{z}^2 \limsup_{\nu \to \infty} \frac{2\nu + 1}{2\nu + 3}\\
				&= \abs[0]{z}^2
			\end{align*}

			\noindent we deduce that $\abs[0]{z}^2 < 1$ must hold that the series is convergent. This is equivalent to $z \in \mathbb{E}$. Thus the arcustangens series converges in $\mathbb{E}$ since the point of expansion $z_0$ is clearly $0$. Since $R > 0$ we have that the limit function $g$ is holomorphic in $\mathbb{E}$ by the theorem on the \emph{interchangeability of differentiation and summation}. Furthermore, from the same theorem also follows that the derivative of the limit function coincides with the naive termwise differentiation of the power series within $\mathbb{E}$. First of all we have to bring the power series in an apropriate form. We have
			\begin{align*}
				g(z) = \sum_{\nu = 0}^\infty a_\nu = \sum_{\nu = 0}^\infty b_\nu z^\nu \qquad \text{where} \qquad b_\nu := \begin{cases}
					0 & \nu \equiv 0 \bmod 2,\\
					1/\nu & \nu \equiv 1 \bmod 4,\\
					-1/\nu & \nu \equiv 3 \bmod 4.
				\end{cases}
			\end{align*}
			\noindent Hence
			\begin{equation}
				g'(z) = \sum_{\nu = 1}^\infty \nu b_\nu z^{\nu-1} = \sum_{\nu = 0}^\infty (-1)^\nu z^{2\nu} = \sum_{\nu = 0}^\infty \del[0]{-z^2}^\nu = \frac{1}{1 + z^2}
			\end{equation}

			\noindent by the formula for the sum of a geometric series (if $z \in \mathbb{E}$ so is $-z^2\in \mathbb{E}$).
	\end{enumerate}

\item
	~
	\begin{enumerate}[label = (\alph*),wide = 0pt, itemsep=1.5ex]
		\item Define $\gamma_0 \ast \dots \ast \gamma_n: I \to U$ where 
			\begin{equation}
				\textstyle I := \intcc{a_0, b_0 + \sum_{\nu=1}^n (b_\nu - a_\nu)}
			\end{equation}

			\noindent by
			\begin{align*}
				\gamma_0 \ast \dots \ast \gamma_n(t) := \begin{cases}
					\gamma_0(t) & t \in A_0,\\
					\gamma_1(t + a_1 - b_0) & t \in A_1,\\
					\gamma_\nu\del{t + a_\nu - b_0 - \sum_{\mu = 1}^{\nu-1} (b_\mu - a_\mu)} & t \in A_\nu, \nu = 2,\dots,n,
				\end{cases}
			\end{align*}
			\noindent where 
			\begin{align*}
				A_\nu:= \begin{cases}
					\intcc{a_0,b_0} & \nu = 0,\\
					\intcc{b_0,b_1 - a_1 + b_0} & \nu = 1,\\
					\intcc{b_0 + \sum_{\mu = 1}^{\nu-1} (b_\mu - a_\mu),b_0 + \sum_{\mu = 1}^{\nu}(b_\mu - a_\mu)} & \nu = 2,\dots,n.
				\end{cases}
			\end{align*}

			Let $n \in \mathbb{N}_{> 0}$. Recall, that for $z_0,\dots,z_n$ the path $\intcc{z_0,\dots,z_n}: \intcc{0,n} \to \mathbb{C}$ defined by 
			\begin{equation}
				\intcc{z_0,\dots,z_n}(t) := z_\nu + (t - \nu)(z_{\nu + 1} - z_\nu) \qquad t \in \intcc{\nu, \nu + 1}
			\end{equation}

			\noindent for $\nu = 0,\dots,n - 1$ is called a \bld{polygon}. Consider the paths $\gamma_\nu := \intcc{z_\nu,z_{\nu + 1}}$, $\nu = 0,\dots,n-1$. Then we have
			\begin{equation}
				\textstyle I = \intcc[1]{0, 1 + \sum_{\nu = 1}^{n - 1}1} = \intcc{0,n}
			\end{equation}

			\noindent and
			\begin{align*}
				A_\nu = \begin{cases}
					\intcc{a_0,b_0} = \intcc{0,1} & \nu = 0,\\
					\intcc{b_0,b_1 - a_1 + b_0} = \intcc{1,2} & \nu = 1,\\
					\intcc{b_0 + \sum_{\mu = 1}^{\nu-1} (b_\mu - a_\mu),b_0 + \sum_{\mu = 1}^{\nu}(b_\mu - a_\mu)} = \intcc{\nu,\nu + 1} & \nu = 2,\dots,n - 1.
				\end{cases}
			\end{align*}

			Hence $A_\nu = \intcc{\nu,\nu + 1}$ for $\nu = 0,\dots,n - 1$. Furthermore
			\begin{align*}
				\gamma_0 \ast \dots \ast \gamma_{n-1}(t) = \begin{cases}
					\gamma_0(t) = \intcc{z_0,z_1} & t \in A_0,\\
					\gamma_1(t + a_1 - b_0) = z_1 + (t - 1)(z_2 - z_1) & t \in A_1,\\	
				\end{cases}
			\end{align*}
			\noindent and
			\begin{equation}
				\textstyle\gamma_0 \ast \dots \ast \gamma_{n-1}(t) = \gamma_\nu\del{t + a_\nu - b_0 - \sum_{\mu = 1}^{\nu-1} (b_\mu - a_\mu)} = z_\nu + (t - \nu)(z_{\nu + 1} - z_\nu)
			\end{equation}
			\noindent for $t \in A_\nu, \nu = 2,\dots,n - 1$. Hence we conclude that
			\begin{equation}
				\intcc{z_0,\dots,z_n} = \intcc{z_0,z_1} \ast \dots \ast \intcc{z_{n-1},z_n}.
			\end{equation}
		\item An integration path in $U$ is by definition a piecewise continuously differentiable mapping. Hence there exists a partition $a = t_0 < t_1 < \dots < t_n = b$ of $\intcc{a,b}$ such that $\gamma\vert_{\intcc{t_\nu,t_{\nu + 1}}}$ is continuously differentiable for $\nu = 0,\dots,n-1$. Let $\gamma_{\nu} := \gamma\vert_{\intcc{t_\nu,t_{\nu + 1}}}$ for $\nu = 0,\dots,n-1$. Clearly
			\begin{equation}
				\gamma_\nu : \intcc{t_\nu,t_{\nu + 1}} \to U
			\end{equation}

			Using the terminology established in part (a) we get
			\begin{equation}
				\textstyle I = \intcc{a,t_1 + \sum_{\nu = 1}^{n-1}(t_{\nu + 1} - t_\nu)} = \intcc{a,t_n} = \intcc{a,b}
			\end{equation}
			\noindent and
			\begin{align*}
				A_\nu = \begin{cases}
					\intcc{a_0,b_0} = \intcc{a,t_1} & \nu = 0,\\
					\intcc{t_1,b_1 - a_1 + b_0} = \intcc{t_1,t_2} & \nu = 1,\\
					\intcc{b_0 + \sum_{\mu = 1}^{\nu-1} (b_\mu - a_\mu),b_0 + \sum_{\mu = 1}^{\nu}(b_\mu - a_\mu)} = \intcc{t_{\nu},t_{\nu + 1}} & \nu = 2,\dots,n - 1.
				\end{cases}
			\end{align*}

			Furthermore
			\begin{align*}
				\gamma_0 \ast \dots \ast \gamma_{n-1}(t) = \begin{cases}
					\gamma_0(t) = \gamma(t) & t \in A_0,\\
					\gamma_1(t + a_1 - b_0) = \gamma(t) & t \in A_1,\\	
					\gamma_\nu\del{t + a_\nu - b_0 - \sum_{\mu = 1}^{\nu-1} (b_\mu - a_\mu)} = \gamma(t) & t \in A_\nu, \nu = 2,\dots,n - 1.
				\end{cases}
			\end{align*}

			Hence we conclude
			\begin{equation}
				\gamma_{0} \ast \dots \ast \gamma_{n-1} = \gamma.
			\end{equation}
	\end{enumerate}
\item 
	~
	\begin{lemma}
		For $z_0 \in \mathbb{C}$ we have
		\begin{equation}
			\frac{1}{\zeta - z} = \frac{1}{\zeta - z_0} \sum_{\nu = 0}^\infty \del{\frac{z - z_0}{\zeta - z_0}}^{\nu} \qquad \text{for all } \zeta,z \text{ with } \abs[0]{z - z_0} < \abs[0]{\zeta - z_0}
			\label{eq:ser_1}
		\end{equation}
		\noindent and
		\begin{equation}
			\frac{1}{\zeta - z} = -\frac{1}{z - z_0} \sum_{\nu = 0}^\infty \del{\frac{\zeta - z_0}{z - z_0}}^{\nu} \qquad \text{for all } \zeta,z \text{ with } \abs[0]{\zeta - z_0} < \abs[0]{z - z_0}.
			\label{eq:ser_2}
		\end{equation}
		For $B_r(z_0)$ fixed, series (\ref{eq:ser_1}) converges normally as a function series in the argument $\zeta$ on $\partial B_r(z_0)$ for any $z \in B_r(z_0)$ whereas series (\ref{eq:ser_2}) converges normally as a function series in the argument $\zeta$ on $\partial B_r(z_0)$ for any $z \in \mathbb{C} \setminus \overline{B_r}(z_0)$.
		\label{lem:geo}
	\end{lemma}

	\begin{proof}
		In the first case we have
		\begin{equation}
			\frac{1}{\zeta - z} = \frac{1}{\zeta - z_0}\frac{\zeta - z_0}{\zeta - z} = \frac{1}{\zeta - z_0}\frac{1}{1 - (z - z_0)/(\zeta - z_0)} = \frac{1}{\zeta - z_0} \sum_{\nu = 0}^\infty \del{\frac{z - z_0}{\zeta - z_0}}^{\nu}
		\end{equation}
		\noindent since by assumption $\abs[0]{(z - z_0)/(\zeta - z_0)} < 1$ and in the second
		\begin{equation}
			\frac{1}{\zeta - z} = \frac{1}{z - z_0}\frac{z - z_0}{\zeta - z} = -\frac{1}{\zeta - z_0}\frac{1}{1 - (\zeta - z_0)/(z - z_0)} = -\frac{1}{\zeta - z_0} \sum_{\nu = 0}^\infty \del{\frac{\zeta - z_0}{z - z_0}}^{\nu}
		\end{equation}
		\noindent since again by assumption $\abs[0]{(\zeta - z_0)/(z - z_0)} < 1$.\\
		Fix some $B_r(z_0)$. Let $z \in B_r(z_0)$. For any $\nu \in \mathbb{N}_0$ and $q := \abs[0]{z - z_0}/r$ we have
		\begin{equation}
			\max_{\zeta \in \partial B_r(z_0)}\abs[3]{\del{\frac{z - z_0}{\zeta - z_0}}^\nu} = \frac{\abs[0]{z - z_0}^\nu}{\min_{\zeta \in \partial B_r(z_0)}\abs[0]{\zeta - z_0}^\nu} = \del[3]{\frac{\abs[0]{z - z_0}}{r}}^\nu = q^\nu
		\end{equation}
		Hence if we define $f_\nu: \partial B_r(z_0) \to \mathbb{C}$ by
		\begin{equation}
			f_\nu(\zeta) := \del{\frac{z - z_0}{\zeta - z_0}}^\nu \qquad \nu \in \mathbb{N}_0
		\end{equation}
		\noindent and let $\zeta \in \partial B_r(z_0)$ fixed, we have
		\begin{equation}
			\sum_{\nu = 0}^\infty \abs[0]{f_\nu}_{\partial B_r(z_0)} = \sum_{\nu = 0}^\infty q^\nu < \infty
		\end{equation}
		\noindent since $\partial B_r(z_0)$ is compact and $q < 1$. Thus the series (\ref{eq:ser_1}) is normally convergent. If $z \in \mathbb{C}\setminus \overline{B_r}(z_0)$, we have
		\begin{equation}
			\max_{\zeta \in \partial B_r(z_0)}\abs[3]{\del{\frac{\zeta - z_0}{z - z_0}}^\nu} = \frac{\max_{\zeta \in \partial B_r(z_0)}\abs[0]{\zeta - z_0}^\nu}{\abs[0]{z - z_0}^\nu} = \del[3]{\frac{r}{\abs[0]{z - z_0}}}^\nu = p^\nu
		\end{equation}

		\noindent for $p := r/\abs[0]{z - z_0}$. Observe again that $p < 1$ since $\abs[0]{z - z_0} > r$ by assumption. Hence we conclude similarly as in the previous case that the series (\ref{eq:ser_2}) is normally convergent.
	\end{proof}

	\begin{proposition}
		For $z_0 \in \mathbb{C}$ we have
		\begin{align}
			\frac{1}{2\pi i}\int_{\partial B_r(z_0)} \frac{\d \zeta}{\zeta - z} = \begin{cases}
				1 & z \in B_r(z_0),\\
				0 & z \in \mathbb{C} \setminus \overline{B_r}(z_0).
			\end{cases}
		\end{align}
	\end{proposition}

	\begin{proof}
		Consider first the case where $z \in B_r(z_0)$. Then lemma \ref{lem:geo} yields
		\begin{align*}
			\int_{\partial B_r(z_0)} \frac{\d \zeta}{\zeta - z} &= \int_{\partial B_r(z_0)} \frac{1}{\zeta - z_0}\sum_{\nu = 0}^\infty \del{\frac{z - z_0}{\zeta - z_0}}^{\nu} \d \zeta\\
			&= \sum_{\nu = 0}^\infty\int_{\partial B_r(z_0)} \frac{1}{\zeta - z_0}\del{\frac{z - z_0}{\zeta - z_0}}^{\nu} \d \zeta\\
			&= \sum_{\nu = 0}^\infty (z - z_0)^\nu \int_{\partial B_r(z_0)} \frac{\d\zeta}{(\zeta - z_0)^{\nu + 1}}\\
			&= 2\pi i
		\end{align*}

		\noindent since for $n \in \mathbb{Z}$
		\begin{align}
			\int_{\partial B_r(z_0)} \del[0]{\zeta - z_0}^n \d\zeta = \begin{cases}
				0 & n \neq -1,\\
				2\pi  & n = -1.
			\end{cases}
		\end{align}

		In the case $z \in \mathbb{C} \setminus \overline{B_r}(z_0)$ we get
		\begin{align*}
			\int_{\partial B_r(z_0)} \frac{\d \zeta}{\zeta - z} &= -\int_{\partial B_r(z_0)} \frac{1}{z - z_0} \sum_{\nu = 0}^\infty \del{\frac{\zeta - z_0}{z - z_0}}^{\nu} \d \zeta\\
			&= -\sum_{\nu = 0}^\infty \frac{1}{(z - z_0)^{\nu + 1}}\int_{\partial B_r(z_0)} (\zeta - z_0)^\nu \d \zeta\\
			&= 0. 
		\end{align*}
	\end{proof}
\end{enumerate}
%\originalsectionstyle
\printbibliography
\end{document}
