%%%%%%%%%%%%%%%%%%%%%%%%%%%%%%%%%%%%%%%%%%%%%%%%%%%%%%%%%%%%%%%%%%%%%%%%%%
%Author:																 %
%-------																 %
%Yannis Baehni at University of Zurich									 %
%baehni.yannis@uzh.ch													 %
%																		 %
%Version log:															 %
%------------															 %
%06/02/16 . Basic structure												 %
%04/08/16 . Layout changes including section, contents, abstract.		 %
%%%%%%%%%%%%%%%%%%%%%%%%%%%%%%%%%%%%%%%%%%%%%%%%%%%%%%%%%%%%%%%%%%%%%%%%%%

%Page Setup
\documentclass[
	11pt, 
	oneside, 
	a4paper,
	reqno,
	final
]{amsart}

\usepackage[
	left = 3cm, 
	right = 3cm, 
	top = 3cm, 
	bottom = 3cm
]{geometry}

%Headers and footers
\usepackage{fancyhdr}
	\pagestyle{fancy}
	%Clear fields
	\fancyhf{}
	%Header right
	\fancyhead[R]{
		\footnotesize
		Yannis B\"{a}hni\\
		\href{mailto:yannis.baehni@uzh.ch}{yannis.baehni@uzh.ch}
	}
	%Header left
	\fancyhead[L]{
		\footnotesize
		MAT604: Complex Analysis\\
		Spring Semester 2017
	}
	%Page numbering in footer
	\fancyfoot[C]{\thepage}
	%Separation line header and footer
	\renewcommand{\headrulewidth}{0.4pt}
	%\renewcommand{\footrulewidth}{0.4pt}
	
	\setlength{\headheight}{19pt} 

%Title
\usepackage[foot]{amsaddr}
%\usepackage{mathptmx}
\usepackage{xspace}
\makeatletter
\def\@textbottom{\vskip \z@ \@plus 1pt}
\let\@texttop\relax
\usepackage{etoolbox}
\patchcmd{\abstract}{\scshape\abstractname}{\textbf{\abstractname}}{}{}

%Switching commands for different section formats
%Mainsectionsytle
\newcommand{\mainsectionstyle}{%
  	\renewcommand{\@secnumfont}{\bfseries}
  	\renewcommand\section{\@startsection{section}{1}%
    	\z@{.5\linespacing\@plus.7\linespacing}{-.5em}%
    	{\normalfont\bfseries}}%
	\renewcommand\subsection{\@startsection{subsection}{2}%
    	\z@{.5\linespacing\@plus.7\linespacing}{-.5em}%
    	{\normalfont\bfseries}}%
	\renewcommand\subsubsection{\@startsection{subsubsection}{3}%
    	\z@{.5\linespacing\@plus.7\linespacing}{-.5em}%
    	{\normalfont\bfseries}}%
}
\newcommand{\originalsectionstyle}{%
\def\@secnumfont{\bfseries}%\mdseries
\def\section{\@startsection{section}{1}%
  \z@{.7\linespacing\@plus\linespacing}{.5\linespacing}%
  {\normalfont\bfseries\centering}}
}
%Formatting title of TOC
\renewcommand{\contentsnamefont}{\bfseries}
%Table of Contents
\setcounter{tocdepth}{3}

% Add bold to \section titles in ToC and remove . after numbers
\renewcommand{\tocsection}[3]{%
  \indentlabel{\@ifnotempty{#2}{\bfseries\ignorespaces#1 #2\quad}}\bfseries#3}
% Remove . after numbers in \subsection
\renewcommand{\tocsubsection}[3]{%
  \indentlabel{\@ifnotempty{#2}{\ignorespaces#1 #2\quad}}#3}
\let\tocsubsubsection\tocsubsection% Update for \subsubsection
%...

\newcommand\@dotsep{4.5}
\def\@tocline#1#2#3#4#5#6#7{\relax
  \ifnum #1>\c@tocdepth % then omit
  \else
    \par \addpenalty\@secpenalty\addvspace{#2}%
    \begingroup \hyphenpenalty\@M
    \@ifempty{#4}{%
      \@tempdima\csname r@tocindent\number#1\endcsname\relax
    }{%
      \@tempdima#4\relax
    }%
    \parindent\z@ \leftskip#3\relax \advance\leftskip\@tempdima\relax
    \rightskip\@pnumwidth plus1em \parfillskip-\@pnumwidth
    #5\leavevmode\hskip-\@tempdima{#6}\nobreak
    \leaders\hbox{$\m@th\mkern \@dotsep mu\hbox{.}\mkern \@dotsep mu$}\hfill
    \nobreak
    \hbox to\@pnumwidth{\@tocpagenum{\ifnum#1=1\bfseries\fi#7}}\par% <-- \bfseries for \section page
    \nobreak
    \endgroup
  \fi}
\AtBeginDocument{%
\expandafter\renewcommand\csname r@tocindent0\endcsname{0pt}
}
\def\l@subsection{\@tocline{2}{0pt}{2.5pc}{5pc}{}}
\def\l@subsubsection{\@tocline{2}{0pt}{4.5pc}{5pc}{}}
\makeatother

\advance\footskip0.4cm
\textheight=54pc    %a4paper
\textheight=50.5pc %letterpaper
\advance\textheight-0.4cm
\calclayout

%Font settings
%\usepackage{anyfontsize}
%Footnote settings
%\usepackage{mathptmx}
\usepackage{footmisc}
%	\renewcommand*{\thefootnote}{\fnsymbol{footnote}}
\usepackage{commath}
%Further math environments
%Further math fonts (loads amsfonts implicitely)
\usepackage{amssymb}
%Redefinition of \text
%\usepackage{amstext}
\usepackage{upref}
%Graphics
%\usepackage{graphicx}
%\usepackage{caption}
%\usepackage{subcaption}
%Frames
\usepackage{mdframed}
\allowdisplaybreaks
%\usepackage{interval}
\newcommand{\toup}{%
  \mathrel{\nonscript\mkern-1.2mu\mkern1.2mu{\uparrow}}%
}
\newcommand{\todown}{%
  \mathrel{\nonscript\mkern-1.2mu\mkern1.2mu{\downarrow}}%
}
\AtBeginDocument{\renewcommand*\d{\mathop{}\!\mathrm{d}}}
\renewcommand{\Re}{\operatorname{Re}}
\renewcommand{\Im}{\operatorname{Im}}
\DeclareMathOperator\Log{Log}
\DeclareMathOperator\Arg{Arg}
\DeclareMathOperator\sech{sech}
\DeclareMathOperator*\esssup{ess.sup}
%\usepackage{hhline}
%\usepackage{booktabs} 
%\usepackage{array}
%\usepackage{xfrac} 
%\everymath{\displaystyle}
%Enumerate
\usepackage{tikz}
\usetikzlibrary{external}
\tikzexternalize % activate!
\usetikzlibrary{patterns}
\pgfdeclarepatternformonly{adjusted lines}{\pgfqpoint{-1pt}{-1pt}}{\pgfqpoint{40pt}{40pt}}{\pgfqpoint{39pt}{39pt}}%
{
  \pgfsetlinewidth{.8pt}
  \pgfpathmoveto{\pgfqpoint{0pt}{0pt}}
  \pgfpathlineto{\pgfqpoint{39.1pt}{39.1pt}}
  \pgfusepath{stroke}
}
\usepackage{enumitem} 
%\renewcommand{\labelitemi}{$\bullet$}
%\renewcommand{\labelitemii}{$\ast$}
%\renewcommand{\labelitemiii}{$\cdot$}
%\renewcommand{\labelitemiv}{$\circ$}
%Colors
%\usepackage{color}
%\usepackage[cmtip, all]{xy}
%Theorems
\newtheoremstyle{bold}              	 %Name
  {}                                     %Space above
  {}                                     %Space below
  {\itshape}		                     %Body font
  {}                                     %Indent amount
  {\scshape}                             %Theorem head font
  {.}                                    %Punctuation after theorem head
  { }                                    %Space after theorem head, ' ', 
  										 %	or \newline
  {} 
\theoremstyle{bold}
\newtheorem*{definition*}{Definition}
\newtheorem{definition}{Definition}[section]
\newtheorem*{lemma*}{Lemma}
\newtheorem{lemma}{Lemma}[section]
\newtheorem{Proof}{Proof}[section]
\newtheorem{proposition}{Proposition}[section]
\newtheorem{properties}{Properties}[section]
\newtheorem{corollary}{Corollary}[section]
\newtheorem*{theorem*}{Theorem}
\newtheorem{theorem}{Theorem}[section]
\newtheorem{example}{Example}[section]
\newtheorem*{remark*}{Remark}
\newtheorem{remark}{Remark}[section]
%German non-ASCII-Characters
%Graphics-Tool
%\usepackage{tikz}
%\usepackage{tikzscale}
%\usepackage{bbm}
%\usepackage{bera}
%Listing-Setup
%Bibliographie
\usepackage[backend=bibtex, style=alphabetic]{biblatex}
%\usepackage[babel, german = swiss]{csquotes}
\bibliography{Bibliography}
%PDF-Linking
%\usepackage[hyphens]{url}
\usepackage[bookmarksopen=true,bookmarksnumbered=true]{hyperref}
%\PassOptionsToPackage{hyphens}{url}\usepackage{hyperref}
\hypersetup{
  colorlinks   = true, %Colours links instead of ugly boxes
  urlcolor     = blue, %Colour for external hyperlinks
  linkcolor    = blue, %Colour of internal links
  citecolor    = blue %Colour of citations
}
%Weierstrass-P symbol for power set
\newcommand{\powerset}{\raisebox{.15\baselineskip}{\Large\ensuremath{\wp}}}


\title{Solutions Sheet 6}
\author{Yannis B\"{a}hni}
\address[Yannis B\"{a}hni]{University of Zurich, R\"{a}mistrasse 71, 8006 Zurich}
\email[Yannis B\"{a}hni]{\href{mailto:yannis.baehni@uzh.ch}{yannis.baehni@uzh.ch}}

\begin{document}
\maketitle
\thispagestyle{fancy}
\begin{enumerate}[label = \textbf{Exercise \arabic*.},wide = 0pt, itemsep=1.5ex]
	\item
		~
		\begin{enumerate}[label = (\alph*),wide = 0pt, itemsep=1.5ex]
			\item Let $r \in \mathbb{R}_{>0} \setminus \cbr[0]{1}$. Partial fraction decomposition yields
				\begin{align}
					\int_{\partial B_r(0)} \frac{2\zeta - 1}{\zeta(\zeta - 1)}\d \zeta = \int_{\partial B_r(0)} \frac{\d\zeta}{\zeta} + \int_{\partial B_r(0)}\frac{\d\zeta}{\zeta - 1} = \begin{cases}
						4\pi i & 1 < r,\\
						2\pi i & 0 < r < 1.
					\end{cases}
				\end{align} 
			\item The solution of this exercise is based on the following proposition (see \cite[165]{remmert2002funktionentheorie}).
				\begin{proposition}
					Let $D \subseteq \mathbb{C}$ be open and $f \in \mathscr{C}(D)$. For a function $F: D \to \mathbb{C}$ we have that $F$ is holomorphic in $D$ and $F' = f$ if and only if for all paths $\gamma$ in $D$ we have
					\begin{equation}
						\int_{\gamma} f \d \zeta = F(\gamma(b)) - F(\gamma(a)).
					\end{equation}
				\end{proposition}
				\begin{enumerate}[label = (\roman*),wide = 0pt, itemsep=1.5ex]
					\item Clearly $f_1 \in \mathscr{C}(\mathbb{C})$ as a composition of continuous functions. Define $F_1: \mathbb{C} \to \mathbb{C}$ by
						\begin{equation}
							F_1(z) := \frac{1}{1 + i}\sin\del[0]{(1 + i)z}.
						\end{equation}

						$F_1$ is clearly entire since
						\begin{equation}
							F_1'(z) = \frac{1}{1 + i}\sin'\del[0]{(1 + i)z} = \cos\del[0]{(1 + i)z} = f_1(z).
						\end{equation}

						\noindent exists for all $z \in \mathbb{C}$ since 
						\begin{equation}
							\sin'(z) = \frac{1}{2i}\del[1]{\del[0]{e^{iz}}' - \del[0]{e^{-iz}}'} = \frac{1}{2i}\del[0]{ie^{iz} + ie^{-iz}} = \cos(z).
						\end{equation}

						Hence for any path $\gamma$ starting at $z_0$ and ending at $z_1$ we have
						\begin{equation}
							\int_\gamma f_1 \d \zeta = F(2i) - F(1 + i) = \frac{1}{1 + i}\del[0]{\sin(2i - 2) - \sin(2i)}. 
						\end{equation}
					\item Again clearly $f_2 \in \mathscr{C}(\mathbb{C}^\times)$. Define $F_2: \mathbb{C}^\times \to \mathbb{C}$ by
						\begin{equation}
							F_2(z) := \frac{i}{3}z^3 + z + 2i\frac{1}{z}.
						\end{equation}

						Then we have $F_2' = f_2$ and so
						\begin{equation}
							\int_\gamma f_2 \d \zeta = F_2(2i) - F_2(1 + i) = \frac{7}{3} + \frac{2}{3}i.
						\end{equation}
					\item Again $f_3 \in \mathscr{C}(\mathbb{C} \setminus \cbr[0]{-1})$. Define $F_3: \mathbb{C} \setminus \cbr[0]{-1} \to \mathbb{C}$ by
						\begin{equation}
							F_3(z) := -\frac{1}{2}\frac{1}{(z + 1)^2}.
						\end{equation}
						Clearly $F_3' = f_3$ and so
						\begin{equation}
							\int_\gamma f_3 \d \zeta = F_3(2i) - F_3(1 + i) = \frac{3}{25}. 
						\end{equation}

					\item Again $f_4 \in \mathscr{C}(\mathbb{C})$. Define $F_4: \mathbb{C} \to \mathbb{C}$ by
						\begin{equation}
							F_4(z) := \frac{1}{2i}e^{iz^2}.
						\end{equation}

						Clearly $F_4' = f_4$ and thus we get
						\begin{equation}
							\int_\gamma f_4 \d \zeta = F_4(2i) - F_4(1 + i) = \frac{1}{2i}\del[0]{e^{-4i} - e^{-2}}.
						\end{equation}
				\end{enumerate}
		\end{enumerate}

	\item 
		\begin{proposition}[Zentrierungslemma for Rectangles]
			Let $D \subseteq \mathbb{C}$ be open and $f: D \to \mathbb{C}$ holomorphic in $D$. Furthermore let $R \subseteq D$ be a rectangle in $D$ such that $\overline{R} \subseteq D$. Let $z \in R$. If $B_r(z) \subseteq R$, we have 
			\begin{equation}
				\int_{\partial R} f\d \zeta = \int_{\partial B_r(z)} f\d \zeta.
			\end{equation}	
			\label{prop:zentrierungslemma}
		\end{proposition}

		\begin{proof}
			We make use of the labeling on the separate sheet. Clearly
			\begin{equation}
				\partial R = \sbr[0]{z_0,z_1,z_2,z_3,z_0} \qquad \text{and} \qquad 	B_r(z) = \alpha + \beta.
			\end{equation}
			Define
			\begin{align*}
				\gamma_1 &:= \intcc{w_3,z_1} + \intcc{z_1,z_2} + \intcc{z_2,w_0} + \intcc{w_0,w_1} -\alpha + \intcc{w_2,w_3},\\ 
				\gamma_2 &:= \intcc{z_0,w_3} - \intcc{w_2,w_3} - \beta - \intcc{w_0,w_1} + \intcc{w_0,z_3} + \intcc{z_3,z_0}.
			\end{align*}
			Hence
			\begin{align*}
				\int_{\gamma_1 + \gamma_2} f \d \zeta &= \int_{\intcc{z_0,w_3} + \intcc{w_3,z_1} + \intcc{z_1,z_2} + \intcc{z_2,w_0} + \intcc{w_0,z_3} + \intcc{z_3,z_0}}f\d \zeta - \int_{\alpha + \beta}f\d \zeta\\
				&= \int_{\partial R}f\d \zeta - \int_{\partial B_r(z)} f \d\zeta.
			\end{align*}
			Since $\overline{R} \subseteq D$, there exists a rectangle $R'$ with $\overline{R} \subseteq R' \subseteq D$. Clearly a rectangle is a star-shaped domain with any center since it is convex. Hence the Cauchy integral theorem for star-shaped domains implies that
			\begin{equation}
				\int_{\gamma_1} f \d \zeta = 0 \qquad \text{and} \qquad \int_{\gamma_2} f \d \zeta = 0
			\end{equation}
			\noindent since $\gamma_1$ and $\gamma_2$ are closed. Thus
			\begin{equation}
				\int_{\partial R}f\d \zeta - \int_{\partial B_r(z)} f \d\zeta = \int_{\gamma_1 + \gamma_2} f \d \zeta = \int_{\gamma_1} f \d \zeta + \int_{\gamma_2} f \d \zeta = 0. 
			\end{equation}
			This implies
			\begin{equation}
				\int_{\partial R}f\d \zeta = \int_{\partial B_r(z)} f \d\zeta.
			\end{equation}
		\end{proof}

		\begin{theorem}[Cauchy Integral Formula for Rectangles]
			Let $f: D \to \mathbb{C}$ be holomorphic in $D$ and let $R$ be a rectangle in $R$ such that $\overline{R} \subseteq D$. Then we have for any $z \in R$:
			\begin{equation}
				f(z) = \frac{1}{2\pi i}\int_{\partial R}\frac{f(\zeta)}{\zeta - z} \d \zeta.
			\end{equation}
		\end{theorem}

		\begin{proof}
			Let $z \in R$. Define $g: D \to \mathbb{C}$ by  
			\begin{align}
				g(\zeta) := \begin{cases}
					\frac{f(\zeta) - f(z)}{\zeta - z} & \zeta \in D \setminus \cbr[0]{z},\\
					f'(z) & \zeta = z.
				\end{cases}
			\end{align}
			Then $g$ is holomorphic in $D\setminus \cbr[0]{z}$ and continuous at $z$. Now we find $\overline{B_r}(z) \subseteq R$. Since $\overline{B_r}(z)$ is compact we have that $\abs[0]{g} : \overline{B_r}(z) \to \mathbb{R}$ is bounded, say $\abs[0]{g}_{\overline{B_r}(z)} \leq M$. Fix $0 < \varepsilon < r$. Then the standard estimate and proposition \ref{prop:zentrierungslemma} yields
			\begin{equation}
				\abs[3]{\int_{\partial R}g \d \zeta} = \abs[3]{\int_{\partial B_\varepsilon(z)} g \d \zeta} \leq \abs[0]{g}_{\partial B_\varepsilon(z)}2\pi\varepsilon \leq \abs[0]{g}_{B_r(z)}2\pi\varepsilon \leq 2\pi M \varepsilon.
			\end{equation}
			Hence 
			\begin{equation}
				\int_{\partial R}g \d \zeta = 0.
				\label{eq:intg0}
			\end{equation}

			Using again proposition \ref{prop:zentrierungslemma} we find
			\begin{equation}
				\int_{\partial R}\frac{\d \zeta}{\zeta - z} = 2\pi i.
				\label{eq:int}
			\end{equation}
			Putting (\ref{eq:intg0}) and (\ref{eq:int}) together yields
			\begin{equation}
				0 = \int_{\partial R}g \d \zeta= \int_{\partial R} \frac{f(\zeta)}{\zeta - z} \d \zeta - f(z)\int_{\partial R}\frac{\d \zeta}{\zeta - z}= \int_{\partial R} \frac{f(\zeta)}{\zeta - z} \d \zeta - 2\pi i f(z).
			\end{equation}
		\end{proof}
	\item
		~
		\begin{enumerate}[label = (\roman*),wide = 0pt, itemsep=1.5ex]
			\item Consider the function $f: U \to \mathbb{C}$ defined by
				\begin{equation}
					f(z) := \frac{(z - 2)(z^7 + 1)}{z^2(z^4 + 1)}
				\end{equation}
				\noindent where $U$ is $\mathbb{C}$ without the roots of the denominator. The roots are given by $0,e^{\pm i\pi/4}$ and $e^{\pm i3\pi/4}$. Hence $\overline{B_1}(2) \subseteq U$. Since $f$ is holomorphic in $U$ as a well-defined rational function, the Cauchy integral formula yields
	\begin{equation}
		\int_{\partial B_1(2)}\frac{z^7 + 1}{z^2(z^4 + 1)}\d z = \int_{\partial B_1(2)} \frac{f(z)}{z - 2}\d z = 2\pi i f(2) = 0.
	\end{equation}
			\item 
			\item
			\item Partial fraction decomposition yields
				\begin{equation}
					\int_{\partial B_3(0)} \frac{\cos(\pi z)}{z^2 - 1}\d z = \frac{1}{2}\sbr{\int_{\partial B_3(0)}\frac{\cos(\pi z)}{z - 1}\d z - \int_{\partial B_3(0)}\frac{\cos(\pi z)}{z + 1}\d z}
				\end{equation}

				Now $f(z) := \cos(\pi z)$ is entire, and since $\pm 1 \in B_3(0)$ we get
				\begin{equation}
					\int_{\partial B_3(0)} \frac{\cos(\pi z)}{z^2 - 1}\d z = \pi i \sbr[0]{f(1) - f(-1)} = 0.
				\end{equation}
		\end{enumerate}
	\item
		~
		\begin{enumerate}[label = (\alph*),wide = 0pt, itemsep=1.5ex]
			\item Partial fraction decomposition yields for $z\in \mathbb{C} \setminus \overline{\mathbb{E}}$ fixed
				\begin{equation}
					f(z) = -\frac{1}{2\pi i z} \sbr{\int_{\partial \mathbb{E}} \frac{\d \zeta}{\zeta} - \int_{\partial \mathbb{E}}\frac{\d \zeta}{\zeta - z}} = -\frac{1}{2\pi i z}2\pi i = -\frac{1}{z}.
				\end{equation}

			\item This can directly be copied from my solution to exercise $2$ on sheet $4$ with slight improvements.
				\begin{lemma}
					For $z \in \mathbb{C} \setminus \cbr[0]{1}$ and $k \in \mathbb{N}_0$ we have 
		\begin{equation}
			\frac{\d^k}{\d z^k} \frac{1}{1 - z} = \frac{k!}{(1 - z)^{k + 1}}.
		\end{equation}
		\label{lem:ind}
	\end{lemma}

	\begin{proof}
		Proof by induction on $k \in \mathbb{N}_0$. The statement obviously holds for $k = 0$. Assume the statement holds for some $k \in \mathbb{N}_0$. Then we get
		\begin{align*}
			\frac{\d^{k+1}}{\d z^{k+1}} \frac{1}{1 - z} &= \frac{\d}{\d z} \sbr[3]{\frac{\d^k}{\d z^k} \frac{1}{1 - z}}\\
			&= \frac{\d}{\d z} \frac{k!}{(1 - z)^{k + 1}}\\
			&= k!\frac{(k + 1)(1 - z)^k}{(1 - z)^{2k + 2}}\\
			&= \frac{(k + 1)!}{(1 - z)^{k + 2}}.
		\end{align*}
	\end{proof}

	The geometric series $\sum_{\nu = 0}^\infty z^\nu$ converges for all $z \in \mathbb{E}$. Hence by the theorem on the interchangeability of differentiation and summation \cite[110]{remmert2002funktionentheorie} we have that the limit function is differentiable within the radius of convergence (here $R = 1$) and the $k$-th derivative is given by
	\begin{equation}
		\frac{\d^k}{\d z^k}\frac{1}{1 - z} = \frac{\d^k}{\d z^k} \sum_{\nu = 0}^\infty z^\nu = \sum_{\nu \geq k} k! {\nu \choose k} z^{\nu - k} \qquad k\in \mathbb{N}_0, z \in \mathbb{E}.
	\end{equation}
 		\end{enumerate}
\end{enumerate}
%\originalsectionstyle
\printbibliography
\end{document}
