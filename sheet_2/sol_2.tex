%%%%%%%%%%%%%%%%%%%%%%%%%%%%%%%%%%%%%%%%%%%%%%%%%%%%%%%%%%%%%%%%%%%%%%%%%%
%Author:																 %
%-------																 %
%Yannis Baehni at University of Zurich									 %
%baehni.yannis@uzh.ch													 %
%																		 %
%Version log:															 %
%------------															 %
%06/02/16 . Basic structure												 %
%04/08/16 . Layout changes including section, contents, abstract.		 %
%%%%%%%%%%%%%%%%%%%%%%%%%%%%%%%%%%%%%%%%%%%%%%%%%%%%%%%%%%%%%%%%%%%%%%%%%%

%Page Setup
\documentclass[
	11pt, 
	oneside, 
	a4paper,
	reqno,
	final
]{amsart}

\usepackage[
	left = 3cm, 
	right = 3cm, 
	top = 3cm, 
	bottom = 3cm
]{geometry}

%Headers and footers
\usepackage{fancyhdr}
	\pagestyle{fancy}
	%Clear fields
	\fancyhf{}
	%Header right
	\fancyhead[R]{
		\footnotesize
		Yannis B\"{a}hni\\
		\href{mailto:yannis.baehni@uzh.ch}{yannis.baehni@uzh.ch}
	}
	%Header left
	\fancyhead[L]{
		\footnotesize
		MAT604: Complex Analysis\\
		Spring Semester 2017
	}
	%Page numbering in footer
	\fancyfoot[C]{\thepage}
	%Separation line header and footer
	\renewcommand{\headrulewidth}{0.4pt}
	%\renewcommand{\footrulewidth}{0.4pt}
	
	\setlength{\headheight}{19pt} 

%Title
\usepackage[foot]{amsaddr}
%\usepackage{mathptmx}
\usepackage{xspace}
\makeatletter
\def\@textbottom{\vskip \z@ \@plus 1pt}
\let\@texttop\relax
\usepackage{etoolbox}
\patchcmd{\abstract}{\scshape\abstractname}{\textbf{\abstractname}}{}{}

%Switching commands for different section formats
%Mainsectionsytle
\newcommand{\mainsectionstyle}{%
  	\renewcommand{\@secnumfont}{\bfseries}
  	\renewcommand\section{\@startsection{section}{1}%
    	\z@{.5\linespacing\@plus.7\linespacing}{-.5em}%
    	{\normalfont\bfseries}}%
	\renewcommand\subsection{\@startsection{subsection}{2}%
    	\z@{.5\linespacing\@plus.7\linespacing}{-.5em}%
    	{\normalfont\bfseries}}%
	\renewcommand\subsubsection{\@startsection{subsubsection}{3}%
    	\z@{.5\linespacing\@plus.7\linespacing}{-.5em}%
    	{\normalfont\bfseries}}%
}
\newcommand{\originalsectionstyle}{%
\def\@secnumfont{\bfseries}%\mdseries
\def\section{\@startsection{section}{1}%
  \z@{.7\linespacing\@plus\linespacing}{.5\linespacing}%
  {\normalfont\bfseries\centering}}
}
%Formatting title of TOC
\renewcommand{\contentsnamefont}{\bfseries}
%Table of Contents
\setcounter{tocdepth}{3}

% Add bold to \section titles in ToC and remove . after numbers
\renewcommand{\tocsection}[3]{%
  \indentlabel{\@ifnotempty{#2}{\bfseries\ignorespaces#1 #2\quad}}\bfseries#3}
% Remove . after numbers in \subsection
\renewcommand{\tocsubsection}[3]{%
  \indentlabel{\@ifnotempty{#2}{\ignorespaces#1 #2\quad}}#3}
\let\tocsubsubsection\tocsubsection% Update for \subsubsection
%...

\newcommand\@dotsep{4.5}
\def\@tocline#1#2#3#4#5#6#7{\relax
  \ifnum #1>\c@tocdepth % then omit
  \else
    \par \addpenalty\@secpenalty\addvspace{#2}%
    \begingroup \hyphenpenalty\@M
    \@ifempty{#4}{%
      \@tempdima\csname r@tocindent\number#1\endcsname\relax
    }{%
      \@tempdima#4\relax
    }%
    \parindent\z@ \leftskip#3\relax \advance\leftskip\@tempdima\relax
    \rightskip\@pnumwidth plus1em \parfillskip-\@pnumwidth
    #5\leavevmode\hskip-\@tempdima{#6}\nobreak
    \leaders\hbox{$\m@th\mkern \@dotsep mu\hbox{.}\mkern \@dotsep mu$}\hfill
    \nobreak
    \hbox to\@pnumwidth{\@tocpagenum{\ifnum#1=1\bfseries\fi#7}}\par% <-- \bfseries for \section page
    \nobreak
    \endgroup
  \fi}
\AtBeginDocument{%
\expandafter\renewcommand\csname r@tocindent0\endcsname{0pt}
}
\def\l@subsection{\@tocline{2}{0pt}{2.5pc}{5pc}{}}
\def\l@subsubsection{\@tocline{2}{0pt}{4.5pc}{5pc}{}}
\makeatother

\advance\footskip0.4cm
\textheight=54pc    %a4paper
\textheight=50.5pc %letterpaper
\advance\textheight-0.4cm
\calclayout

%Font settings
%\usepackage{anyfontsize}
%Footnote settings
%\usepackage{mathptmx}
\usepackage{footmisc}
%	\renewcommand*{\thefootnote}{\fnsymbol{footnote}}
\usepackage{commath}
%Further math environments
%Further math fonts (loads amsfonts implicitely)
\usepackage{amssymb}
%Redefinition of \text
%\usepackage{amstext}
\usepackage{upref}
%Graphics
%\usepackage{graphicx}
%\usepackage{caption}
%\usepackage{subcaption}
%Frames
\usepackage{mdframed}
\allowdisplaybreaks
%\usepackage{interval}
\newcommand{\toup}{%
  \mathrel{\nonscript\mkern-1.2mu\mkern1.2mu{\uparrow}}%
}
\newcommand{\todown}{%
  \mathrel{\nonscript\mkern-1.2mu\mkern1.2mu{\downarrow}}%
}
\AtBeginDocument{\renewcommand*\d{\mathop{}\!\mathrm{d}}}
\renewcommand{\Re}{\operatorname{Re}}
\renewcommand{\Im}{\operatorname{Im}}
\DeclareMathOperator\Log{Log}
\DeclareMathOperator\Arg{Arg}
\DeclareMathOperator\sech{sech}
\DeclareMathOperator*\esssup{ess.sup}
%\usepackage{hhline}
%\usepackage{booktabs} 
%\usepackage{array}
%\usepackage{xfrac} 
%\everymath{\displaystyle}
%Enumerate
\usepackage{tikz}
\usetikzlibrary{external}
\tikzexternalize % activate!
\usetikzlibrary{patterns}
\pgfdeclarepatternformonly{adjusted lines}{\pgfqpoint{-1pt}{-1pt}}{\pgfqpoint{40pt}{40pt}}{\pgfqpoint{39pt}{39pt}}%
{
  \pgfsetlinewidth{.8pt}
  \pgfpathmoveto{\pgfqpoint{0pt}{0pt}}
  \pgfpathlineto{\pgfqpoint{39.1pt}{39.1pt}}
  \pgfusepath{stroke}
}
\usepackage{enumitem} 
%\renewcommand{\labelitemi}{$\bullet$}
%\renewcommand{\labelitemii}{$\ast$}
%\renewcommand{\labelitemiii}{$\cdot$}
%\renewcommand{\labelitemiv}{$\circ$}
%Colors
%\usepackage{color}
%\usepackage[cmtip, all]{xy}
%Theorems
\newtheoremstyle{bold}              	 %Name
  {}                                     %Space above
  {}                                     %Space below
  {\itshape}		                     %Body font
  {}                                     %Indent amount
  {\scshape}                             %Theorem head font
  {.}                                    %Punctuation after theorem head
  { }                                    %Space after theorem head, ' ', 
  										 %	or \newline
  {} 
\theoremstyle{bold}
\newtheorem*{definition*}{Definition}
\newtheorem{definition}{Definition}[section]
\newtheorem*{lemma*}{Lemma}
\newtheorem{lemma}{Lemma}[section]
\newtheorem{Proof}{Proof}[section]
\newtheorem{proposition}{Proposition}[section]
\newtheorem{properties}{Properties}[section]
\newtheorem{corollary}{Corollary}[section]
\newtheorem*{theorem*}{Theorem}
\newtheorem{theorem}{Theorem}[section]
\newtheorem{example}{Example}[section]
\newtheorem*{remark*}{Remark}
\newtheorem{remark}{Remark}[section]
%German non-ASCII-Characters
%Graphics-Tool
%\usepackage{tikz}
%\usepackage{tikzscale}
%\usepackage{bbm}
%\usepackage{bera}
%Listing-Setup
%Bibliographie
\usepackage[backend=bibtex, style=alphabetic]{biblatex}
%\usepackage[babel, german = swiss]{csquotes}
\bibliography{Bibliography}
%PDF-Linking
%\usepackage[hyphens]{url}
\usepackage[bookmarksopen=true,bookmarksnumbered=true]{hyperref}
%\PassOptionsToPackage{hyphens}{url}\usepackage{hyperref}
\hypersetup{
  colorlinks   = true, %Colours links instead of ugly boxes
  urlcolor     = blue, %Colour for external hyperlinks
  linkcolor    = blue, %Colour of internal links
  citecolor    = blue %Colour of citations
}
%Weierstrass-P symbol for power set
\newcommand{\powerset}{\raisebox{.15\baselineskip}{\Large\ensuremath{\wp}}}


\title{Solutions Sheet 2}
\author{Yannis B\"{a}hni}
\address[Yannis B\"{a}hni]{University of Zurich, R\"{a}mistrasse 71, 8006 Zurich}
\email[Yannis B\"{a}hni]{\href{mailto:yannis.baehni@uzh.ch}{yannis.baehni@uzh.ch}}

\begin{document}
\maketitle
\thispagestyle{fancy}
\begin{enumerate}[label = \textbf{Exercise \arabic*.},wide = 0pt, itemsep=1.5ex]
	\item 
		~
		\begin{enumerate}[label = (\roman*),wide = 10pt, itemsep=1.5ex]
			\item We have already showed that $(\mathbb{C},\mathcal{Z})$ is not a Hausdorff space (see sheet $1$ exercise $3$), hence not compact. Therefore it is enough to show that any open cover of $(\mathbb{C},\mathcal{Z})$ has a finite subcover. 
					\begin{lemma}
						$(\mathbb{C},\mathcal{Z})$ is quasi-compact.
						\label{lem:finitecover}
					\end{lemma}
				
				\begin{proof}
				Let $(U_\alpha)_{\alpha \in A}$ be an open cover of $(\mathbb{C},\mathcal{Z})$, i.e. 
				\begin{equation}
					\mathbb{C} = \bigcup_{\alpha \in A}U_\alpha \qquad \text{and} \qquad \forall \alpha \in A: U_\alpha \in \mathcal{Z}.
				\end{equation}

				We can explicitely construct a finite subcover. Pick some $\alpha_0 \in A$ such that $U_{\alpha_0} \neq \varnothing$. Since $U_{\alpha_0} \in \mathcal{Z}$, $U_{\alpha_0}^c$ is finite, i.e. $U_{\alpha_0}^c = \cbr[0]{z_1,\dots,z_n} \subseteq \mathbb{C}$. Thus we can write 
		\begin{equation}
			\mathbb{C} = U_{\alpha_0} \cup U_{\alpha_0}^c = U_{\alpha_0} \cup \cbr[0]{z_1,\dots,z_n}.
		\end{equation}

		Since $\mathbb{C} = \bigcup_{\alpha \in A}U_\alpha$, we find $\alpha_i \in A$ for $i = 1,\dots,n$ such that $z_i \in U_{\alpha_i}$. Hence $(U_{\alpha_\nu})_{\nu \in \cbr[0]{0,\dots,n}}$ is a finite subcover of $(U_\alpha)_{\alpha \in A}$. Since the construction was general, we conclude that $(\mathbb{C},\mathcal{Z})$ is quasi-compact. 
	\end{proof}
\item The reasoning is similar to part i).
		\begin{lemma}
			$(\cbr[0]{z_0}^c, \cbr[0]{z_0}^c \cap \mathcal{Z})$ is quasi-compact.
			\label{lem:z_0_quasi}
		\end{lemma}

		\begin{proof}
			Let $(U_\alpha)_{\alpha \in A}$ be an open cover of $(\cbr[0]{z_0}^c, \cbr[0]{z_0}^c \cap \mathcal{Z})$, i.e. 
				\begin{equation}
					\cbr[0]{z_0}^c = \bigcup_{\alpha \in A}U_\alpha \qquad \text{and} \qquad \forall \alpha \in A: U_\alpha \in \cbr[0]{z_0}^c \cap \mathcal{Z}.
				\end{equation}

				We can explicitely construct a finite subcover. Pick some $\alpha_0 \in A$ such that $U_{\alpha_0} \ne \varnothing$. Since $U_{\alpha_0} \in \cbr[0]{z_0}^c \cap \mathcal{Z}$, there exists $V \in \mathcal{Z}$ such that $U_{\alpha_0} = \cbr[0]{z_0}^c \cap V$. By considering the relative complement
				\begin{equation}
					U_{\alpha_0}^c = \cbr[0]{z_0}^c \cap \del[0]{\cbr[0]{z_0}^c \cap V}^c = \cbr[0]{z_0}^c \cap (\cbr[0]{z_0} \cup V^c) = \cbr[0]{z_0}^c \cap V^c \subseteq V^c
				\end{equation}
				\noindent and using the fact that $V^c$ is finite we conclude that $U_{\alpha_0}^c$ is finite, i.e. $U_{\alpha_0}^c = \cbr[0]{z_1,\dots,z_n} \subseteq \cbr[0]{z_0}^c$. Thus we can write 
		\begin{equation}
			\cbr[0]{z_0}^c = U_{\alpha_0} \cup U_{\alpha_0}^c = U_{\alpha_0} \cup \cbr[0]{z_1,\dots,z_n}.
		\end{equation}

		Since $\cbr[0]{z_0}^c = \bigcup_{\alpha \in A}U_\alpha$, we find $\alpha_i \in A$ for $i = 1,\dots,n$ such that $z_i \in U_{\alpha_i}$. Hence $(U_{\alpha_\nu})_{\nu \in \cbr[0]{0,\dots,n}}$ is a finite subcover of $(U_\alpha)_{\alpha \in A}$. Since the construction was general, we conclude that $(\cbr[0]{z_0}^c, \cbr[0]{z_0}^c \cap \mathcal{Z})$ is quasi-compact. 
	\end{proof}

		\begin{lemma}
			$(\cbr[0]{z_0}^c, \cbr[0]{z_0}^c \cap \mathcal{Z})$ is not Hausdorff.
			\label{lem:z_0_not_haus}
		\end{lemma}

		\begin{proof}
			Towards a contradiction assume that $(\cbr[0]{z_0}^c, \cbr[0]{z_0}^c \cap \mathcal{Z})$ is Hausdorff. Thus for $p,q \in \cbr[0]{z_0}^c$ there exists open neighbourhoods $U$ and $V$ of $p$ and $q$ respectively such that $U \cap V = \varnothing$. From the latter it follows that $U \subseteq V^c$. Since $V$ is open we find $W_1 \in \mathcal{Z}$ such that $V = \cbr[0]{z_0}^c \cap W_1$. Hence taking relative complements yields 
			\begin{equation*}
				V^c = \cbr[0]{z_0}^c \cap (\cbr[0]{z_0}^c \cap W_1)^c = \cbr[0]{z_0}^c \cap W_1^c \subseteq W_1^c
			\end{equation*}

			\noindent So $V^c$ is finite and therefore also $U$. Since $U$ is open we have that there exists $W_2 \in \mathcal{Z}$ such that $U = \cbr[0]{z_0}^c \cap W_2$. Taking again relative complements yields 
			\begin{equation*}
				U^c = \cbr[0]{z_0}^c \cap (\cbr[0]{z_0}^c \cap W_2)^c = \cbr[0]{z_0}^c \cap W_2^c \subseteq W_2^c
			\end{equation*}

			\noindent So $U^c$ is also finite. Therefore the decomposition $\cbr[0]{z_0}^c = U \cup U^c$ implies that $\cbr[0]{z_0}^c$ is finite. Contradiction, since $\abs[0]{\cbr[0]{z_0}^c} \geq \abs[0]{\mathbb{R}} = \mathfrak{c}$, which is clearly not finite.
		\end{proof}

		Therefore by lemma \ref{lem:z_0_quasi} we conclude that $(\cbr[0]{z_0}^c, \cbr[0]{z_0}^c \cap \mathcal{Z})$ is quasi-compact, but from lemma \ref{lem:z_0_not_haus} follows that $(\cbr[0]{z_0}^c, \cbr[0]{z_0}^c \cap \mathcal{Z})$ is not compact.
		
	\item By $\del[0]{\cbr[0]{z_0}^c}^c = \cbr[0]{z_0}$ which is finite immediately follows $\cbr[0]{z_0}^c \in \mathcal{Z}$. But $\cbr[0]{z_0}^c = \mathbb{C} \setminus \cbr[0]{z_0}$ is clearly not finite, thus $\cbr[0]{z_0} \notin \mathcal{Z}$, hence $\cbr[0]{z_0}^c$ cannot be closed. 
	\end{enumerate}

\item Let $G \subseteq \mathbb{C}$ be non-empty and open. We show the equivalences (i) $\Leftrightarrow$ (ii) and (ii) $\Leftrightarrow$ (iii). This is due to the fact that I am aware of the latter equivalence by considering \cite[86]{lee:topological_manifolds:2011} and the first one by \cite[90]{lee:topological_manifolds:2011} and the fact that every open connected subset of $\mathbb{R}^n$ is path-connected. However, working out detailed and appropriate proofs is still alot of work.\\
	Assume that (i) holds. Proof by contradiction. Let $G = G_1 \cup G_2$ for some open sets $G_1,G_2 \subseteq \mathbb{C}$ with $G_1 \cap G_2 = \varnothing$ and $G_1,G_2 \neq G$. Evidently $G_1,G_2 \neq \varnothing$ and thus we find $p \in G \cap G_1$, $q \in G \cap G_2$. Let $\gamma: \intcc{a,b} \to G$ be a path joining $p$ and $q$, i.e. $\gamma(a) = p$ and $\gamma(b) = q$. Since $\gamma$ is continuous, $G \cap G_1$ and $G \cap G_2$ are relatively open in $G$ we have that $\gamma^{-1}(G \cap G_1)$ and $\gamma^{-1}(G \cap G_2)$ are open in $\intcc{a,b}$. Furthermore, since $a \in \gamma^{-1}(G \cap G_1)$ and $b \in \gamma^{-1}(G \cap G_2)$ we have that both preimages are non-empty. By
	\begin{equation*}
		\gamma^{-1}(G \cap G_1) \cup \gamma^{-1}(G \cap G_2) = \gamma^{-1}((G \cap G_1) \cup (G \cap G_2)) = \gamma^{-1}(G) = \intcc{a,b}
	\end{equation*}

	\noindent and
	\begin{equation*}
		\gamma^{-1}(G \cap G_1) \cap \gamma^{-1}(G \cap G_2) = \gamma^{-1}((G \cap G_1) \cap (G \cap G_2)) = \gamma^{-1}(\varnothing) = \varnothing  
	\end{equation*}
	
	\noindent we have that $\gamma^{-1}(G \cap G_1)$ and $\gamma^{-1}(G \cap G_2)$ disconnect $\intcc{a,b}$ which is impossible since a real interval is always connected (see \cite[89]{lee:topological_manifolds:2011}).\\

	Now assume that (ii) holds. Let $z_0 \in G$. Since joinability by paths in $G$ is an equivalence relation (let us denote it simply by $\sim$) define
	\begin{equation}
		G_1 := \sbr[0]{z_0}_\sim.
	\end{equation}
		\begin{lemma}
			$G_1$ is open.
			\label{lem:G_1}
		\end{lemma}

	\begin{proof}
		Let $z_1 \in G_1$. Since $G$ is open we find $\varepsilon > 0$ such that $B_\varepsilon(z_1) \subseteq G$. $B_\varepsilon(z_1)$ is evidently path connected (consider just straight lines joining different points). Since $z_1 \in G_1$, we have that there is a path joining $z_0$ and $z_1$. By concatenating paths, there is a path from $z_0$ to every point in $B_\varepsilon(z_1)$ and thus $B_\varepsilon(z_1) \subseteq G_1$.
	\end{proof}

		\begin{lemma}
			The relative complement $G_1^c$ in $G$ is open.
			\label{lem:G_1^c}
		\end{lemma}

	\begin{proof}
	Let $z_1 \in G_1^c$. Again we find $\varepsilon > 0$ such that $B_\varepsilon(z_1) \subseteq G$ by the openness of $G$. Towards a contradiction assume that $B_\varepsilon(z_1) \cap G_1 \neq \varnothing$. Hence we find $z_2 \in B_\varepsilon(z_1) \cap G_1$. This means, that there is a path joining $z_0$ and $z_2$. But since $B_\varepsilon(z_1)$ is path connected there would be a path joining $z_1$ and $z_0$ which yields a contradiction. Hence $B_\varepsilon(z_1) \subseteq G_1^c$. 
\end{proof}

Since evidently $G = G_1 \cup G_1^c$ and by lemma \ref{lem:G_1} and \ref{lem:G_1^c} $G_1$, $G_1^c$ are open and clearly disjoint, (ii) implies that either $G_1 = G$ or $G_1^c = G$. The latter is impossible since $z_0 \notin G_1^c$. Hence we conclude that $G = G_1$. Thus $G$ is a single equivalence class under joinability by paths, hence path-connected.\\
	Next we show that (ii) $\Rightarrow$ (iii). To be completely rigorous, we state the following lemma which can be found as an exercise in \cite[50]{lee:topological_manifolds:2011}.
	\begin{lemma}
		Let $(X,\mathcal{T})$ be a topological space and $S \subseteq X$. Then $B \subseteq S$ is closed in $S$ if and only if $B = S \cap A$ for some closed set $A$ in $X$.
		\label{lem:subspaceclosed}
	\end{lemma}

	\begin{proof}
		Assume $B \subseteq S$ is closed in $S$. Hence the relative complement $B^c$ is open in $S$. Therefore we have $B^c = S \cap U$ for some open set $U$ in $X$. Thus
		\begin{equation*}
			B = (S \cap U)^c = S \cap (S \cap U)^c = S \cap (S^c \cup U^c) = (S \cap S^c) \cup (S \cap U^c) = S \cap U^c.
		\end{equation*}

		But since $U$ is open in $X$ we have that $U^c$ is closed in $X$. Conversly assume that $B = S \cap A$ for some closed set $A$ in $X$. Taking relative complements yields
		\begin{equation*}
			B^c = (S \cap A)^c = S \cap (S \cap A)^c = S \cap A^c
		\end{equation*}

		\noindent and since $A$ is closed in $X$ we have that $A^c$ is open in $X$ which means $B^c$ is open in $S$.
	\end{proof}

	Now assume that (ii) holds. Let $U \subseteq G$ be a non-empty, open and relatively closed (with respect to $G$) subset. Since $U$ is relatively closed, by lemma \ref{lem:subspaceclosed} there exists a closed set $A \subseteq \mathbb{C}$ such that $U = G \cap A$. Observe, that by
	\begin{equation*}
		U^c = G \cap (G \cap A)^c = G \cap (G^c \cup A^c) = G \cap A^c
	\end{equation*}

	\noindent $U^c$ is open in $\mathbb{C}$ since $G$ and $A^c$ are open in $\mathbb{C}$. Clearly, $G = U \cup U^c$ and $U \cap U^c = \varnothing$. Therefore (ii) implies that either $U = G$ or $U^c = G$ where the latter is impossible since by assumption $U \neq \varnothing$. Hence we conclude that $U = G$.\\
	Finally we show (iii) $\Rightarrow$ (ii). This is equivalent to showing that not (ii) implies not (iii). So we have $G = G_1 \cup G_2$ for some open disjoint sets $G_1, G_2 \subseteq \mathbb{C}$ where $G_1,G_2 \neq G$. Now clearly $\varnothing \subsetneq G_1 \subsetneq G$, $G_1$ is open and $G_1$ is relatively closed in $G$ since
	\begin{equation*}
		G \cap G_2^c = (G_1 \cup G_2) \cap G_2^c = G_1 \cap G_2^c = G_1
	\end{equation*}

	\noindent since $G_2^c$ is closed and $G_1 \cap G_2 = \varnothing$.

\item The proof is given as a sequence of lemmata.
	\begin{lemma}
		The mapping $h$ is well-defined, i.e. $h(\mathbb{H}) \subseteq \mathbb{E}$.
		\label{lem:well_defined}
	\end{lemma}

	\begin{proof}
		 Let $z \in \mathbb{H}$. Then $\Im(z) > 0$ and thus
	\begin{align*}
		\abs[3]{\frac{z - i}{z + i}}^2 &= \frac{(z - i)(\overline{z} + i)}{(z + i)(\overline{z} - i)}\\
		&= \frac{\abs[0]{z}^2 + i(z - \overline{z}) + 1}{\abs[0]{z}^2 + i( \overline{z} - z) + 1}\\
		&= \frac{\abs[0]{z}^2 -2 \Im (z) + 1}{\abs[0]{z}^2 + 2\Im(z) + 1}\\
		&\leq \frac{\abs[0]{z}^2 -2 \Im (z) + 1}{\abs[0]{z}^2 + 1}\\
		&= 1 - \frac{2\Im(z)}{\abs[0]{z}^2 + 1}\\
		& < 1.
	\end{align*}
	\end{proof}
	
	\begin{lemma}
		The mapping $h$ is invertible with inverse
		\begin{equation}
			h^{-1}: \mathbb{E} \to \mathbb{H}, w \mapsto i\frac{1 + w}{1 - w}.
		\end{equation}
		\label{lem:invertible}
	\end{lemma}

	\begin{proof}
	Let $z \in \mathbb{H}$. Then we have 
	\begin{equation}
		1 + h(z) = \frac{2z}{z + i} \qquad \text{and} \qquad 1 - h(z) = \frac{2i}{z + i}.
	\end{equation}

	Therefore
	\begin{equation}
		\frac{1 + h(z)}{1 - h(z)} = \frac{z}{i} \qquad \Leftrightarrow \qquad z = i\frac{1 + h(z)}{1 - h(z)}.
	\end{equation}

	This quotient is well-defined since $h(z) \neq 1$ for $z \in \mathbb{H}$. Thus consider the mapping
	\begin{equation*}
		g: \mathbb{E} \to \mathbb{C}, w \mapsto i\frac{1 + w}{1 - w}.	
	\end{equation*}
	
	Now let $w \in \mathbb{E}$. Then
	\begin{align*}
		\Im\del[3]{i\frac{1 + w}{1 - w}} &= \frac{1}{2i}\sbr[3]{i\frac{1 + w}{1 - w} + i\frac{1 + \overline{w}}{1 - \overline{w}}}\\
		&= \frac{1 - \abs[0]{w}^2}{1 - (w + \overline{w}) + \abs[0]{w}^2}\\
		&= \frac{1 - \abs[0]{w}^2}{1 - 2\Re(w) + \abs[0]{w}^2}\\
		&\geq \frac{1 - \abs[0]{w}^2}{\del[0]{1 + \abs[0]{w}}^2}\\
		&= \frac{1 - \abs[0]{w}}{1 + \abs[0]{w}}\\
		&> 0
	\end{align*}

	\noindent since $\abs[0]{\Re(w)} \leq \abs[0]{w} < 1$. Hence $g(\mathbb{E}) \subseteq \mathbb{H}$. Furthermore, for $z \in \mathbb{H}$ and $w \in \mathbb{E}$ we have
	\begin{equation*}
		g(h(z)) = i\frac{1 + (z - i)/(z + i)}{1 - (z - i)/(z + i)} = z \qquad \text{and} \qquad h(g(w)) = \frac{(1 + w)/(1-w)- 1}{(1 + w)/(1-w) + 1} = w.
	\end{equation*}

	Therefore $g = h^{-1}$.	
	\end{proof}
	
	\begin{lemma}
		It holds that $h \in \mathcal{O}(\mathbb{H})$ and $h^{-1} \in \mathcal{O}(\mathbb{E})$.
		\label{lem:holo}
	\end{lemma}

	\begin{proof}
		$h$ aswell as $h^{-1}$ are well-defined rational functions, hence holomorphic in their respective domains.		
	\end{proof}
	
	By lemma \ref{lem:well_defined}, \ref{lem:invertible} and \ref{lem:holo} we conclude that the Cayley-map is biholomorphic. 

\item
	First we consider the function $f$. We can decompose $f = u + iv$ where $u,v: \mathbb{C} \to \mathbb{R}$ are defined by
	\begin{equation}
		u(x + iy) := x^3y^2 \qquad \text{and} \qquad v(x + iy) := x^2y^3.	
	\end{equation}

	Calculating the partial derivatives yields
	\begin{equation*}
		\pd{u}{x}(x,y) = 3x^2y^2 \qquad \pd{u}{y}(x,y) = 2x^3y \qquad \pd{v}{x}(x,y) = 2xy^3 \qquad \pd{v}{y}(x,y) = 3x^2 y^2.
	\end{equation*}

	The points in which $f$ is complex differentiable are exactly the points in which above partial derivatives fulfill the Cauchy-Riemann equations. Clearly 
	\begin{equation}
		\pd{u}{x}(x,y) = 3x^2 y^2 = \pd{v}{y}(x,y)
	\end{equation}

	\noindent holds for any $x,y \in \mathbb{R}$ and thus for any $z \in \mathbb{R}$. However, for the second equation we get the requirement
	\begin{equation}
		\pd{u}{y}(x,y) = 2x^3y = -2xy^3 = -\pd{v}{x}(x,y). 
	\end{equation}

	\noindent or equivalently $x^3y = -xy^3$. If $x = 0$, then the equality holds for any $y \in \mathbb{R}$ and if $y = 0$ then the equality holds for any $x \in \mathbb{R}$. If $x,y \neq 0$ we get the equality $x^2 = -y^2$ which cannot be true since $y^2 > 0$ and thus $x^2 < 0$. Hence the set in which $f$ is complex differentiable is exactly $\cbr[0]{z \in \mathbb{C} : \Re(z) = 0 \vee \Im(z) = 0}$.\\
	Now consider the function $g$. Similar to the previous part we can write $g = u + iv$ where $u,v: \mathbb{C} \to \mathbb{R}$ are defined by 
	\begin{equation}
		u(x + iy) := e^x\cos y \qquad \text{and} \qquad v(x + iy) := e^x \sin y.	
	\end{equation}

	Calculating the partial derivatives yields
	\begin{align*}
		\pd{u}{x}(x,y) &= e^x\cos y = u(x + iy)\\
		\pd{u}{y}(x,y) &= -e^x\sin y = -v(x + iy)\\
		\pd{v}{x}(x,y) &= e^x\sin y = v(x + iy)\\
		\pd{v}{y}(x,y) &= e^x\cos y = u(x + iy)
	\end{align*}

	\noindent which immediately implies that the Cauchy-Riemann equations holds on $\mathbb{C}$ and thus $g$ is holomorphic in $\mathbb{C}$ or \emph{entire}.

\end{enumerate}
%\originalsectionstyle
\printbibliography
\end{document}
