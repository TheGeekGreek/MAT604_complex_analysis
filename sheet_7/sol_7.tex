%%%%%%%%%%%%%%%%%%%%%%%%%%%%%%%%%%%%%%%%%%%%%%%%%%%%%%%%%%%%%%%%%%%%%%%%%%
%Author:																 %
%-------																 %
%Yannis Baehni at University of Zurich									 %
%baehni.yannis@uzh.ch													 %
%																		 %
%Version log:															 %
%------------															 %
%06/02/16 . Basic structure												 %
%04/08/16 . Layout changes including section, contents, abstract.		 %
%%%%%%%%%%%%%%%%%%%%%%%%%%%%%%%%%%%%%%%%%%%%%%%%%%%%%%%%%%%%%%%%%%%%%%%%%%

%Page Setup
\documentclass[
	11pt, 
	oneside, 
	a4paper,
	reqno,
	final
]{amsart}

\usepackage[
	left = 3cm, 
	right = 3cm, 
	top = 3cm, 
	bottom = 3cm
]{geometry}

%Headers and footers
\usepackage{fancyhdr}
	\pagestyle{fancy}
	%Clear fields
	\fancyhf{}
	%Header right
	\fancyhead[R]{
		\footnotesize
		Yannis B\"{a}hni\\
		\href{mailto:yannis.baehni@uzh.ch}{yannis.baehni@uzh.ch}
	}
	%Header left
	\fancyhead[L]{
		\footnotesize
		MAT604: Complex Analysis\\
		Spring Semester 2017
	}
	%Page numbering in footer
	\fancyfoot[C]{\thepage}
	%Separation line header and footer
	\renewcommand{\headrulewidth}{0.4pt}
	%\renewcommand{\footrulewidth}{0.4pt}
	
	\setlength{\headheight}{19pt} 

%Title
\usepackage[foot]{amsaddr}
%\usepackage{mathptmx}
\usepackage{xspace}
\makeatletter
\def\@textbottom{\vskip \z@ \@plus 1pt}
\let\@texttop\relax
\usepackage{etoolbox}
\patchcmd{\abstract}{\scshape\abstractname}{\textbf{\abstractname}}{}{}

%Switching commands for different section formats
%Mainsectionsytle
\newcommand{\mainsectionstyle}{%
  	\renewcommand{\@secnumfont}{\bfseries}
  	\renewcommand\section{\@startsection{section}{1}%
    	\z@{.5\linespacing\@plus.7\linespacing}{-.5em}%
    	{\normalfont\bfseries}}%
	\renewcommand\subsection{\@startsection{subsection}{2}%
    	\z@{.5\linespacing\@plus.7\linespacing}{-.5em}%
    	{\normalfont\bfseries}}%
	\renewcommand\subsubsection{\@startsection{subsubsection}{3}%
    	\z@{.5\linespacing\@plus.7\linespacing}{-.5em}%
    	{\normalfont\bfseries}}%
}
\newcommand{\originalsectionstyle}{%
\def\@secnumfont{\bfseries}%\mdseries
\def\section{\@startsection{section}{1}%
  \z@{.7\linespacing\@plus\linespacing}{.5\linespacing}%
  {\normalfont\bfseries\centering}}
}
%Formatting title of TOC
\renewcommand{\contentsnamefont}{\bfseries}
%Table of Contents
\setcounter{tocdepth}{3}

% Add bold to \section titles in ToC and remove . after numbers
\renewcommand{\tocsection}[3]{%
  \indentlabel{\@ifnotempty{#2}{\bfseries\ignorespaces#1 #2\quad}}\bfseries#3}
% Remove . after numbers in \subsection
\renewcommand{\tocsubsection}[3]{%
  \indentlabel{\@ifnotempty{#2}{\ignorespaces#1 #2\quad}}#3}
\let\tocsubsubsection\tocsubsection% Update for \subsubsection
%...

\newcommand\@dotsep{4.5}
\def\@tocline#1#2#3#4#5#6#7{\relax
  \ifnum #1>\c@tocdepth % then omit
  \else
    \par \addpenalty\@secpenalty\addvspace{#2}%
    \begingroup \hyphenpenalty\@M
    \@ifempty{#4}{%
      \@tempdima\csname r@tocindent\number#1\endcsname\relax
    }{%
      \@tempdima#4\relax
    }%
    \parindent\z@ \leftskip#3\relax \advance\leftskip\@tempdima\relax
    \rightskip\@pnumwidth plus1em \parfillskip-\@pnumwidth
    #5\leavevmode\hskip-\@tempdima{#6}\nobreak
    \leaders\hbox{$\m@th\mkern \@dotsep mu\hbox{.}\mkern \@dotsep mu$}\hfill
    \nobreak
    \hbox to\@pnumwidth{\@tocpagenum{\ifnum#1=1\bfseries\fi#7}}\par% <-- \bfseries for \section page
    \nobreak
    \endgroup
  \fi}
\AtBeginDocument{%
\expandafter\renewcommand\csname r@tocindent0\endcsname{0pt}
}
\def\l@subsection{\@tocline{2}{0pt}{2.5pc}{5pc}{}}
\def\l@subsubsection{\@tocline{2}{0pt}{4.5pc}{5pc}{}}
\makeatother

\advance\footskip0.4cm
\textheight=54pc    %a4paper
\textheight=50.5pc %letterpaper
\advance\textheight-0.4cm
\calclayout

%Font settings
%\usepackage{anyfontsize}
%Footnote settings
%\usepackage{mathptmx}
\usepackage{footmisc}
%	\renewcommand*{\thefootnote}{\fnsymbol{footnote}}
\usepackage{commath}
%Further math environments
%Further math fonts (loads amsfonts implicitely)
\usepackage{amssymb}
%Redefinition of \text
%\usepackage{amstext}
\usepackage{upref}
%Graphics
%\usepackage{graphicx}
%\usepackage{caption}
%\usepackage{subcaption}
%Frames
\usepackage{mdframed}
\allowdisplaybreaks
%\usepackage{interval}
\newcommand{\toup}{%
  \mathrel{\nonscript\mkern-1.2mu\mkern1.2mu{\uparrow}}%
}
\newcommand{\todown}{%
  \mathrel{\nonscript\mkern-1.2mu\mkern1.2mu{\downarrow}}%
}
\AtBeginDocument{\renewcommand*\d{\mathop{}\!\mathrm{d}}}
\renewcommand{\Re}{\operatorname{Re}}
\renewcommand{\Im}{\operatorname{Im}}
\DeclareMathOperator\Log{Log}
\DeclareMathOperator\Arg{Arg}
\DeclareMathOperator\sech{sech}
\DeclareMathOperator*\esssup{ess.sup}
%\usepackage{hhline}
%\usepackage{booktabs} 
%\usepackage{array}
%\usepackage{xfrac} 
%\everymath{\displaystyle}
%Enumerate
\usepackage{tikz}
\usetikzlibrary{external}
\tikzexternalize % activate!
\usetikzlibrary{patterns}
\pgfdeclarepatternformonly{adjusted lines}{\pgfqpoint{-1pt}{-1pt}}{\pgfqpoint{40pt}{40pt}}{\pgfqpoint{39pt}{39pt}}%
{
  \pgfsetlinewidth{.8pt}
  \pgfpathmoveto{\pgfqpoint{0pt}{0pt}}
  \pgfpathlineto{\pgfqpoint{39.1pt}{39.1pt}}
  \pgfusepath{stroke}
}
\usepackage{enumitem} 
%\renewcommand{\labelitemi}{$\bullet$}
%\renewcommand{\labelitemii}{$\ast$}
%\renewcommand{\labelitemiii}{$\cdot$}
%\renewcommand{\labelitemiv}{$\circ$}
%Colors
%\usepackage{color}
%\usepackage[cmtip, all]{xy}
%Theorems
\newtheoremstyle{bold}              	 %Name
  {}                                     %Space above
  {}                                     %Space below
  {\itshape}		                     %Body font
  {}                                     %Indent amount
  {\scshape}                             %Theorem head font
  {.}                                    %Punctuation after theorem head
  { }                                    %Space after theorem head, ' ', 
  										 %	or \newline
  {} 
\theoremstyle{bold}
\newtheorem*{definition*}{Definition}
\newtheorem{definition}{Definition}[section]
\newtheorem*{lemma*}{Lemma}
\newtheorem{lemma}{Lemma}[section]
\newtheorem{Proof}{Proof}[section]
\newtheorem{proposition}{Proposition}[section]
\newtheorem{properties}{Properties}[section]
\newtheorem{corollary}{Corollary}[section]
\newtheorem*{theorem*}{Theorem}
\newtheorem{theorem}{Theorem}[section]
\newtheorem{example}{Example}[section]
\newtheorem*{remark*}{Remark}
\newtheorem{remark}{Remark}[section]
%German non-ASCII-Characters
%Graphics-Tool
%\usepackage{tikz}
%\usepackage{tikzscale}
%\usepackage{bbm}
%\usepackage{bera}
%Listing-Setup
%Bibliographie
\usepackage[backend=bibtex, style=alphabetic]{biblatex}
%\usepackage[babel, german = swiss]{csquotes}
\bibliography{Bibliography}
%PDF-Linking
%\usepackage[hyphens]{url}
\usepackage[bookmarksopen=true,bookmarksnumbered=true]{hyperref}
%\PassOptionsToPackage{hyphens}{url}\usepackage{hyperref}
\hypersetup{
  colorlinks   = true, %Colours links instead of ugly boxes
  urlcolor     = blue, %Colour for external hyperlinks
  linkcolor    = blue, %Colour of internal links
  citecolor    = blue %Colour of citations
}
%Weierstrass-P symbol for power set
\newcommand{\powerset}{\raisebox{.15\baselineskip}{\Large\ensuremath{\wp}}}


\title{Solutions Sheet 7}
\author{Yannis B\"{a}hni}
\address[Yannis B\"{a}hni]{University of Zurich, R\"{a}mistrasse 71, 8006 Zurich}
\email[Yannis B\"{a}hni]{\href{mailto:yannis.baehni@uzh.ch}{yannis.baehni@uzh.ch}}

\begin{document}
\maketitle
\thispagestyle{fancy}
\begin{enumerate}[label = \textbf{Exercise \arabic*.},wide = 0pt, itemsep=1.5ex]
	\item
		We will abreviate $\mathbb{C}^- := \mathbb{C} \setminus \mathbb{R}_{\leq 0}$.
		\begin{enumerate}[label = (\alph*),wide = 0pt, itemsep=1.5ex]
			\item The set $\mathbb{C}^-$ is clearly a star shaped domain with possible centers on the ray $\mathbb{R}_{>0}$. Furtheremore, the function $1/z$ is holomorphic in $\mathbb{C}^-$ since it is a well-defined rational function there. By the Cauchy integral theorem for star shaped domains $f$ has a primitive $F: \mathbb{C}^- \to \mathbb{C}$ which is explicitely given by 
				\begin{equation}
					F(z) := \int_{\intcc{z_0,z}} \frac{\d\zeta}{\zeta}
				\end{equation}
				\noindent for any $z_0 \in \mathbb{R}_{>0}$. The choice $z_0 = 1$ yields
				\begin{equation}
					F(1) = \int_{\intcc{1,1}} \frac{\d\zeta}{\zeta} = 0
				\end{equation}
				\noindent since the path $\intcc{1,1}(t) = 1$, $t \in \intcc{0,1}$, is clearly closed (we have that $\intcc{1,1}(0) = 1 = \intcc{1,1}(1)$) and thus again the Cauchy integral theorem implies that the integral over any closed path vanishes. Hence the primitive $F$ of $1/z$ on $\mathbb{C}^-$ fulfilling $F(1) = 0$ is given by 
				\begin{equation}
					\boxed{F(z) = \int_{\intcc{1,z}} \frac{\d\zeta}{\zeta} \qquad z \in \mathbb{C}^-.}
				\end{equation}
			\item Let $z_0 \in \mathbb{C}^-$. Let $B_r(z_0)$ denote the largest ball around $z_0$ contained in $\mathbb{C}^-$. By the Cauchy-Taylor expansion theorem we have that 
				\begin{equation}
					F = \sum_{\nu = 0}^\infty a_\nu (z - z_0)^\nu \qquad a_\nu = \frac{F^{(\nu)}(z_0)}{\nu !}
				\end{equation}
				\noindent in $B_r(z_0)$ since $F$ is clearly holomorphic in $\mathbb{C}^-$ as a primitive. In order to calculate $F^{(\nu)}$, we have to compute $f^{(\nu)}$ since $F' = f$.
				\begin{lemma}
					Consider the function $f: \mathbb{C}^\times \to \mathbb{C}$ defined by $f(z) := 1/z$. Then
					\begin{equation}
						f^{(\nu)}(z_0) = (-1)^{\nu}\frac{\nu !}{z^{\nu + 1}_0} \qquad \nu \in \mathbb{N}_0, z_0 \in \mathbb{C}^-.
					\end{equation}
					\label{lem:derf}
				\end{lemma}

				\begin{proof}
					Proof by induction over $\nu \in \mathbb{N}_0$. For $\nu = 0$ the equation clearly holds. Assume it is true for some $\nu \in \mathbb{N}_0$. Then 
					\begin{equation}
						f^{(\nu + 1)}(z_0) = \del[1]{f^{(\nu)}}'(z_0) = (-1)^\nu \nu! (-(\nu + 1))\frac{1}{z_0^{\nu + 2}} = (-1)^{\nu + 1}\frac{(\nu + 1)!}{z_0^{\nu + 2}}.
					\end{equation}
				\end{proof}

				Since $F^{(\nu)}(z_0)/\nu! = f^{(\nu - 1)}(z_0)/\nu !$ for all $\nu \in \mathbb{N}$, lemma \ref{lem:derf} implies that 
				\begin{equation}
					\boxed{F = F(z_0) + \sum_{\nu = 1}^\infty\frac{(-1)^{\nu - 1}}{\nu}\frac{1}{z_0^\nu} (z - z_0)^\nu \qquad z \in B_r(z_0).}
				\end{equation}
				By
				\begin{equation}
					\limsup_{\nu \to \infty} \abs[3]{\frac{(-1)^{\nu - 1}}{\nu}\frac{1}{z_0^\nu}}^{1/\nu} = \frac{1}{\abs[0]{z_0}}\limsup_{\nu \to \infty}\frac{1}{\nu^{1/\nu}} = \frac{1}{\abs[0]{z_0}}\lim_{\nu \to \infty}\frac{1}{\nu^{1/\nu}} = \frac{1}{\abs[0]{z_0}}
				\end{equation}
				\noindent we see that $R = \abs[0]{z_0}$ using the Cauchy-Hadamard formula.
		\end{enumerate}
\end{enumerate}
%\originalsectionstyle
\printbibliography
\end{document}
