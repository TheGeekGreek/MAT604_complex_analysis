%%%%%%%%%%%%%%%%%%%%%%%%%%%%%%%%%%%%%%%%%%%%%%%%%%%%%%%%%%%%%%%%%%%%%%%%%%
%Author:																 %
%-------																 %
%Yannis Baehni at University of Zurich									 %
%baehni.yannis@uzh.ch													 %
%																		 %
%Version log:															 %
%------------															 %
%06/02/16 . Basic structure												 %
%04/08/16 . Layout changes including section, contents, abstract.		 %
%%%%%%%%%%%%%%%%%%%%%%%%%%%%%%%%%%%%%%%%%%%%%%%%%%%%%%%%%%%%%%%%%%%%%%%%%%

%Page Setup
\documentclass[
	11pt, 
	oneside, 
	a4paper,
	reqno,
	final
]{amsart}

\usepackage[
	left = 3cm, 
	right = 3cm, 
	top = 3cm, 
	bottom = 3cm
]{geometry}

%Headers and footers
\usepackage{fancyhdr}
	\pagestyle{fancy}
	%Clear fields
	\fancyhf{}
	%Header right
	\fancyhead[R]{
		\footnotesize
		Yannis B\"{a}hni\\
		\href{mailto:yannis.baehni@uzh.ch}{yannis.baehni@uzh.ch}
	}
	%Header left
	\fancyhead[L]{
		\footnotesize
		MAT604: Complex Analysis\\
		Spring Semester 2017
	}
	%Page numbering in footer
	\fancyfoot[C]{\thepage}
	%Separation line header and footer
	\renewcommand{\headrulewidth}{0.4pt}
	%\renewcommand{\footrulewidth}{0.4pt}
	
	\setlength{\headheight}{19pt} 

%Title
\usepackage[foot]{amsaddr}
%\usepackage{mathptmx}
\usepackage{xspace}
\makeatletter
\def\@textbottom{\vskip \z@ \@plus 1pt}
\let\@texttop\relax
\usepackage{etoolbox}
\patchcmd{\abstract}{\scshape\abstractname}{\textbf{\abstractname}}{}{}

%Switching commands for different section formats
%Mainsectionsytle
\newcommand{\mainsectionstyle}{%
  	\renewcommand{\@secnumfont}{\bfseries}
  	\renewcommand\section{\@startsection{section}{1}%
    	\z@{.5\linespacing\@plus.7\linespacing}{-.5em}%
    	{\normalfont\bfseries}}%
	\renewcommand\subsection{\@startsection{subsection}{2}%
    	\z@{.5\linespacing\@plus.7\linespacing}{-.5em}%
    	{\normalfont\bfseries}}%
	\renewcommand\subsubsection{\@startsection{subsubsection}{3}%
    	\z@{.5\linespacing\@plus.7\linespacing}{-.5em}%
    	{\normalfont\bfseries}}%
}
\newcommand{\originalsectionstyle}{%
\def\@secnumfont{\bfseries}%\mdseries
\def\section{\@startsection{section}{1}%
  \z@{.7\linespacing\@plus\linespacing}{.5\linespacing}%
  {\normalfont\bfseries\centering}}
}
%Formatting title of TOC
\renewcommand{\contentsnamefont}{\bfseries}
%Table of Contents
\setcounter{tocdepth}{3}

% Add bold to \section titles in ToC and remove . after numbers
\renewcommand{\tocsection}[3]{%
  \indentlabel{\@ifnotempty{#2}{\bfseries\ignorespaces#1 #2\quad}}\bfseries#3}
% Remove . after numbers in \subsection
\renewcommand{\tocsubsection}[3]{%
  \indentlabel{\@ifnotempty{#2}{\ignorespaces#1 #2\quad}}#3}
\let\tocsubsubsection\tocsubsection% Update for \subsubsection
%...

\newcommand\@dotsep{4.5}
\def\@tocline#1#2#3#4#5#6#7{\relax
  \ifnum #1>\c@tocdepth % then omit
  \else
    \par \addpenalty\@secpenalty\addvspace{#2}%
    \begingroup \hyphenpenalty\@M
    \@ifempty{#4}{%
      \@tempdima\csname r@tocindent\number#1\endcsname\relax
    }{%
      \@tempdima#4\relax
    }%
    \parindent\z@ \leftskip#3\relax \advance\leftskip\@tempdima\relax
    \rightskip\@pnumwidth plus1em \parfillskip-\@pnumwidth
    #5\leavevmode\hskip-\@tempdima{#6}\nobreak
    \leaders\hbox{$\m@th\mkern \@dotsep mu\hbox{.}\mkern \@dotsep mu$}\hfill
    \nobreak
    \hbox to\@pnumwidth{\@tocpagenum{\ifnum#1=1\bfseries\fi#7}}\par% <-- \bfseries for \section page
    \nobreak
    \endgroup
  \fi}
\AtBeginDocument{%
\expandafter\renewcommand\csname r@tocindent0\endcsname{0pt}
}
\def\l@subsection{\@tocline{2}{0pt}{2.5pc}{5pc}{}}
\def\l@subsubsection{\@tocline{2}{0pt}{4.5pc}{5pc}{}}
\makeatother

\advance\footskip0.4cm
\textheight=54pc    %a4paper
\textheight=50.5pc %letterpaper
\advance\textheight-0.4cm
\calclayout

%Font settings
%\usepackage{anyfontsize}
%Footnote settings
%\usepackage{mathptmx}
\usepackage{footmisc}
%	\renewcommand*{\thefootnote}{\fnsymbol{footnote}}
\usepackage{commath}
%Further math environments
%Further math fonts (loads amsfonts implicitely)
\usepackage{amssymb}
%Redefinition of \text
%\usepackage{amstext}
\usepackage{upref}
%Graphics
%\usepackage{graphicx}
%\usepackage{caption}
%\usepackage{subcaption}
%Frames
\usepackage{mdframed}
\allowdisplaybreaks
%\usepackage{interval}
\newcommand{\toup}{%
  \mathrel{\nonscript\mkern-1.2mu\mkern1.2mu{\uparrow}}%
}
\newcommand{\todown}{%
  \mathrel{\nonscript\mkern-1.2mu\mkern1.2mu{\downarrow}}%
}
\AtBeginDocument{\renewcommand*\d{\mathop{}\!\mathrm{d}}}
\renewcommand{\Re}{\operatorname{Re}}
\renewcommand{\Im}{\operatorname{Im}}
\DeclareMathOperator\Log{Log}
\DeclareMathOperator\Arg{Arg}
\DeclareMathOperator\sech{sech}
\DeclareMathOperator*\esssup{ess.sup}
%\usepackage{hhline}
%\usepackage{booktabs} 
%\usepackage{array}
%\usepackage{xfrac} 
%\everymath{\displaystyle}
%Enumerate
\usepackage{tikz}
\usetikzlibrary{external}
\tikzexternalize % activate!
\usetikzlibrary{patterns}
\pgfdeclarepatternformonly{adjusted lines}{\pgfqpoint{-1pt}{-1pt}}{\pgfqpoint{40pt}{40pt}}{\pgfqpoint{39pt}{39pt}}%
{
  \pgfsetlinewidth{.8pt}
  \pgfpathmoveto{\pgfqpoint{0pt}{0pt}}
  \pgfpathlineto{\pgfqpoint{39.1pt}{39.1pt}}
  \pgfusepath{stroke}
}
\usepackage{enumitem} 
%\renewcommand{\labelitemi}{$\bullet$}
%\renewcommand{\labelitemii}{$\ast$}
%\renewcommand{\labelitemiii}{$\cdot$}
%\renewcommand{\labelitemiv}{$\circ$}
%Colors
%\usepackage{color}
%\usepackage[cmtip, all]{xy}
%Theorems
\newtheoremstyle{bold}              	 %Name
  {}                                     %Space above
  {}                                     %Space below
  {\itshape}		                     %Body font
  {}                                     %Indent amount
  {\scshape}                             %Theorem head font
  {.}                                    %Punctuation after theorem head
  { }                                    %Space after theorem head, ' ', 
  										 %	or \newline
  {} 
\theoremstyle{bold}
\newtheorem*{definition*}{Definition}
\newtheorem{definition}{Definition}[section]
\newtheorem*{lemma*}{Lemma}
\newtheorem{lemma}{Lemma}[section]
\newtheorem{Proof}{Proof}[section]
\newtheorem{proposition}{Proposition}[section]
\newtheorem{properties}{Properties}[section]
\newtheorem{corollary}{Corollary}[section]
\newtheorem*{theorem*}{Theorem}
\newtheorem{theorem}{Theorem}[section]
\newtheorem{example}{Example}[section]
\newtheorem*{remark*}{Remark}
\newtheorem{remark}{Remark}[section]
%German non-ASCII-Characters
%Graphics-Tool
%\usepackage{tikz}
%\usepackage{tikzscale}
%\usepackage{bbm}
%\usepackage{bera}
%Listing-Setup
%Bibliographie
\usepackage[backend=bibtex, style=alphabetic]{biblatex}
%\usepackage[babel, german = swiss]{csquotes}
\bibliography{Bibliography}
%PDF-Linking
%\usepackage[hyphens]{url}
\usepackage[bookmarksopen=true,bookmarksnumbered=true]{hyperref}
%\PassOptionsToPackage{hyphens}{url}\usepackage{hyperref}
\hypersetup{
  colorlinks   = true, %Colours links instead of ugly boxes
  urlcolor     = blue, %Colour for external hyperlinks
  linkcolor    = blue, %Colour of internal links
  citecolor    = blue %Colour of citations
}
%Weierstrass-P symbol for power set
\newcommand{\powerset}{\raisebox{.15\baselineskip}{\Large\ensuremath{\wp}}}


\title{Solutions Sheet 3}
\author{Yannis B\"{a}hni}
\address[Yannis B\"{a}hni]{University of Zurich, R\"{a}mistrasse 71, 8006 Zurich}
\email[Yannis B\"{a}hni]{\href{mailto:yannis.baehni@uzh.ch}{yannis.baehni@uzh.ch}}

\begin{document}
\maketitle
\thispagestyle{fancy}
\textbf{Remark:} We will abreviate $\mathbb{C}^\times := \mathbb{C} \setminus \cbr[0]{0}$.
\begin{enumerate}[label = \textbf{Exercise \arabic*.},wide = 0pt, itemsep=1.5ex]
	\item
		~
		\begin{enumerate}[label = (\alph*),wide = 0pt, itemsep=1.5ex]
			\item We use the quotient criterion \cite[100]{remmert2002funktionentheorie}. Consider the sequence
				\begin{equation}
					a_\nu := \frac{\nu!}{2^\nu (2\nu)!} \qquad \nu \in \mathbb{N}_{> 0}.
				\end{equation}

				Clearly $a_\nu \neq 0$ for all $\nu \in \mathbb{N}_{>0}$. We have 
				\begin{align*}
					\lim_{\nu \to \infty} \frac{\abs[0]{a_\nu}}{\abs[0]{a_{\nu + 1}}} &= \lim_{\nu \to \infty} \frac{\nu!}{2^\nu(2\nu)!}\frac{2^{\nu + 1}(2(\nu + 1))!}{(\nu + 1)!}\\
					&= \lim_{\nu \to \infty} \frac{2(2\nu + 1)(2\nu + 2)}{\nu + 1}\\
					&= 4\lim_{\nu \to \infty}(2\nu + 1)\\
					&= \infty
				\end{align*}

				Thus by
				\begin{equation}
					\lim_{\nu \to \infty} \frac{\abs[0]{a_\nu}}{\abs[0]{a_{\nu + 1}}} = \limsup_{\nu \to \infty} \frac{\abs[0]{a_\nu}}{\abs[0]{a_{\nu + 1}}} = \liminf_{\nu \to \infty} \frac{\abs[0]{a_\nu}}{\abs[0]{a_{\nu + 1}}}
				\end{equation}

				\noindent and 
				\begin{equation}
					 \liminf_{\nu \to \infty} \frac{\abs[0]{a_\nu}}{\abs[0]{a_{\nu + 1}}} \leq R \leq \limsup_{\nu \to \infty} \frac{\abs[0]{a_\nu}}{\abs[0]{a_{\nu + 1}}}
				\end{equation}

				\noindent we get
				\begin{equation}
					R = \infty.
				\end{equation}

			\item We use the Cauchy-Hadamard formula \cite[99]{remmert2002funktionentheorie}. Let $c \in C^\times$. Consider the sequence
				\begin{align*}
					a_\nu := \begin{cases}
						\frac{1}{c^{\nu/2}}& \nu \equiv 0 \bmod 2\\
						0 & \nu \equiv 1 \bmod 2
					\end{cases}
				\end{align*}

				Since $0 \leq 1/c^{\nu/2}$ we get
				\begin{align*}
					\limsup_{\nu \to \infty} \abs[0]{a_\nu}^{1/\nu} &= \lim_{\nu \to \infty}\sup_{\mu \geq \nu} \cbr[0]{\abs[0]{a_\mu}^{1/\mu}}\\
					&= \lim_{\nu \to \infty}\sup_{\substack{\mu \geq \nu\\\mu \equiv 0 \bmod 2}}\cbr[0]{\abs[0]{a_\mu}^{1/\mu}}\\
					&= \limsup_{\nu \to \infty} \abs[3]{\frac{1}{c^{\nu/2}}}^{1/\nu}\\
					&= \limsup_{\nu \to \infty} \frac{1}{\sqrt{\abs[0]{c}}}\\
					&= \lim_{\nu \to \infty} \frac{1}{\sqrt{\abs[0]{c}}}\\
					&= \frac{1}{\sqrt{\abs[0]{c}}}
				\end{align*}

				\noindent and therefore
				\begin{equation}
					R = \frac{1}{\limsup_{\nu \to \infty} \abs[0]{a_\nu}^{1/\nu}} = \sqrt{\abs[0]{c}}.
				\end{equation}
		\end{enumerate}

	\item Proof by induction over $k \in \mathbb{N}$. If $k = 0$ we have 
		\begin{equation}
			\sum_{\nu = 0}^\infty {\nu \choose 0} z^\nu = \sum_{\nu = 0}^\infty z^\nu = \frac{1}{1 - k}
		\end{equation}
		\noindent by the well-known identity for the geometric series (see \cite[24]{remmert2002funktionentheorie}) which holds for all $z \in \mathbb{E}$. Now assume the stated identity is true for some $k \in \mathbb{N}$. Pascal's identity and the substitution $\mu := \nu - 1$ yields
	\begin{align}
		\sum_{\nu = k + 1}^\infty {\nu \choose {k + 1}} z^{\nu - (k + 1)} &= 1 + \sum_{\nu = k + 2}^\infty {\nu \choose {k + 1}} z^{(\nu - 1) - k }\nonumber\\
		&= 1 +\sum_{\nu = k + 2}^\infty \sbr[4]{{{\nu - 1} \choose k } + {{\nu - 1} \choose {k + 1}}} z^{(\nu - 1) - k }\nonumber\\
		&= 1 + \sum_{\nu = k + 2}^\infty {{\nu - 1} \choose k }z^{(\nu - 1) - k} + \sum_{\nu = k + 2}^\infty{{\nu - 1} \choose {k + 1}} z^{(\nu - 1) - k }\label{eq:ind_step}\\
		&= {k \choose k}z^0 + \sum_{\mu = k + 1}^\infty {\mu \choose k }z^{\mu - k} + \sum_{\mu = k + 1}^\infty{\mu \choose {k + 1}} z^{\mu - k }\nonumber\\
		&= \sum_{\mu = k}^\infty {\mu \choose k }z^{\mu - k} + z\sum_{\mu = k + 1}^\infty{\mu \choose {k + 1}} z^{\mu - k - 1}\nonumber
	\end{align}

	Applying the induction hypothesis on (\ref{eq:ind_step}) and rearranging yields 
	\begin{equation}
		\sum_{\nu = k + 1}^\infty {\nu \choose {k + 1}} z^{\nu - (k + 1)} = \frac{1}{(1 - z)(1 - z)^{k + 1}} = \frac{1}{(1 - z)^{k + 2}}.	
	\end{equation}

	Therefore we conclude by the principle of induction.

	A more elegant proof can be deduced from the interchangeability of differentiation and summation of power series (see \cite[110]{remmert2002funktionentheorie}).

	\begin{lemma}
		For $z \in \mathbb{E}$ and $k \in \mathbb{N}$ we have 
		\begin{equation}
			\frac{\d^k}{\d z^k} \frac{1}{1 - z} = \frac{k!}{(1 - z)^{k + 1}}.
		\end{equation}
		\label{lem:ind}
	\end{lemma}

	\begin{proof}
		Proof by induction over $k \in \mathbb{N}$. The statement obviously holds for $k = 0$. Assume the statement holds for $k \in \mathbb{N}$. Then we get
		\begin{align*}
			\frac{\d^{k+1}}{\d z^{k+1}} \frac{1}{1 - z} &= \frac{\d}{\d z} \sbr[3]{\frac{\d^k}{\d z^k} \frac{1}{1 - z}}\\
			&= \frac{\d}{\d z} \frac{k!}{(1 - z)^{k + 1}}\\
			&= k!\frac{(k + 1)(1 - z)^k}{(1 - z)^{2k + 2}}\\
			&= \frac{(k + 1)!}{(1 - z)^{k + 2}}.
		\end{align*}
	\end{proof}

	By lemma \ref{lem:ind} and the formula for the $k$-th derivative of a power series \cite[110]{remmert2002funktionentheorie} applied to the \emph{geometric series} $\sum_{\nu = 0}^\infty z^\nu$ we get
	\begin{equation}
		\frac{k!}{(1 - z)^{k + 1}} = \frac{\d^k}{\d z^k} \frac{1}{1 - z} = \sum_{\nu = k}^\infty k!{\nu \choose k}z^{\nu - k}
		\label{eq:der}
	\end{equation}

	\noindent for $k \in \mathbb{N}$. Dividing (\ref{eq:der}) by $k!$ yields
	\begin{equation}
		\sum_{\nu = k}^\infty {\nu \choose k}z^{\nu - k} = \frac{1}{(1 - z)^{k + 1}}.
	\end{equation}
\item
	~
	\begin{enumerate}[label = (\alph*),wide = 0pt, itemsep=1.5ex]
		\item Consider the auxiliary function $\varphi: \mathbb{C} \to \mathbb{C}$ defined by
			\begin{equation}
				\varphi(z) := f(z)\exp(-z).
			\end{equation}
		Clearly $\varphi \in \mathcal{O}(\mathbb{C})$ as a product of holomorphic functions. Furthermore by $f' = f$ we get 
	\begin{equation}
		\varphi'(z) = f'(z)\exp(-z) - f(z)\exp(-z) = \varphi(z) - \varphi(z) = 0
		\label{eq:const}
	\end{equation}

	\noindent for any $z \in \mathbb{C}$. By \cite[55]{remmert2002funktionentheorie} $\varphi$ is locally constant on $\mathbb{C}$ and since $\mathbb{C}$ is connected we have that $\varphi$ is constant on $\mathbb{C}$ (see \cite[35]{remmert2002funktionentheorie}). Since $f(0) = 1$ we have
	\begin{equation}
		\varphi(0) = f(0)\exp(0) = 1.
	\end{equation}

	Thus $\varphi \equiv 1$ which yields
	\begin{equation}
		\exp(z) = 1\cdot \exp(z) = \varphi(z)\exp(z) = f(z)\exp(-z)\exp(z) = f(z)
	\end{equation}

	\noindent for all $z \in \mathbb{C}$.
\item Differentiation of the addition formula with respect to $w$ yields
	\begin{equation}
		g'(z + w) = g(z)g'(w).
		\label{eq:diff}
	\end{equation}

	\noindent for all $w,z \in \mathbb{C}$. Evaluating (\ref{eq:diff}) at $w = 0$ gives
	\begin{equation}
		g'(z) = g(z)g'(0).
	\end{equation}

	Define $\psi: \mathbb{C} \to \mathbb{C}$ by
	\begin{equation}
		\psi(z) := g(z)\exp(-g'(0)z).
	\end{equation}

	Clearly $\psi \in \mathcal{O}(\mathbb{C})$ as a product of holomorphic functions. Similar to part (a) we get
	\begin{align*}
		\psi'(z) &= g'(z)\exp(-g'(0)z) - g(z)g'(0)\exp(-g'(0)z)\\
		&= g(z)g'(0)\exp(-g'(0)z) - g(z)g'(0)\exp(-g'(0)z)\\
		&= 0	
	\end{align*}
	\noindent for all $z \in \mathbb{C}$. Again $\psi$ is constant on $\mathbb{C}$ and with 
	\begin{equation}
		g(0) = g(0 + 0) = g(0)g(0)
	\end{equation}

	\noindent immediately follows $g(0) = 1$ by $g(0) \neq 0$. Thus $\psi(0) = 1$ and hence $\psi \equiv 1$ on $\mathbb{C}$ which implies
	\begin{equation}
		g(z) = \exp(g'(0)z) = \exp(bz)
	\end{equation}

	\noindent for $b := g'(0)$.
	\end{enumerate}

\item
	~
	\begin{enumerate}[label = (\alph*),wide = 0pt, itemsep=1.5ex]
		\item 
			~
			\begin{proposition}
				For all $w,z \in \mathbb{C}$ holds: 
				\begin{align}
					&\cos(w + z) = \cos w \cos z - \sin w \sin z\label{eq:cos}\\
					&\sin(w + z) = \sin w \cos z + \cos w \sin z\label{eq:sin}.
				\end{align}
			\end{proposition}

			\begin{proof}
				First we prove (\ref{eq:cos}). We have 
				\begin{align*}
					\cos w \cos z &= \frac{1}{4}\del[0]{e^{iw} + e^{-iw}}\del[0]{e^{iz} + e^{-iz}}\\ &= \frac{1}{4}\del[0]{e^{i(w + z)} + e^{i(w - z)} + e^{i(z - w)} + e^{-i(w + z)}}
				\end{align*}

				\noindent and
				\begin{align*}
					\sin w \sin z &= -\frac{1}{4}\del[0]{e^{iw} - e^{-iw}}\del[0]{e^{iz} - e^{-iz}}\\ &= -\frac{1}{4}\del[0]{e^{i(w + z)} - e^{i(w - z)} - e^{i(z - w)} + e^{-i(w + z)}}
				\end{align*}

				Thus
				\begin{equation}
					\cos w \cos z - \sin w \sin z = \frac{1}{2}(e^{i(w + z)} + e^{-i(w + z)}) = \cos(w + z).
				\end{equation}

				Now we prove (\ref{eq:sin}). We have 
				\begin{align*}
					\sin w \cos z &= \frac{1}{4i}\del[0]{e^{iw} - e^{-iw}}\del[0]{e^{iz} + e^{-iz}}\\ &= \frac{1}{4i}\del[0]{e^{i(w + z)} + e^{i(w - z)} - e^{i(z - w)} - e^{-i(w + z)}}
				\end{align*}

				\noindent and
				\begin{align*}
					\cos w \sin z &= \frac{1}{4i}\del[0]{e^{iw} + e^{-iw}}\del[0]{e^{iz} - e^{-iz}}\\ &= \frac{1}{4i}\del[0]{e^{i(w + z)} - e^{i(w - z)} + e^{i(z - w)} - e^{-i(w + z)}}
				\end{align*}

				Thus
				\begin{equation}
					\sin w \cos z + \cos w \sin z = \frac{1}{2i}\del[0]{e^{i(w + z)} - e^{-i(w + z)}} = \sin(w + z). 
				\end{equation}
			\end{proof}

		\item

		\item 
	\end{enumerate}
\end{enumerate}
%\originalsectionstyle
\printbibliography
\end{document}
