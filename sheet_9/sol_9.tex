%%%%%%%%%%%%%%%%%%%%%%%%%%%%%%%%%%%%%%%%%%%%%%%%%%%%%%%%%%%%%%%%%%%%%%%%%%
%Author:																 %
%-------																 %
%Yannis Baehni at University of Zurich									 %
%baehni.yannis@uzh.ch													 %
%																		 %
%Version log:															 %
%------------															 %
%06/02/16 . Basic structure												 %
%04/08/16 . Layout changes including section, contents, abstract.		 %
%%%%%%%%%%%%%%%%%%%%%%%%%%%%%%%%%%%%%%%%%%%%%%%%%%%%%%%%%%%%%%%%%%%%%%%%%%

%Page Setup
\documentclass[
	11pt, 
	oneside, 
	a4paper,
	reqno,
	final
]{amsart}

\usepackage[
	left = 3cm, 
	right = 3cm, 
	top = 3cm, 
	bottom = 3cm
]{geometry}

%Headers and footers
\usepackage{fancyhdr}
	\pagestyle{fancy}
	%Clear fields
	\fancyhf{}
	%Header right
	\fancyhead[R]{
		\footnotesize
		Yannis B\"{a}hni\\
		\href{mailto:yannis.baehni@uzh.ch}{yannis.baehni@uzh.ch}
	}
	%Header left
	\fancyhead[L]{
		\footnotesize
		MAT604: Complex Analysis\\
		Spring Semester 2017
	}
	%Page numbering in footer
	\fancyfoot[C]{\thepage}
	%Separation line header and footer
	\renewcommand{\headrulewidth}{0.4pt}
	%\renewcommand{\footrulewidth}{0.4pt}
	
	\setlength{\headheight}{19pt} 

%Title
\usepackage[foot]{amsaddr}
%\usepackage{mathptmx}
\usepackage{xspace}
\makeatletter
\def\@textbottom{\vskip \z@ \@plus 1pt}
\let\@texttop\relax
\usepackage{etoolbox}
\patchcmd{\abstract}{\scshape\abstractname}{\textbf{\abstractname}}{}{}

%Switching commands for different section formats
%Mainsectionsytle
\newcommand{\mainsectionstyle}{%
  	\renewcommand{\@secnumfont}{\bfseries}
  	\renewcommand\section{\@startsection{section}{1}%
    	\z@{.5\linespacing\@plus.7\linespacing}{-.5em}%
    	{\normalfont\bfseries}}%
	\renewcommand\subsection{\@startsection{subsection}{2}%
    	\z@{.5\linespacing\@plus.7\linespacing}{-.5em}%
    	{\normalfont\bfseries}}%
	\renewcommand\subsubsection{\@startsection{subsubsection}{3}%
    	\z@{.5\linespacing\@plus.7\linespacing}{-.5em}%
    	{\normalfont\bfseries}}%
}
\newcommand{\originalsectionstyle}{%
\def\@secnumfont{\bfseries}%\mdseries
\def\section{\@startsection{section}{1}%
  \z@{.7\linespacing\@plus\linespacing}{.5\linespacing}%
  {\normalfont\bfseries\centering}}
}
%Formatting title of TOC
\renewcommand{\contentsnamefont}{\bfseries}
%Table of Contents
\setcounter{tocdepth}{3}

% Add bold to \section titles in ToC and remove . after numbers
\renewcommand{\tocsection}[3]{%
  \indentlabel{\@ifnotempty{#2}{\bfseries\ignorespaces#1 #2\quad}}\bfseries#3}
% Remove . after numbers in \subsection
\renewcommand{\tocsubsection}[3]{%
  \indentlabel{\@ifnotempty{#2}{\ignorespaces#1 #2\quad}}#3}
\let\tocsubsubsection\tocsubsection% Update for \subsubsection
%...

\newcommand\@dotsep{4.5}
\def\@tocline#1#2#3#4#5#6#7{\relax
  \ifnum #1>\c@tocdepth % then omit
  \else
    \par \addpenalty\@secpenalty\addvspace{#2}%
    \begingroup \hyphenpenalty\@M
    \@ifempty{#4}{%
      \@tempdima\csname r@tocindent\number#1\endcsname\relax
    }{%
      \@tempdima#4\relax
    }%
    \parindent\z@ \leftskip#3\relax \advance\leftskip\@tempdima\relax
    \rightskip\@pnumwidth plus1em \parfillskip-\@pnumwidth
    #5\leavevmode\hskip-\@tempdima{#6}\nobreak
    \leaders\hbox{$\m@th\mkern \@dotsep mu\hbox{.}\mkern \@dotsep mu$}\hfill
    \nobreak
    \hbox to\@pnumwidth{\@tocpagenum{\ifnum#1=1\bfseries\fi#7}}\par% <-- \bfseries for \section page
    \nobreak
    \endgroup
  \fi}
\AtBeginDocument{%
\expandafter\renewcommand\csname r@tocindent0\endcsname{0pt}
}
\def\l@subsection{\@tocline{2}{0pt}{2.5pc}{5pc}{}}
\def\l@subsubsection{\@tocline{2}{0pt}{4.5pc}{5pc}{}}
\makeatother

\advance\footskip0.4cm
\textheight=54pc    %a4paper
\textheight=50.5pc %letterpaper
\advance\textheight-0.4cm
\calclayout

%Font settings
%\usepackage{anyfontsize}
%Footnote settings
%\usepackage{mathptmx}
\usepackage{footmisc}
%	\renewcommand*{\thefootnote}{\fnsymbol{footnote}}
\usepackage{commath}
%Further math environments
%Further math fonts (loads amsfonts implicitely)
\usepackage{amssymb}
%Redefinition of \text
%\usepackage{amstext}
\usepackage{upref}
%Graphics
%\usepackage{graphicx}
%\usepackage{caption}
%\usepackage{subcaption}
%Frames
\usepackage{mdframed}
\allowdisplaybreaks
%\usepackage{interval}
\newcommand{\toup}{%
  \mathrel{\nonscript\mkern-1.2mu\mkern1.2mu{\uparrow}}%
}
\newcommand{\todown}{%
  \mathrel{\nonscript\mkern-1.2mu\mkern1.2mu{\downarrow}}%
}
\AtBeginDocument{\renewcommand*\d{\mathop{}\!\mathrm{d}}}
\renewcommand{\Re}{\operatorname{Re}}
\renewcommand{\Im}{\operatorname{Im}}
\DeclareMathOperator\Log{Log}
\DeclareMathOperator\Arg{Arg}
\DeclareMathOperator\sech{sech}
\DeclareMathOperator*\esssup{ess.sup}
%\usepackage{hhline}
%\usepackage{booktabs} 
%\usepackage{array}
%\usepackage{xfrac} 
%\everymath{\displaystyle}
%Enumerate
\usepackage{tikz}
\usetikzlibrary{external}
\tikzexternalize % activate!
\usetikzlibrary{patterns}
\pgfdeclarepatternformonly{adjusted lines}{\pgfqpoint{-1pt}{-1pt}}{\pgfqpoint{40pt}{40pt}}{\pgfqpoint{39pt}{39pt}}%
{
  \pgfsetlinewidth{.8pt}
  \pgfpathmoveto{\pgfqpoint{0pt}{0pt}}
  \pgfpathlineto{\pgfqpoint{39.1pt}{39.1pt}}
  \pgfusepath{stroke}
}
\usepackage{enumitem} 
%\renewcommand{\labelitemi}{$\bullet$}
%\renewcommand{\labelitemii}{$\ast$}
%\renewcommand{\labelitemiii}{$\cdot$}
%\renewcommand{\labelitemiv}{$\circ$}
%Colors
%\usepackage{color}
%\usepackage[cmtip, all]{xy}
%Theorems
\newtheoremstyle{bold}              	 %Name
  {}                                     %Space above
  {}                                     %Space below
  {\itshape}		                     %Body font
  {}                                     %Indent amount
  {\scshape}                             %Theorem head font
  {.}                                    %Punctuation after theorem head
  { }                                    %Space after theorem head, ' ', 
  										 %	or \newline
  {} 
\theoremstyle{bold}
\newtheorem*{definition*}{Definition}
\newtheorem{definition}{Definition}[section]
\newtheorem*{lemma*}{Lemma}
\newtheorem{lemma}{Lemma}[section]
\newtheorem{Proof}{Proof}[section]
\newtheorem{proposition}{Proposition}[section]
\newtheorem{properties}{Properties}[section]
\newtheorem{corollary}{Corollary}[section]
\newtheorem*{theorem*}{Theorem}
\newtheorem{theorem}{Theorem}[section]
\newtheorem{example}{Example}[section]
\newtheorem*{remark*}{Remark}
\newtheorem{remark}{Remark}[section]
%German non-ASCII-Characters
%Graphics-Tool
%\usepackage{tikz}
%\usepackage{tikzscale}
%\usepackage{bbm}
%\usepackage{bera}
%Listing-Setup
%Bibliographie
\usepackage[backend=bibtex, style=alphabetic]{biblatex}
%\usepackage[babel, german = swiss]{csquotes}
\bibliography{Bibliography}
%PDF-Linking
%\usepackage[hyphens]{url}
\usepackage[bookmarksopen=true,bookmarksnumbered=true]{hyperref}
%\PassOptionsToPackage{hyphens}{url}\usepackage{hyperref}
\hypersetup{
  colorlinks   = true, %Colours links instead of ugly boxes
  urlcolor     = blue, %Colour for external hyperlinks
  linkcolor    = blue, %Colour of internal links
  citecolor    = blue %Colour of citations
}
%Weierstrass-P symbol for power set
\newcommand{\powerset}{\raisebox{.15\baselineskip}{\Large\ensuremath{\wp}}}


\title{Solutions Sheet 9}
\author{Yannis B\"{a}hni}
\address[Yannis B\"{a}hni]{University of Zurich, R\"{a}mistrasse 71, 8006 Zurich}
\email[Yannis B\"{a}hni]{\href{mailto:yannis.baehni@uzh.ch}{yannis.baehni@uzh.ch}}

\begin{document}
\maketitle
\thispagestyle{fancy}
\begin{enumerate}[label = \textbf{Exercise \arabic*.},wide = 0pt, itemsep=1.5ex]
	\item Let $\Gamma := \sbr[0]{z_1,z_2} + \sbr[0]{z_2,z_3} + \sbr[0]{z_3,z_1}$. $\Gamma$ is a cycle since it is the positively oriented boundary chain of the non-empty domain $\Delta^\circ$ (example $1$ in Fischer/Lieb). Now by Satz $1.3$ we have that $n(\Gamma,z)$ is locally constant for $z \in \mathbb{C} \setminus \abs[0]{\Gamma}$ and $0$ for $z$ in the unbounded pathcomponent. Thus we get 
	\begin{equation*}
		n(\Gamma,z) = 0 \qquad z \in \mathbb{C} \setminus \Delta.
	\end{equation*}
	By the wall-crossing lemma (Satz $3.1$) we get furthermore 
	\begin{equation*}
		n(\Gamma,z) = 1 \qquad z \in \Delta^\circ
	\end{equation*}
	\noindent since by assumption every line segment is not a singleton, hence we find a ball around a point in the line segment and we can apply Satz $3.1$ which just yields
	\begin{equation*}
		n(\Gamma,z_1) = n(\Gamma,z_2) + 1 = 1 \qquad z_1 \in \Delta^\circ, z_2 \in \mathbb{C} \setminus \Delta.
	\end{equation*}
 	\item See separate sheet.
	\item
	~
	\begin{enumerate}[label = (\roman{*}),wide = 0pt, itemsep=1.5ex]
		\item Let $f \in \mathcal{O}(G_1 \cup G_2)$ (clearly $G_1 \cup G_2$ is open and connected since it is path-connected by the non-empty intersection). This implies $f \in \mathcal{O}(G_1)$ and $f \in \mathcal{O}(G_2)$. Since $G_1$ and $G_2$ are elementary domains, there exist primitives $F_1: G_1 \to \mathbb{C}$ and $F_2: G_2 \to \mathbb{C}$ of $f$. Consider the auxiliary function $\varphi: G_1 \cap G_2 \to \mathbb{C}$ defined by $\varphi(z) := F_1(z) - F_2(z)$. This is clearly not the empty function since $G_1 \cap G_2 \neq \varnothing$ by assumption. Now
		\begin{equation*}
			\varphi'(z) = F_1'(z) - F_2'(z) = f(z) - f(z) = 0
		\end{equation*}
		\noindent for all $z \in G_1 \cap G_2$ implies that $\varphi$ is locally constant on $G_1 \cap G_2$ and thus by connectedness, $\varphi$ is constant on $G_1 \cap G_2$. Hence there exists $\lambda \in \mathbb{C}$ such that $F_1(z) = F_2(z) + \lambda$ for all $z \in G_1 \cap G_2$. Thus
		\begin{align*}
			F(z) := \begin{cases}
						F_1(z) & z \in G_1\\
						F_2(z) + \lambda & z \in G_2
					\end{cases}
		\end{align*}
		\noindent is a well defined primitive of $f$ in $G_1 \cup G_2$.
		\item Consider $G_1 := \mathbb{C} \setminus \mathbb{R}_{\leq 0}$ and $G_2 := \mathbb{C} \setminus \mathbb{R}_{\geq 0}$. Clearly, $G_1$ and $G_2$ are elementary domains, since they are star-shaped domains. Now $G_1 \cap G_2 = \mathbb{C} \setminus \mathbb{R} \neq \varnothing$ which is not connected since 
		\begin{equation*}
			\mathbb{C} \setminus \mathbb{R} = \cbr[0]{z \in \mathbb{C} : \Im(z) < 0} \cup \cbr[0]{z \in \mathbb{C} : \Im(z) > 0}.
		\end{equation*}
		Furthermore, $G_1 \cup G_2 = \mathbb{C} \setminus \cbr[0]{0}$. But $\mathbb{C} \setminus \cbr[0]{0}$ is clearly not an elementary domain since $f(z):= \frac{1}{z}$ does not have a primitive there since
		\begin{equation*}
			\int_{\partial \mathbb{E}} f(\zeta) \d \zeta = 2\pi i. 
		\end{equation*}
		Hence the assumption that $G_1 \cap G_2 \neq \varnothing$ is connected is necessary.
		\item
	\end{enumerate}
	\item 
	\begin{definition}
		Let $G \subseteq \mathbb{C}^\times$ be a domain. A function $\varphi \in \mathscr{C}(G;\mathbb{R})$ is said to be a \bld{branch of the argument}, if $z = \abs[0]{z}e^{i \varphi(z)}$ for all $z \in G$.
	\end{definition}

	\begin{proposition}
		Let $G \subseteq \mathbb{C}^\times$ be a domain. There exists a branch of the argument on $G$ if and only if there exists a branch of the logarithm on $G$. 
	\end{proposition}

	\begin{proof}
		Assume that there exists a branch of the argument $\varphi$. Thus for all $z \in G$ we have
		\begin{equation*}
			z = \abs[0]{z}e^{i \varphi(z)} = e^{\log\abs[0]{z} + i \varphi(z)}
		\end{equation*}
	\noindent and by the continuity of $\varphi$, $f(z) := \log\abs[0]{z} + i \varphi(z)$ is a branch of the logarithm. Conversly, by
	\begin{equation*}
		z = e^{f(z)} = e^{\Re f}e^{i \Im f} = \abs[0]{z} e^{i\Im f}
	\end{equation*}
	\noindent for all $z \in G$ we have that $\Im f$ is a branch of the argument on $G$, since $f$ is continuous and so is $\Im f$.
	\end{proof}
\end{enumerate}
%\originalsectionstyle
\printbibliography
\end{document}
